% !TEX root = ../Coherence.tex

\section{Perspectives}

%%%%%%%%%%%%%%%%%%%%%%%%%%%%%%%%%%%%%

\subsection{Further applications} 
\label{sec:further}
One can also use the same strategy to prove coherence for \emph{unital} non-symmetric monoidal categories, using the unital associahedra of F. Muro and A. Tonks \cite{muroUnitalAssociahedra2014}.

It is natural to ask if the construction of unital associahedra could be extended to the permutoassociahedra, in such a way as to provide a topological proof of coherence for unital symmetric monoidal categories. 
The question of the existence of these constructions at the operadic level (i.e. does there exist unital operahedra, symmetric operahedra, and unital symmetric operahedra?) is, to our knowledge, still open as well. 

Another immediate application of \cref{thm:top-coherence} is the coherence of strong non-symmetric monoidal functors between non-symmetric monoidal categories \cite{epsteinFunctorsTensoredCategories1966}. 
The corresponding topological objects are in this case the family of multiplihedra \cite{Stasheff70,Forcey08}.
The generalization to strong morphisms between non-symmetric categorified operads also goes through, involving this time the family of multiploperahedra described at the end of the introduction in \cite{MazuirLA22}.

In the same spirit as in \cref{thm:coherence-operahedra}, one could obtain coherence results for categorifications of many operad-like structures, for instance the ones described in \cite{BMO20}: categorified modular operads, wheeled properads, and permutads (shuffle algebras), among others.
In order to treat cyclic and symmetric structures, one could take inspiration from the reduction process followed in \cite{curienCategorifiedCyclicOperads2020} for the case of cyclic symmetric categorified operads.

\subsection{Higher categories} 
\label{sec:higher}

\correction{\cref{thm:top-coherence} shows the precise relationship between coherence and connectedness.
In addition to Kapranov's claim \cite{kapranov1993}, it clarifies other statements in the literature, such as the proof of \cite[Prop.~3.9]{KapranovVoevodsky94}.
There, the incipit ``since $P_n$ is a convex polytope'' could be replaced by a more precise ``since $P_n$ is simply connected''.}

In the case of (symmetric) monoidal categories, \cref{thm:top-coherence} demonstrates that coherence is equivalent to the vanishing of the first homotopy groups of the (permuto-)associahedra. 
Since the (permuto-)associahedra are contractible, and therefore all their homotopy groups vanish, one could hope for a topological proof of higher dimensional coherence theorems.

\correction{One dimension higher, N. Gurski has shown in \cite[Thms.~22 \& 23]{Gurski11} that coherence for (braided) monoidal bicategories is equivalent to the vanishing of fundamental $2$-groupoids of braid groups.}
Recent results of S. Barkan provide evidence for higher dimensional statements, \correction{relating coherence diagrams of $\infty$-operads to the connectivity of certain operadic partition complexes} \cite{barkanArityApproximationInfty2022}.
It seems likely that the present results could be interpreted as a strict version and a special case of \cite[Thm.~B]{barkanArityApproximationInfty2022}. 
It would be interesting to see how the permuto-associahedra arise in the strictification process, and how they are related to operadic partition complexes.  

%Note that already at the strict level of (braided) monoidal $2$-categories, one needs pasting scheme structures on $3$-dimensional polytopes \cite{KapranovVoevodsky94,KapranovVoevodsky94b}.
%To define combinatorially higher dimensional structures, one thus needs pasting scheme structures on polytopes, \cite{LMP24}


