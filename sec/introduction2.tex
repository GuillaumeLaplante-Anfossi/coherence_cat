% !TEX root = ../Coherence2.tex

\section*{Introduction} 
\label{s:introduction}



\subsection*{From rewriting to coherence}

In his seminal notes~\cite{Huet-notes-cat} for a graduate course at Université Paris 7, Gérard Huet explained Mac Lane's proof of the coherence theorem for monoidal categories through the lenses of equational reasoning and  term rewriting theory. 
Huet remarked that Mac Lane's pentagons can be read as local confluence diagrams, and that
\begin{enumerate}
\item proving  the coherence statement in the case of canonical natural transformations $\lambda:F\rightarrow G$, where $\lambda$ is defined using the associator only and where $G$ is a normal form, amounts to annotating the proof of Newman's lemma with  explicit names for the rewriting steps, 
\item the proof of the general case of the coherence theorem mimicks the proof of the Church-Rosser property, which states that if two terms $P,Q$ can be proved equal in the equational theory obtained by forgetting the orientation of the rewriting rules, then there is some $N$ such that $P\rightarrow\ldots\rightarrow N$ and
$Q\rightarrow\ldots\rightarrow N$.
\end{enumerate}
Moreover, in order to check local confluence, it is enough to check local confluence of {\em critical pairs}, which are minimal situations in which $M\rightarrow P$ and $M\rightarrow Q$ and the respective subterms of $M$ to which the two reductions are applied overlap. Huet observed that Mac Lane's pentagon expresses the unique critical pair of  the rewriting system given by the associator.

\subsection*{Coherence and polytopes}

In a previous paper~\cite{CLA1}, we discussed combinatorial topological proofs of coherence theorems.
In particular, we gave a topological proof of Mac Lane's coherence theorem by using the fact that all diagrams involved live on the $2$-skeleton of a family of polytopes, the associahedra. 
Here, 
\begin{itemize}
\item[(0)] 0-cells correspond to functors, 
\item[(1)] paths in the 1-skeleton correspond to natural transformations,  
\item[(2)] coherence conditions in situation correspond to 2-faces,  
\end{itemize}
and the coherence statement amounts to asking whether any two parallel cellular can be related by repeatedly replacing a portion of a path fitting on the boundary of a $2$-cell by the complementary path on that same boundary.
In fact, our topological/combinatorial results can be applied to give ``one step proofs'' (quoting Kapranov~\cite{kapranov1993}) of a number of other categorical coherence theorems. 

\subsection*{Rewriting on nestohedra}

It is therefore natural to ask if there is a general way to associate a term rewriting system to a polytope, yielding the above coherence results via Huet's correspondence for different families of interest. 
In this paper, we give a positive answer to this question for the family of hypergraph polytopes, a.k.a nestohedra. 
We construct term rewriting systems on the vertices and faces of hypergraph polytopes (\cref{ss:rewriting-constructs,ss:rewriting-constructions}), and show that the former are confluent (\cref{thm:confluent}). 
We characterize their critical pairs as certain types of $2$-faces (\cref{thm:critical-pairs}).
The rewrite rules on the vertices generalize Barnard--McConville's \emph{flip order} on the vertices of graph-associahedra \cite{Barnard-McConville}, and are induced by an orientation vector (\cref{Tamari-orientation-vector}).
Meanwhile, the rewrite rules on the faces seem to generalize the \emph{facial weak order} on the faces of permutahedra \cite{KrobLatapyNovelliPhanSchwer,PalaciosRonco,DermenjianHohlwegPilaud}.

To further study coherence, we restrict our attention to subfamilies of nestohedra that we call \emph{contextual} (\cref{def:contextual-family}).
These include associahedra, permutahedra and operahedra (\cref{thm:examples}), whose term rewriting systems provide, via Huet's correspondence, coherence theorems for monoidal categories, categorified permutads and operads, respectively. 
The idea behind the notion of contextual nestohedra is to require local confluence diagrams, which correspond to $2$-faces of the polytopes, to satisfy a certain uniformity condition, allowing the distinction between ``coherence condition'' and ``coherence condition in situation'' to be preserved.
The observation that this distinction was lost when using topological methods \cite{CLA1} is what initially prompted the present study.  

%Hypergraph polytopes are \emph{generalized permutahedra}~\cite{P09}. 

\subsection*{Notations}

We use $\prec$ to denote cover relations in a poset, and $|-|$ to denote the cardinality of a set.

