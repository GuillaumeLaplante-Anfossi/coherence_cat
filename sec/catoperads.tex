% !TEX root = ../Coherence.tex

\section{Coherence for categorified operads} 
\label{s:catoperads}

\subsection{Categorified operads}

We can now prove a coherence theorem for non-symmetric non-unital categorified operads. 

\begin{thm}
\label{t:coherence-operahedra}
    Every diagram with vertices iterates of the $\circ_i$ and edges expansions of instances of $\beta, \theta, \lambda$ and $\rho$ arrows is commutative. 
\end{thm}

\begin{proof}
    Follows \cref{t:polytopal-coherence} and \cref{c:operahedra-2-coh}.
\end{proof}

\subsection{Weak Cat-operads}

There is an analogous statement for weak Cat-operads \cite[Proposition 14.2]{DP15}. In the same fashion as for \cref{thm:equivalenceDPGLA}, one can prove that the two statements are equivalent. 

\begin{definition}[Weak Cat-operad {\cite{DP15}}]
\end{definition}

\begin{thm} \label{thm:equivalenceDPGLA}
    The data of a categorified ns operad and the data of a weak Cat-operad are equivalent.  
\end{thm}

\begin{proof}
    TBC
\end{proof}


\subsection{MacLane's theorem}

We can now deduce the classical coherence theorem of MacLane for monoidal categories. 

\begin{corollary}
    For any two terms $M,M':T\rightarrow T'$  of the same type we have $\dl M\dr^\rho_{\cat{C}}=\dl M'\dr^\rho_{\cat{C}}$.
\end{corollary}

\begin{proof}
    This is just resricting \cref{t:coherence-operahedra} to the case of a categorified non-symmetric operad with one object, which is just a non-symmetric non-unital monoidal category. 
    Equivalently, it can be obtained by reworking the proof of \cref{t:coherence-operahedra} while restricting the attention to the associahedra, and use again \cref{t:polytopal-coherence} and \cref{c:operahedra-2-coh}.
\end{proof}

\subsubsection{The symmetric case}

\subsubsection{The unital case}

Unital AND symmetric?

\subsection{Perspectives}

Multiplihedra/multiploperahedra
Operadic families

\Guillaume{$n$-coherence related to $n$-categories instead of simply categories??!! A commencer par les 2-cat; le theoreme de coherence ne tient plus pour les n plus grands!!}


The same can be applied to 


