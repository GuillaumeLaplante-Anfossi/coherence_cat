% !TEX root = ../Coherence.tex

\section{Rewriting method for coherence} 
\label{s:catoperads}







\subsection{Coherence for categorified operads}

Recall from [REF] the definition of the $\mathbb{N}$-colored operad $\mathcal{O}$ encoding ns operads. Its minimal resolution is given by the cellular chains on the operahedra [DEF]. 


\begin{thm}[Coherence theorem] \label{thm:coherence}
    Every diagram with vertices iterates of the $\circ_i$ and edges expansions of instances of $\beta, \theta, \lambda$ and $\rho$ arrows is commutative. 
\end{thm}

\begin{proof} TBC
\end{proof}

Restricted to categorified ns operad concentrated in arity 1, i.e. to monoidal categories, we recover MacLane's original coherence tbeorem [REF].

There is an analogous statement for weak Cat-operads \cite[Proposition 14.2]{DP15}. In the same fashion as for \cref{thm:equivalenceDPGLA}, one can prove that the two statements are equivalent. 


\subsection{Weak Cat-operads}

\begin{definition}[Weak Cat-operad {\cite{DP15}}]
\end{definition}

\begin{thm} \label{thm:equivalenceDPGLA}
    The data of a categorified ns operad and the data of a weak Cat-operad are equivalent.  
\end{thm}

\begin{proof}
    TBC
\end{proof}



