% !TEX root = ../Coherence.tex

\section*{Introduction} 
\label{s:introduction}

The $n$-dimensional permuto-associahedron, a CW-complex whose faces are in bijection with parenthesized \correction{ordered partitions} of $n+1$ letters, was first introduced by M. Kapranov in his study of higher dimensional Yang--Baxter equations, through the moduli spaces of curves $\overline{\mathcal{M}_{0,n+1}}(\R)$ and the solutions of the Knizhnik--Zamolodchikov equation \cite{kapranov1993}.
It was later realized as a convex polytope by V. Reiner and G. M. Ziegler \cite{reinerCoxeterassociahedra1994}, and more recently through the nested braid fan in \cite{CastilloLiu21}.
%as a simple polytope in \cite{baralicSimplePermutoassociahedron2019} and .

The present note stems from a desire to understand the epigraph, taken from the introduction of \cite{kapranov1993}: what is the precise relationship between the permuto-associahedron and Mac Lane's coherence theorem for symmetric monoidal categories? 
We show that the \emph{simple connectedness} of the former implies the latter, thereby refining and proving Kapranov's claim (see \cref{thm:coherence-MacLane}).

This is done through a general ``topological coherence theorem" which applies to any simply connected, regular CW complex (\cref{thm:top-coherence}).
\correction{We apply this theorem to another family of polytopes, the operahedra, which encode categorified non-symmetric operads~\cite{DP15,curienSyntacticAspectsHypergraph2019a,laplante-anfossiDiagonalOperahedra2022a}.
We obtain a ``one-step proof'' of coherence (\cref{thm:coherence-operahedra}) which simplifies the original proof of Do{\v s}en and Petri{\'c}~\cite{DP15}.}

We also investigate a topological incarnation of Mac Lane's original argument, in the spirit of rewriting theory. 
Using Morse \correction{functions} on affine cell complexes \cite{bestvinaMorseTheoryFiniteness1997}, \correction{we define terminating and confluent rewriting system on their $1$-skeleton and} obtain a general topological theorem which applies to a certain family of simply connected polyhedral complexes; the ones \correction{whose outgoing links are all connected} (\cref{p:second-proof}). 
In particular, this second theorem can be applied to polytopes, allowing us to give a second, ``rewriting-theoretic'' proof of both previously mentioned coherence results (see \cref{ss:abstract-rewriting}). 
\correction{It is worth noting that while general polyhedral complexes admit \emph{abstract} rewriting systems on their sets of vertices, the specific families of polytopes studied in \cref{s:catoperads} admit \emph{term} rewriting systems, which exhibit more structure and are the subject of a companion paper~\cite{CLA24}.}

Our two general topological coherence theorems can be used to prove other categorical results where polytopes appear, such as coherence for monoidal functors between monoidal categories \cite{epsteinFunctorsTensoredCategories1966}, see \cref{sec:further}.
\correction{They also shed light on some statements in the literature, such as the proof of \cite[Prop.~3.9]{KapranovVoevodsky94}.}
This all points towards further investigation of the relationship between $n$-categorical coherence and $n$-connectedness of appropriate spaces.
\correction{While topological proofs of $2$-categorical coherence already appeared in \cite{Gurski11}, higher dimensional results have been obtained recently by S. Barkan in the context of $\infty$-operads~\cite{barkanArityApproximationInfty2022}}, for which the present results could well be the strict, $n=1$ case.

