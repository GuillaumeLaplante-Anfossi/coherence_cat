% !TEX root = ../Coherence.tex

\section*{Introduction} 
\label{s:introduction}

The present note stems from a desire to understand the epigraph, a claim of M. Kapranov in \cite{kapranov1993}. 
The $n$-dimensional permutassociahedron, a CW complex whose cells are in bijection with parenthesized permutations of a word with $n+1$ letters, was introduced there and later realized as a convex polytope by V. Reiner and G. M. Ziegler \cite{reinerCoxeterassociahedra1994}. 
A new realization was found recently in \cite{CastilloLiu21}, and a simple version has been defined in \cite{baralicSimplePermutoassociahedron2019}.
In both of these works, the epigraph above is restated without proof. 
The present note fills that gap, clarifying Kapranov's statement: in fact, only the \emph{simple connectedness} of the permutoassociahedron is required (\cref{thm:MacLane}).
This is done through a general ``topological coherence theorem" which applies to any regular CW complex (\cref{thm:top-coherence}).
We use it to prove an operadic generalization of MacLane's coherence theorem (\cref{thm:coherence-operahedra}), simplifying a result of K. Do{\v s}en and Z. Petri{\'c} \cite{DP15}.
We present other similar applications in \cref{sec:further}. 
We also present an alternative proof for polyhedral complexes (\cref{p:second-proof}), giving a topological argument which follows closely MacLane's original one, in the spirit of rewriting theory. 
Morse theoretic considerations show that this proof is much less general, but still sufficient for many applications.
The main takeaway of this note is that $1$-categorical coherence is tied, in many cases, to $1$-connectedness of a family of spaces. 
This encourages further study of the relation between strict $n$-categorical coherence and $n$-connectedness of appropriate spaces, in the same vein as recent work of S. Barkan in the $\infty$-categorical setting \cite{barkanArityApproximationInfty2022}, see \cref{sec:higher}.



