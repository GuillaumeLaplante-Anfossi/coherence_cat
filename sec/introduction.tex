% !TEX root = ../Coherence.tex

\section*{Introduction} 
\label{s:introduction}

The present note stems from a desire to understand the epigraph, a claim of M. Kapranov in \cite{kapranov1993}. 
The $n$-dimensional permutassociahedron, a CW complex whose cells are in bijection with parenthesized permutations of a word with $n+1$ letters, was introduced there and later realized as a convex polytope by V. Reiner and G. M. Ziegler \cite{reinerCoxeterassociahedra1994}. 
A new realization was found recently in \cite{CastilloLiu21}, and a simple version has been defined in \cite{baralicSimplePermutoassociahedron2019}.
In both of these works, the epigraph above is restated without proof. 
The present paper fills that gap (\cref{cor:MacLane}).
Moreover, this is done through a general "topological coherence theorem" which applies to any regular CW complex (\cref{thm:top-coherence}). 
We use it to prove an operadic generalization of MacLane's coherence theorem (\cref{thm:coherence-operahedra}), and present other similar applications in \cref{sec:further}. 
The main takeaway of this note it that $1$-categorical coherence is tied, in many cases, to $1$-connectedness of a family of spaces. 
This encourages further study of the relation between $n$-categorical coherence and $n$-connectedness of appropriate spaces (see \cref{sec:higher}).



