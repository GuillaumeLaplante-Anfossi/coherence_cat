% !TEX root = ../Coherence.tex

\section*{Introduction} 
\label{s:introduction}

The $n$-dimensional permuto-associahedron, a CW-complex whose faces are in bijection with parenthesized \correction{ordered partitions} of $n+1$ letters, was first introduced by M. Kapranov in his study of higher dimensional Yang--Baxter equations, through the moduli spaces of curves $\overline{\mathcal{M}_{0,n+1}}(\R)$ and the solutions of the Knizhnik--Zamolodchikov equation \cite{kapranov1993}.
It was later realized as a convex polytope by V. Reiner and G. M. Ziegler \cite{reinerCoxeterassociahedra1994}, and more recently through the nested braid fan by F. Castillo and F. Liu in \cite{CastilloLiu21}.

The present study stems from a desire to understand the epigraph, taken from the introduction of \cite{kapranov1993}: what is the precise relationship between the permuto-associahedron and Mac Lane's coherence theorem for symmetric monoidal categories? 
We show that the \emph{simple connectedness} of the former implies the latter, thereby refining and proving Kapranov's claim (see \cref{thm:coherence-MacLane}). 

This is done through a general ``topological coherence theorem" which applies to any simply connected, regular CW complex (\cref{thm:top-coherence}).
\correction{Applying it to the operahedra, another family of polytopes which encodes categorified non-symmetric operads~\cite{DP15,curienSyntacticAspectsHypergraph2019a,laplante-anfossiDiagonalOperahedra2022a}, we obtain a ``one-step'' proof of the associated coherence theorem as well.}

\smallskip
\correction{There is little price to pay, though. 
For both theorems, one needs to provide a precise bijective correspondence between the 1-skeleton (resp.\ the 2--cells) on the topological side, and canonical morphisms (resp.\ bifunctoriality, naturality, and applications of coherence conditions) on the categorical side (\cref{{bijections-Kapranov},prop:bijection-nestings}). 
Since the 2-skeleton of the permuto-associahedra corresponds to other basic canonical morphisms and coherence conditions than those of Mac Lane (hexagons and naturality of the involutive braiding on one hand versus dodecagons on the other hand), one needs to show that the two presentations are equivalent, which is non-trivial, see~\cref{rem:Kapranov-to-MacLane}.  
There is yet a third equivalent presentation (and hence another proof of coherence) due to D. Barali\'c, J. Ivanovi\'c and D. Petri\'c~\cite{baralicSimplePermutoassociahedron2019}, that matches the 2-skeleton of a different polytope, which unlike the permuto-associahedron is simple, see \cref{rem:simple-permutoassociahedron}.
}

\smallskip
We further investigate a topological incarnation of Mac Lane's original argument, in the spirit of rewriting theory.  
\correction{We study polyhedral complexes endowed with a generic orientation vector, or equivalently a Morse function in the sense of \cite{bestvinaMorseTheoryFiniteness1997}, whose $1$-skeletons naturally feature terminating and confluent rewriting systems (\cref{prop:terminating-confluent}).}
\correction{We focus on the family of simply connected polyhedral complexes whose outgoing links are connected. 
The study of directed paths on their $1$-skeleton leads to a second general proof of coherence (\cref{p:second-proof}).} 
In particular, this second theorem can be applied to all polytopes, allowing us to give a second, ``rewriting-theoretic'' proof of both previously mentioned coherence results.  
\correction{
In the case of operahedra, our rewriting proof simplifies the original proof of Do{\v s}en and Petri{\'c}~\cite{DP15}, see~\cref{rem:DPLA}.}

\correction{It is worth noting that, while the above polyhedral complexes admit \emph{abstract} rewriting systems on their $1$-skeleton,
the family of operahedra (which includes the associahedra, encoding non-symmetric monoidal categories) further admits \emph{term} rewriting systems, which exhibit more structure and are the subject of a companion paper~\cite{CLA24}. In contrast, we shall argue that the abstract rewriting approach to \emph{symmetric} monoidal categories is not informative, see \cref{MacLane-Kapranov-Simple}.}

\correction{Using Morse \correction{theory} on affine cell complexes \cite{bestvinaMorseTheoryFiniteness1997},
 we relate our two approaches by showing that the second is (strictly) less general than the first (\cref{lemma:outgoing-link}).}
 
 \smallskip
Our two general topological coherence theorems can be used to prove other categorical results where polytopes appear, such as coherence for monoidal functors between monoidal categories \cite{epsteinFunctorsTensoredCategories1966}, see \cref{sec:further}.
\correction{They also shed light on some statements in the literature, such as the proof of \cite[Prop.~3.9]{KapranovVoevodsky94}, see \cref{sec:higher}.}
This all points towards further investigation of the relationship between $n$-categorical coherence and $n$-connectedness of appropriate spaces.
\correction{While topological proofs of $2$-categorical coherence already appeared in \cite{Gurski11}, higher dimensional results have been obtained recently by S. Barkan in the context of $\infty$-operads~\cite{barkanArityApproximationInfty2022}}, for which the present results could well be the strict, $n=1$ case.

