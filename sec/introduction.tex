% !TEX root = ../Coherence.tex

\section*{Introduction} 
\label{s:introduction}

The $n$-dimensional permutoassociahedron, a CW-complex whose faces are in bijection with parenthesized permutations of $n+1$ letters, was first introduced by M. Kapranov in his study of higher dimensional Yang--Baxter equations, through the moduli spaces of curves $\overline{\mathcal{M}_{0,n+1}}(\R)$ and the solutions of the Knizhnik--Zamolodchikov equation \cite{kapranov1993}.
It was later realized as a convex polytope by V. Reiner and G. M. Ziegler \cite{reinerCoxeterassociahedra1994}, and more recently as a simple polytope in \cite{baralicSimplePermutoassociahedron2019} and through the nested braid fan in \cite{CastilloLiu21}.

The present note stems from a desire to understand the epigraph, taken from the introduction of \cite{kapranov1993}: what is the precise relationship between the permutoassociahedron and MacLane's coherence theorem for symmetric monoidal categories? 
We show that the \emph{simple connectedness} of the former implies the latter, thereby refining and proving Kapranov's conjecture (see \cref{thm:MacLane}).

This is done through a general ``topological coherence theorem" which applies to any simply connected, regular CW complex (\cref{thm:top-coherence}).
We use it to prove coherence result for categorified non-symmetric operads, symplifying the proof of \cite{DP15}. 
These objects, introduced in \cite{DP15}, are an operadic generalization of non-symmetric monoidal categories. 

We also investigate a topological incarnation of MacLane's original argument, in the spirit of rewriting theory. 
Using Morse theory on affine cell complexes \cite{bestvinaMorseTheoryFiniteness1997}, we obtain a general topological theorem which applies to a certain family of simply connected polyhedral complexes; the ones that admit a terminating and confluent rewriting system on their $1$-skeleton (\cref{p:second-proof}). 
In particular, this second theorem can be applied to polytopes, allowing us to give a second, rewriting-theoretic proof of both previously mentioned coherence results. 

These two general topological coherence theorems can be used to prove other categorical results where polytopes appear, such as coherence for monoidal functors between monoidal categories \cite{epsteinFunctorsTensoredCategories1966}, which we present in \cref{sec:further}.
This all points towards further investigation of the relationship between $n$-categorical coherence and $n$-connectedness of appropriate spaces.
As a motivation in this direction, the present results could well be the strict, $n=1$ case of recent theorems of S. Barkan in the $\infty$-categorical setting \cite{barkanArityApproximationInfty2022}, see \cref{sec:higher}.

