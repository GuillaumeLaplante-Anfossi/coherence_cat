% !TEX root = ../Coherence2.tex

\section{Hypergraphic rewrite systems} 
\label{s:rewriting}

We associate to each hypergraph a term rewrite system given by its constructs.

%%%%%%%%%%%%%%%%%%%%%%%%%%%%%%%%%%%%%%%

\subsection{Definition}

A \defn{signature} $\Sigma$ is a tuple $(V,F,S,\ari,\sort,\arisort)$ made of 
\begin{itemize}
  \item a set $V$ of \defn{variables},
  \item a non-empty set $F$ of \defn{function symbols}, and
  \item a set $S$ of \defn{sorts},
\end{itemize}
together with an \defn{arity}, \defn{sort} and \defn{arity-sort} functions
\begin{itemize}
  \item $\ari : F \to \mathbb{N}$,
  \item $\sort : F \cup V \to S$,
  \item $\arisort : F \to \prod_{n \geq 0} S^{n}$,
\end{itemize}
such that for $f \in F$, we have $\arisort(f) \in S^{\ari(f)}$. 
The $i$th component of $\arisort(f)$ is denoted $\arisort(f,i)$.
The set $\Ter(\Sigma)$ of \defn{terms} over a signature $\Sigma$ is defined inductively as follows. 
\begin{enumerate}
  \item If $t \in V$ is a variable, then $t$ is a term.
  \item If $f \in F$ is an arity $n$ function symbol, and $t_1,\ldots,t_n$ are terms such that $\sort(t_i)=\arisort(f,i)$, then $f(t_1,\ldots,t_n)$ is a term, and $\sort(f(t_1,\ldots,t_n)):=\sort(f)$.
\end{enumerate}
For a term $t \in \Ter(\Sigma)$, its set of \defn{variables} is defined as 
\begin{equation*}
  \var(t) := 
  \begin{cases}
    \{t\} & \text{ if } t \in V, \\
    \bigcup_{1 \leq i \leq n}\var(t_i) & \text{ if } t=f(t_1,\ldots,t_n).
  \end{cases}
\end{equation*}
A \defn{rewrite rule} over $\Sigma$ is an ordered pair $(l,r)$ of terms in $\Ter(\Sigma)$, denoted $l \to r$, such that
\begin{enumerate}
  \item the first term $l$ is not a variable, that is $l \notin V$.
  \item the variables of the second term are already in the first term, that is $\var(r) \subseteq \var(l)$.
\end{enumerate}

\begin{definition}
  A (many-sorted) \defn{term rewrite system} is a pair $(\Sigma,R)$ made of a signature and a set of rewrite rules $R$ over $\Sigma$.
\end{definition}

Let $\hyper{H}$ be a connected hypergraph. 
Consider the following \defn{signature} $\Sigma_\hyper{H}$ of $\hyper{H}$,  
\begin{itemize}
  \item variables are pairs $(X,Y)$ of non-empty subsets $X \subseteq Y \subseteq H$,
  \item function symbols are non-empty subsets $Y \subseteq H$,
  \item sorts are non-empty subsets $Y \subseteq H$,
  \item $\ari(Y)$ is the number of connected components of $\hyper{H}\setminus Y$,
  \item $\sort(X,Y):=Y$, $\sort(Y):=Y$.
  \item $\arisort(Y,i):=H_i$
\end{itemize}

\begin{lemma}
  There is a bijection between the set of terms of sort $H$ over $\Sigma_\hyper{H}$ and the set of constructs of $\hyper{H}$.
\end{lemma}

\begin{proof}
  
\end{proof}

contextuel c'est $\hyper{H}_{\cap X}=(\hyper{H}_E)_{\cap X}$ pour tout $X$ de taille $3$ et tout sous-ensemble connexe $E$ de cardinalite plus grande ou egale a $3$.
%contextuel c'est par rapport à la cohérence! Pas au système de réécriture

%je veux que les termes soient les constructs

%instantiation 
%in context 
%critical pair

%Est-ce qu'on n'est pas en train de montrer que c'est une opérade colorée...

