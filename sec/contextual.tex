% !TEX root = ../Coherence2.tex

\section{Contextual nestohedra} 
\label{s:contextual}

We define special families of hypergraph polytopes which we call ``contextual'', and exhibit several examples.

%%%%%%%%%%%%%%%%%%%%%%%%%%%%%%%%%%%%%%%

\subsection{Definition}

For a 3-element subset $X=\set{x_1,x_2,x_3}$ of $H$, we say that a 2-face $T$ of $\hyper{H}$ is an \defn{$X$-face} if its unique non-singleton node is decorated by $X$.  
If $X$ is the root of $T$, the following  lemma allows us to see $T=X(\ldots)$ as an ``instantiation'' of $X$, viewed as the maximum face of $\recrestr{\hyper{H}}{X}$.

\begin{lemma} 
  \label{instance-construct} 
  If $\hyper{H}$ is a connected hypergraph, if $X$ is a subset of $H$ such that $|X|=3$ and $T$ is a 2-dimensional construct with root $X$, then the poset of faces of $T$ is isomorphic to the poset of faces of $\recrestr{\hyper{H}}{X}$.
\end{lemma}

\begin{proof}
In Section~\ref{ss:type-B}, we have described up to permutation all the possible connected hypergraphs on the set $X$ of vertices and their respective posets of faces. 
We will treat the case where ${\cal A}(\recrestr{\hyper{H}}{X})$ is the poset of (c), the others are similar.
Pick the 0-face~$S:= x_1(x_2,x_3)$. 
We map $S$ to a $0$-face $\phi(S)$ of $T$ as follows. 
Let $\hyper{H},\set{x}\leadsto H_1,\ldots,H_n$. 
By \cref{xyz-reconnected}, we have $\xyz{x_1}{\hyper{H}}{\set{x_2},\set{x_3}}$, hence we have, say $x_2\in H_1$ and $x_3\in H_2$. Then let
  $\hyper{H_1},\set{x_2}\leadsto H_{1,1},\ldots,H_{1,p}$ and $\hyper{H_2},\set{x_3}\leadsto H_{2,1},\ldots,H_{2,q}$.
  Then we have $\hyper{H},X\leadsto H_{1,1},\ldots,H_{1,p},H_{2,1},\ldots,H_{2,q},H_3,\ldots H_n$, so that $T$ writes as
  $$T= X(T_{1,1},\ldots,T_{1,p},T_{2,1},\ldots,T_{2,q},T_3,\ldots T_n).$$ 
All these data determine uniquely a 0-dimensional subface of $T$, namely
  $$\phi(S)=x_1(x_2(T_{1,1},\ldots,T_{1,p}),x_3(T_{2,1},\ldots,T_{2,q}),T_3,\ldots T_n)$$
 and one recovers $S$
  by  pruning in $\phi(S)$ all nodes except those decorated by subsets of $X$.
  The same applies to all other 0-dimensional (resp. 1-dimensional) faces of $\recrestr{\hyper{H}}{X}$, establishing $\phi$ as a bijection, which is also easily seen to be monotonic: for example, we have
 $$\phi(\set{x_1,x_2}(x_3))= \set{x_1,x_2}(x_3(T_{2,1},\ldots,T_{2,q}),T_{1,1},\ldots,T_{1,p},T_3,\ldots T_n),$$
 evidencing $\phi(x_1(x_2,x_3))\preceq \phi(\set{x_1,x_2}(x_3))$. 
 The inverse of $\phi$ is also monotonic, since the above pruning does not affect the place where the contraction occurs -- e.g., the edge of $\phi(x_1(x_2,x_3))$ that is contracted to get $\phi(\set{x_1,x_2}(x_3))$ is the edge between $x_1$ and $x_2$, which can thus be contracted in the preimage $x_1(x_2,x_3)$ to yield the preimage $\set{x_1,x_2}(x_3)$.
 Finally, we set
 $$\phi(\set{x_1,x_2,x_3})=\set{x_1,x_2,x_3}(T_{1,1},\ldots,T_{1,p},T_{2,1},\ldots,T_{2,q},T_3,\ldots T_n)=T.$$
 \end{proof}

Now, if $T'$ is another $X$-face and $\occ{T'}{X}\neq T'$, then we would like to see  $T'$ as $\occ{T'}{X}$ in context, and hence $T'$ as ``$X$ in situation''.  
For this to hold, $T'$ should be of the same form as $T$. 
Setting $Y:=\supp(\occ{T}{X})$, we have that the poset of faces of $T'$ is isomorphic to the poset of faces of $\occ{T'}{X}$, which is a construct of $\hyper{H}_Y$ and has $X$ as root, so is in turn isomorphic to the poset of faces of $\recrestr{\hyper{(\hyper{H}_Y)}}{X}$ (\cref{instance-construct}).
Is it always the case that $\recrestr{(\hyper{H}_Y)}{X}=\recrestr{\hyper{H}}{X}$ for~$Y$ connected in $\hyper{H}$? 
The following examples give a negative answer.

\begin{example} \label{non-contextual-1}
Consider the hypergraph 
\[
  \hyper{H}:= \set{\set{x},\set{y},\set{z},\set{u},\set{x,y,z}, \set{x,u,z}},
  \]
the set $X:=\set{x,y,z}$ and the two $X$-faces $S:=u(X)$ and $T:=X(u)$. 
Then $\occ{S}{X}$ is a construct of $\hyper{K}:=\restrH{H}{\set{u}}$ while $\occ{T}{X}=T$ is a construct of $\hyper{H}$.
But we have $\xyz{y}{\hyper{K}}{\set{x}\!,\!\set{z}}$ while $\xyz{y}{\hyper{H}}{\set{x,z}}$, and $S$ is a triangle while $T$ is a quadrilateral, as
$\recrestr{\hyper{K}}{\set{x,y,z}} = \hyper{K}  =  \set{\set{x},\set{y},\set{z},\set{x,y,z}}$ and $\recrestr{\hyper{H}}{\set{x,y,z}}  =  \set{\set{x},\set{y},\set{z},\set{u},\set{x,z}\set{x,y,z}}$.
\end{example}
  
\begin{example} 
  \label{non-contextual-2}
Consider the graph 
$$\set{\set{x},\set{y},\set{z},\set{u},\set{x,y}, \set{y,z}, \set{x,u}, \set{u,z}}$$
Then exactly the same data as in Example \ref{non-contextual-1} provide evidence that this graph, whose realisation is the three-dimensional cyclohedron, is not contextual. 
\end{example}

This motivates the following definition.

\begin{definition} 
A connected hypergraph $\hyper{H}$ is \defn{contextual} if for all connected subsets $Y\inc H$ of cardinal $|Y|\geq 3$, and for all $3$-elements subsets $X=\{x,y,z\} \subseteq Y$, we have 
$$\begin{array}{lll}
  \xyz{x}{{\hyper{H}_Y}}{\set{y,z}} & \Leftrightarrow & \xyz{x}{\hyper{H}}{\set{y,z}}.
  \end{array}$$
\end{definition}

\begin{lemma} \label{context-lemma}
  A connected hypergraph $\hyper{H}$ is contextual if for all connected subsets $Y \subseteq H$ of cardinal $|Y|\geq 3$, and for all subsets $X \subseteq Y$ of cardinal $|X|=3$, we have
  $$\hyper{H}_{\cap X} = (\hyper{H}_Y)_{\cap X}.$$ 
\end{lemma}

\begin{proof} 
  This is a direct consequence of Lemma~\ref{xyz-reconnected}.
\end{proof}

\begin{proposition} \label{situation-construct}
Let $\hyper{H}$ be a contextual hypergraph.
If $X$ is a subset of $H$ such that $|X|=3$ and $T$ is an $X$-face of $\hyper{H}$, then the poset of faces of $T$ is isomorphic to the poset of faces of~$\recrestr{\hyper{H}}{X}$.
\end{proposition}

\begin{proof} 
  Let $\hyper{K}:=\hyper{H}_{\supp(S)}$ where $S:=\occ{T}{X}$. 
  By \cref{subconstruct-restriction}, $S$ is a constuct of $\hyper{K}$. By definition of the face relation, and since the only non-singleton (and hence ``splittable'') node of $T$ is $X$, we have that the poset of subfaces of $T$ is isomorphic to the poset of subfaces of $S$, which by \Cref{instance-construct} is isomorphic to ${\cal A}(\recrestr{\hyper{K}}{X})$, which is isomorphic to ${\cal A}(\recrestr{\hyper{H}}{X})$ since~$\hyper{H}$ is contextual.
\end{proof}

Proposition~\ref{situation-construct} allows us to see all $X$-faces as ``instantiations in context'' of $\recrestr{\hyper{H}}{X}$, which therefore acts  as a rule or axiom in the terminology of equational theories.


%%%%%%%%%%%%%%%%%%%%%%%%%%%%%%%%%%%%%%%%%%%%%%%%%%%%%%%

\subsection{Contextual families}

Motivated by the examples presented in \cref{ss:hypergraph-polytopes,ss:examples} and their associated categorical coherence theorems listed in \cref{table:contextual-hyper}, we define now the notion of a contextual \emph{family} of nestohedra.

Identifying an hypergraph $\hyper{H}$ with the maximal construct $T$ of $({\cal A}(\hyper{H}),\preceq)$, we say that $\hyper{H}$ has \defn{dimension} $\dim T$.
For a family of hypergraphs $\calH$, we denote by $\calH(n)$ the subset of hypergraphs of dimension $n \geq 0$.

We will consider families of ordered hypergraphs. 
Note that when $\hyper{H}$ is ordered, all the restrictions $\hyper{H}_X$ and reconnected restrictions $\hyper{H}_{\cap X}$ are naturally ordered hypergraphs.

\begin{definition}
  \label{def:contextual-family}
    A family $\calH$ of ordered hypergraphs is \defn{contextual} if 
    \begin{enumerate}
      \item any ordered hypergraph $\hyper{H} \in \calH$ is contextual.
      \item for any $\hyper{H} \in \calH$ an any $X \subseteq H$, all the connected components of $\hyper{H}\setminus X$ are in $\calH$.
      \item we have $\{\hyper{H}_{\cap X} \ | \ X \subset H, |X|=3, \hyper{H} \in \calH \} \subseteq \calH(2)$.
    \end{enumerate}
\end{definition}

The term rewrite systems from \cref{ss:rewriting-constructs,ss:rewriting-constructions} can be adapted to a rewrite system on \emph{all} hypergraphs of $\calH$.
We shall focus on the constructions rewrite system. 

\begin{definition}
  For a contextual family of hypergraphs $\calH$, we consider the \defn{constructions signature}~$\Sigma_\calH^c$ defined by the following data:
  \begin{itemize}
    \item Variables and sorts are elements of $\calH$, 
    \item Function symbols are pairs of an hypergraph $\hyper{H} \in \calH$ and one of its elements 
    $$F:=\{(x,\hyper{H}) \ | \ x \in H, \ \hyper{H} \in \calH \}.$$
    \item For $(x,\hyper{H}) \in F$, we define $\ari(x,H)$ as the number of connected components of~$\hyper{H} \setminus {x}$.
    \item Variables $\hyper{H} \in V$ are their own output sort $\outsort(\hyper{H}):=\hyper{H}$, while function symbols~$(x,\hyper{H}) \in F$ have output sort $\outsort(x,\hyper{H}):=\hyper{H}$.
    \item For function symbols $(x,\hyper{H}) \in F$ such that $\hyper{H},x \leadsto H_1,\ldots,H_n$, and for $1 \leq i \leq n$, we define $\insort((x,\hyper{H}),i):=\hyper{H}_i$.
  \end{itemize}
\end{definition}

It is clear from \cref{def:contextual-family} and the fact that the (reconnected) restriction of a contextual hypergraph is contextual, that this signature is well-defined.
Moreover, it is straightforward to adapt \cref{l:bijection-constructions} and \cref{def:rules-2} to obtain a term rewrite system $(\Sigma_\calH,R_\calH)$ on the constructions of $\calH$. 

From \cref{thm:critical-pairs}, we have that all local confluence diagrams for $(\Sigma_\calH,R_\calH)$ have the form of some $X$-face, for $X \subseteq H$, $|X|=3$ and $\hyper{H} \in \calH$. 
The fact that $\calH$ is contextual imposes an additional uniformity constraint on these diagrams.

\begin{thm} 
Let $\calH$ be a contextual family of ordered hypergraphs.
For any $\hyper{H} \in \calH$ and subset $X \subseteq H$ with $|X|=3$, all the $X$-local confluence diagrams have the same form $\hyper{H}_{\cap X}$. 
\end{thm}

\begin{proof} 
  The proof is an easy consequence of Lemma~\ref{instance-construct} and \cref{situation-construct}. 
\end{proof}

\cref{thm:confluent} implies that $(\Sigma_\calH,R_\calH)$ is confluent.
Moreover, in virtue of Condition (3) in \cref{def:contextual-family} all the possible forms of the local confluence diagrams of $(\Sigma_\calH,R_\calH)$ are in~$\calH(2)$.
We argue that these diagrams should be called \emph{coherence conditions}, in view of the following examples of contextual families and their coherence theorems. 

%%%%%%%%%%%%%%%%%%%%%%%%%%%%%%%%%%%%%%%

\subsection{Examples}
\label{ss:examples}

We call \defn{contextual graph-associahedra} (resp.\ \defn{contextual nestohedra}) the hypergraph polytopes whose underlying hypergraph is a connected (hyper)graph which is moreover contextual.
Here, we include a copy of each (hyper)graph for each possible total order on its vertices.
Recall that simplices, cubes, associahedra, permutahedra and operahedra were introduced in \cref{ss:hypergraph-polytopes}.

\begin{samepage}
  \begin{thm}
    \label{thm:examples}
    The following families of hypergraph polytopes are contextual:
    \begin{enumerate}[label=(\alph*)]
      \item simplices,
      \item cubes,
      \item associahedra,
      \item permutahedra,
      \item operahedra,
      \item contextual graph-associahedra,
      \item contextual nestohedra.
    \end{enumerate}
  \end{thm}
\end{samepage}

\begin{proof}
  Let us proceed one family at a time.
  For each one, we check Conditions (1)-(3) in \cref{def:contextual-family}.
  We consider sets of vertices to be $H=\{1,\ldots,n\}$.
  \begin{enumerate}[label=(\alph*)]
    \item Conditions (1)-(3) follow easily from the fact that hyperedges of simplices are all either singletons or the maximal hyperedge.
    \item We first prove Condition (1). 
    Note $\hyper{C}_n$ is saturated, and that $(\hyper{C}_n)_{\set{1,\ldots,m}}=\hyper{C}_m$ if $m\leq n$.
    So we have to check that for all $m\leq n$ and all $i,j,k\leq m$, we have $\xyz{k}{{\hyper{C}_n}}{\set{i,j}}$ iff
    $\xyz{k}{{\hyper{C}_m}}{\set{i,j}}$, which follows immediately from the observation that for all $p\geq m$ we have
    $\xyz{k}{{\hyper{C}_p}}{\set{i,j}}$ iff $i<k$ and $j<k$.
    For Conditions (2) and (3) it suffices to observe that the connected components of $\hyper{C}_n\setminus X$, for some $X$, are all cubes $\hyper{C}_m$ with $m<n$.
    \item[(c)-(e)] Conditions (1) and (2) follow from the fact that any connected subgraph of a linear (resp. complete, clawfree block) graph is a linear (resp. complete, clawfree block) graph. 
    Condition (3) follows from the fact that any reconnected complement of a subset in a linear (resp. complete, clawfree block) graph is a linear (resp. complete, clawfree block) graph.
    %As to Condition (1), it is proved in~\cite[Lem.~12]{COI} that the connected subsets of $\hyper{L}({\cal T})$ are in bijective correspondence with the subtrees of $\cal T$ having at least two nodes, through a map $E\mapsto  {\cal T}_E$ such that $\hyper{L}({\cal T})_E=\hyper{L}({\cal T_E})$. Suppose, say, that $\set{x,y,z}\inc E$ and $\xyz{x}{\hyper{L}({\cal T})_E}{\set{y},\set{z}}$. Then it means on the tree side that after removing the edge $x$ from ${\cal T}_E$, resulting in two disjoint subtrees ${\cal T}_E^1$ and ${\cal T}_E^2$ of ${\cal T}_E$, we have, say, $y\in{\cal T}_E^1$ and $z\in{\cal T}_E^2$. 
    %On the other hand, removing $x$ from ${\cal T}$ results in subtrees ${\cal T}^1$ and ${\cal T}^2$, containing ${\cal T}_E^1$ and ${\cal T}_E^2$, respectively. 
    %Therefore $\xyz{x}{\hyper{L}({\cal T}))}{\set{y},\set{z}}$. 
    %And vice-versa.
    \item[(f)-(g)] This is immediate from the definitions.
  \end{enumerate}
\end{proof}

%toute restriction d'un hypergraphe contextuel est contextuelle

\begin{rem}
  Note that contextual (hyper)graphs do not contain all graph-associahedra.
  For instance, we have seen in \cref{non-contextual-2} that the cyclohedra are not contextual. 
  It would be interesting to characterize combinatorially contextual (hyper)graphs.
\end{rem}


%%%%%%%%%%%%%%%%%%%%%%%%%%%%%%%%%%%%

\subsection{Categorical coherence}
\label{ss:coherence}

Let us quickly recall MacLane's coherence theorem.
The scene is the data of a category $\mathbf C$, a bifunctor $\otimes:\mathbf{C}^2\rightarrow \mathbf C$ and a natural iso $\alpha$ from the functor
$(X,Y,Z)\mapsto (X\otimes Y)\otimes Z$ to the functor  $(X,Y,Z)\mapsto X \otimes (Y\otimes Z)$. 
The coherence theorem states that for any two functors $F,G$ from $\mathbf{C}^{n+1}$ to $\mathbf{C}$ arising from $n$ iterations of $\otimes$, any two  natural transformations $\lambda_1,\lambda_2$ from $F$ to $G$  ``written using $\alpha$ or its inverse'' are equal, provided the statement holds in the following special case  -- called  {\em coherence condition}: 
\begin{itemize}
\item $F=(X,Y,Z,U)\mapsto ((X\otimes Y)\otimes Z)\otimes U,$ 
\item $G=(X,Y,Z,U)\mapsto X\otimes (Y\otimes (Z\otimes U)),$ 
\item $\lambda_1=(X\otimes\alpha_{Y,Z,U})\circ\alpha_{X,Y\otimes Z,U} \circ (\alpha_{X,Y,Z}\otimes U),$ and
\item $\lambda_2= \alpha_{X,Y,Z\otimes U}\circ \alpha_{X\otimes Y,Z,U},$ 
\end{itemize}
i.e., provided the following diagram (Mac Lane's pentagon) commutes:
\begin{center}
\vspace{-.5cm}
$$
 \xymatrix @-1.65pc {&& ((X\otimes Y)\otimes Z)\otimes U \ar @{->}[ddll]^{\alpha_{X,Y,Z}\otimes U} \ar @{->}[dddrr]^{\alpha_{X\otimes Y,Z,U}}&& \\
 &&&&\\
 (X\otimes (Y\otimes Z))\otimes U  \ar @{->}[dd]^{\alpha_{X,Y\otimes Z,U} }&&   && \\
 &&&&(X\otimes Y)\otimes (Z\otimes U) \ar @{->}[dddll]_{\alpha_{X,Y,Z\otimes U}}\\
 X\otimes ((Y\otimes Z)\otimes U) \ar @{->}[ddrr]_{X\otimes\alpha_{Y,Z,U}} &  &\\
 &&&&\\
 &&X\otimes (Y\otimes (Z\otimes U))&&}
$$
\end{center}

Via Huet's correspondence \cite{Huet-notes-cat}, the annotated proof of confluence of the rewriting system $(\Sigma_\calH, R_\calH)$ associated to the contextual family of associahedra $\calH$ provides a proof of MacLane's coherence theorem, with the pentagon in $\calH(2)$ acting as the coherence condition. 
The following examples explain the translation between the language of hypergraph polytopes and the language of monoidal categories. 

\begin{example}
Consider the linear tree

\vspace{-.7cm}
\begin{center}
$$\xymatrix @-1.65pc {{\cal T} & := &X \ar @{-}[rr]^{1}&& Y \ar @{-}[rr]^{2}&& Z \ar @{-}[rr]^{3}&& U}
  $$
\end{center}
%\vspace{-.2cm}
Then $\hyper{L}({\cal T})$ is the associahedron $\hyper{K}^3$. 
The constructs of ${\cal T}$ decorate a pentagon as follows
%$$
% \xymatrix @-1.65pc {&& 3(2(1)) \ar @{->}[ddll]_{3(\set{1,2})} \ar @{->}[dddrr]^{\set{2,3}(1)}&& \\
% &&&&\\
%3(1(2)) \ar @{->}[dd]^{\set{1,3}(2)} }&&   && \\
% &&&&2(1,3) \ar @{->}[dddll]_{\set{1,2}(3)}\\
%1(3(2)) \ar @{->}[ddrr]^{1(2,3)} &  &\\
% &&&&\\
% &&1(2(3))&&}
%$$
\begin{center}
$$\xymatrix @-1.65pc {&& 3(2(1)) \ar @{->}[ddll]_{3(\set{1,2})} \ar @{->}[dddrr]^{\set{2,3}(1)}&& \\
 &&&&\\
3(1(2))   \ar @{->}[dd]^{\set{1,3}(2)}&&   && \\
 &&&&2(1,3) \ar @{->}[dddll]^{\set{1,2}(3)}\\
 1(3(2)) \ar @{->}[ddrr]^{1(\set{2,3})} &  &\\
 &&&&\\
 &&1(2(3))&&}$$
\end{center}
and are in bijective correspondence with the vertices and edges of Mac Lane's pentagon. 
The encoding is given as follows:
  \begin{itemize}
  \item $(X\otimes_1 Y)\otimes_2 (Z\otimes_3 U)$, where we annotaded the ``compositions'' $\otimes$ with the vertices of $\hyper{K}^3$, can be written $\otimes_2(\otimes_1(X,Y),\otimes_3(Z,U))$ in prefix (or tree) notation. Then we get 
 $2(1,3)$ by removing the leaf nodes of that tree.
 \item $\alpha_{X,Y,Z}\otimes_3 U$ can be interpreted as $(X\otimes_1 Y\otimes_2 Z)\otimes_3 U$ (a non fully parenthesed expression), which likewise 
 translates as $3(\set{1,2})$,where $3(-)$ makes the job of contextualisation.
 \item Likewise, we can move from
$\alpha_{X,Y\otimes_2 Z,U}$ to $X\otimes_1(Y\otimes_2 Z)\otimes_3 U$ to $\set{1,3}(2)$, where $2$ makes the job of instantiation.
\end{itemize}
\end{example}

\begin{example}
Taking the 4-dimensional associahedron $\hyper{K}^5$ (with vertex set $\set{0,1,2,3,4}$), we get the following instance in context of $\hyper{K}^3=\recrestr{\hyper{K}^5}{\set{1,2,3}}$, i.e. of Mac Lane's condition:
\begin{center}
$$\xymatrix @-1.65pc {&& 4(3(2(1(0)))) \ar @{->}[ddll]_{4(3(\set{1,2}(0)))} \ar @{->}[dddrr]^{4(\set{2,3}(1(0)))}&& \\
 &&&&\\
4(3(1(0,2)))  \ar @{->}[dd]^{4(\set{1,3}(0,2))}&&   && \\
 &&&&4(2(1(0),3)) \ar @{->}[dddll]^{4(\set{1,2}(0,3))}\\
 4(1(0,3(2))) \ar @{->}[ddrr]^{4(1(0,\set{2,3}))} &  &\\
 &&&&\\
 &&4(1(0,2(3)))&&}$$
\end{center}
We recover the (encoding of the) edge 
 $$
 \xymatrix @-2pc {&& ((((X_1\otimes_0 X_2)\otimes_1 Y)\otimes_2 Z)\otimes_3 U)\otimes_4 V \ar @{->}[dddddll]_(.6){(\alpha_{(X_1\otimes X_2),Y,Z}\otimes U)\otimes V\quad} && \\
 &&&&\\
 &&&&\\
  &&&&\\
    &&&&\\
( ((X_1\otimes_0 X_2)\otimes_1 (Y\otimes_2 Z))\otimes_3 U)\otimes_4 V  &&   && \\
\ }
$$
as the top left edge above.
\end{example}

Here, the fact that the family of associahedra is contextual implies in particular that the local confluence diagram associated to the expression
$$ (-\otimes_0 - \otimes_1 - \otimes_2 (U \otimes_3 V)),$$
which takes place on the $4$-dimensional associahedron, has the same form as the local confluence diagram associated to the expression
$$ ((X \otimes_0 Y \otimes_1 Z \otimes_2 U) \otimes_3 -),$$
and thus that the former one can be seen as an ``instance'' of MacLane's pentagon, while the latter can be seen as the same pentagon ``in context''.
As we have seen in \cref{non-contextual-2}, this interpretation does not hold anymore if one considers the cycle graph instead of the linear graph (that is, if one was to identify $X$ and $V$ in the expressions above). 

A similar description holds for the permutahedra and the operahedra \cite{CLA1}, giving coherence theorems for categorified permutads and operads, respectively. 
All contextual families of nestohedra and their associated coherence theorems considered so far are summarized in \cref{table:contextual-hyper}.

\begin{table}[h!]
	\begin{center}
	\begin{tabular}{c|c|c}
	Family & Algebraic structure & Coherence theorem \\
	\hline
	Simplices & - & - \\
	Cubes & - & - \\
	Associahedra & Monoidal category & \cite{MacLane63} \\
	Permutahedra & Categorified permutads & \cite{CLA1} \\
	Operahedra & Categorified operads & \cite{DP15,CLA1} \\
	Contextual graph-associahedra & - & - \\
	Contextual nestohedra & - & - 
	\end{tabular}
	\end{center}
  \caption{Families of contextual hypergraphs, the categorical structures that they encode, and their associated coherence theorems.}
  \label{table:contextual-hyper}
\end{table}

\begin{rem}
  It would be interesting to complete \cref{table:contextual-hyper} with appropriate structures and coherence theorems.
  It seems likely that contextual graph-associahedra would be related to a certain type of categorified reconnectads \cite{DotsenkoKeilthyLyskov}.
\end{rem}




