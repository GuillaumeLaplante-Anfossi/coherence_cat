% !TEX root = ../Coherence2.tex

\section{Categorical coherence} 
\label{s:coherence}

%%%%%%%%%%%%%%%%%%%%%%%%%%%%%%%%%%%%%%

\begin{table}[h!]
	\begin{center}
	\begin{tabular}{c|c|c}
	Family & Algebraic structure & Coherence theorem \\
	\hline
	Simplices & - & - \\
	Cubes & - & - \\
	Associahedra & Monoidal category & \cite{MacLane63} \\
	Permutahedra & Categorified permutads & \cite{CLA1} \\
	Operahedra & Categorified operads & \cite{DP15,CLA1} \\
	Contextual graph-associahedra & Categorified reconnectads? & - \\
	Contextual nestohedra & - & - 
	\end{tabular}
	\end{center}
\end{table}

Conjectures: 
\begin{itemize}
  \item contextual graph-associahedra give coherence for (some specific) categorified reconnectads \cite{DotsenkoKeilthyLyskov}
  \item contextual nestohedra give coherence for a hypergraphic generalization
\end{itemize}




\subsection{Interpretation}

Associahedra form a subfamily of operahedra: those obtained from linear trees.  Consider the linear tree

\vspace{-1cm}
\begin{center}
$$\xymatrix @-1.65pc {{\cal L} & = &X \ar @{-}[rr]^{1}&& Y \ar @{-}[rr]^{2}&& Z \ar @{-}[rr]^{3}&& U}
 $$
 \end{center}
 \vspace{-.2cm}
 
 \noindent
 (represented horizontally).  Then $\hyper{G}({\cal L})$ is the associahedron $\hyper{K}^3$. The constructs of
  ${\cal L}$ decorate a pentagon as follows:

%$$
% \xymatrix @-1.65pc {&& 3(2(1)) \ar @{->}[ddll]_{3(\set{1,2})} \ar @{->}[dddrr]^{\set{2,3}(1)}&& \\
% &&&&\\
%3(1(2)) \ar @{->}[dd]^{\set{1,3}(2)} }&&   && \\
% &&&&2(1,3) \ar @{->}[dddll]_{\set{1,2}(3)}\\
%1(3(2)) \ar @{->}[ddrr]^{1(2,3)} &  &\\
% &&&&\\
% &&1(2(3))&&}
%$$
\begin{center}
$$\xymatrix @-1.65pc {&& 3(2(1)) \ar @{->}[ddll]_{3(\set{1,2})} \ar @{->}[dddrr]^{\set{2,3}(1)}&& \\
 &&&&\\
3(1(2))   \ar @{->}[dd]^{\set{1,3}(2)}&&   && \\
 &&&&2(1,3) \ar @{->}[dddll]^{\set{1,2}(3)}\\
 1(3(2)) \ar @{->}[ddrr]^{1(\set{2,3})} &  &\\
 &&&&\\
 &&1(2(3))&&}$$
\end{center}
and are in bijective correspondence with the vertices and edges of Mac Lane's pentagon (cf. Section \ref{preamble-section}). 
  Let us sketch the encoding:
  \begin{itemize}
  \item $(X\otimes_1 Y)\otimes_2 (Z\otimes_3 U)$, where we annotaded the ``compositions'' $\otimes$ with the vertices of $\hyper{K}^3$, can be written $\otimes_2(\otimes_1(X,Y),\otimes_3(Z,U))$ in prefix (or tree) notation. Then we get 
 $2(1,3)$ by removing the leaf nodes of that tree.
 \item $\alpha_{X,Y,Z}\otimes_3 U$ can be interpreted as $(X\otimes_1 Y\otimes_2 Z)\otimes_3 U$ (a non fully parenthesed expression), which likewise 
 translates as $3(\set{1,2})$,where $3(\_)$ makes the job of contextualisation.
 \item Likewise, we can move from
$\alpha_{X,Y\otimes_2 Z,U}$ to $X\otimes_1(Y\otimes_2 Z)\otimes_3 U$ to $\set{1,3}(2)$, where $2$ makes the job of instantiation.
\end{itemize}

Taking the 4-dimensional associahedron $\hyper{K}^5$ (with vertex set $\set{0,1,2,3,4}$), we get the following instance in context of $\hyper{K}^3=\recrestr{\hyper{K}^5}{\set{1,2,3}}$, i.e. of Mac Lane's condition:
\begin{center}
$$\xymatrix @-1.65pc {&& 4(3(2(1(0)))) \ar @{->}[ddll]_{4(3(\set{1,2}(0)))} \ar @{->}[dddrr]^{4(\set{2,3}(1(0)))}&& \\
 &&&&\\
4(3(1(0,2)))  \ar @{->}[dd]^{4(\set{1,3}(0,2))}&&   && \\
 &&&&4(2(1(0),3)) \ar @{->}[dddll]^{4(\set{1,2}(0,3))}\\
 4(1(0,3(2))) \ar @{->}[ddrr]^{4(1(0,\set{2,3}))} &  &\\
 &&&&\\
 &&4(1(0,2(3)))&&}$$
\end{center}
We recover the (encoding of the) edge 
 $$
 \xymatrix @-2pc {&& ((((X_1\otimes_0 X_2)\otimes_1 Y)\otimes_2 Z)\otimes_3 U)\otimes_4 V \ar @{->}[dddddll]_(.6){(\alpha_{(X_1\otimes X_2),Y,Z}\otimes U)\otimes V\quad} && \\
 &&&&\\
 &&&&\\
  &&&&\\
    &&&&\\
( ((X_1\otimes_0 X_2)\otimes_1 (Y\otimes_2 Z))\otimes_3 U)\otimes_4 V  &&   && \\
\ }
$$
displayed of Section \ref{s:introduction} as the top left edge above.


\section{Recovering MacLane}

Let $\hyper{H}$ be a connected hypergraph. 
Suppose that for any connected subset $Y$, and $X \subseteq Y$, the number of connected components of~$\hyper{H}_Y\setminus X$ is less than or equal to $|X|+1$.
Then, we can consider the following variation on \cref{def:signature-hyper}.
Consider the signature $\Sigma_\hyper{H}$ made of the following data: 
\begin{itemize}
  \item Variables are any set $V$. 
  \item Function symbols are pairs of a connected subset of $H$ and one of its subsets 
  $$F:=\{(X,Y) \ | \ X \subseteq Y \subseteq H, \ \hyper{H}_Y \text{ is connected}\}.$$
  \item Sorts are either the ``variable" sort, or a connected subset of $H$, i.e. 
  $$S:=\{ X \subseteq H \ | \ \hyper{H}_X \text{ is connected}\}\cup\{*\}.$$
  \item For $(X,Y) \in F$, we define $\ari(X,Y):=|X|+1$.
  \item All variables $v \in V$ have the same output sort $\outsort(v):=*$, while function symbols $(X,Y) \in F$ have output sort $\outsort(X,Y):=Y$.
  \item For function symbols $(X,Y) \in F$ such that $\hyper{H}_Y,X \leadsto Y_1,\ldots,Y_k$, we define $\insort((X,Y),i):=Y_i$ for $1 \leq i \leq k$, and $\insort((Y,X),i)=*$ for the remaining inputs.
\end{itemize}

Closed terms of output sort $H$ are still in bijection with constructs, i.e. \cref{l:bijection-terms} holds \emph{mutatis mutandis} for this new signature. 
But now one can define rewriting rules that recover precisely MacLane in the case of the associahedra. 







