% !TEX root = ../Coherence2.tex

\section{Categorical coherence} 
\label{s:coherence}

%%%%%%%%%%%%%%%%%%%%%%%%%%%%%%%%%%%%%%

\begin{table}[h!]
	\begin{center}
	\begin{tabular}{c|c|c}
	Family & Algebraic structure & Coherence theorem \\
	\hline
	Simplices & - & - \\
	Cubes & - & - \\
	Associahedra & Monoidal category & \cite{MacLane63} \\
	Permutahedra & Categorified permutads & \cite{CLA1} \\
	Operahedra & Categorified operads & \cite{DP15,CLA1} \\
	Contextual graph-associahedra & Categorified reconnectads? & - \\
	Contextual nestohedra & - & - 
	\end{tabular}
	\end{center}
\end{table}

Conjectures: 
\begin{itemize}
  \item contextual graph-associahedra give coherence for (some specific) categorified reconnectads \cite{DotsenkoKeilthyLyskov}
  \item contextual nestohedra give coherence for a hypergraphic generalization
\end{itemize}




\subsection{Interpretation}




\section{Recovering MacLane}

Let $\hyper{H}$ be a connected hypergraph. 
Suppose that for any connected subset $Y$, and $X \subseteq Y$, the number of connected components of~$\hyper{H}_Y\setminus X$ is less than or equal to $|X|+1$.
Then, we can consider the following variation on \cref{def:signature-hyper}.
Consider the signature $\Sigma_\hyper{H}$ made of the following data: 
\begin{itemize}
  \item Variables are any set $V$. 
  \item Function symbols are pairs of a connected subset of $H$ and one of its subsets 
  $$F:=\{(X,Y) \ | \ X \subseteq Y \subseteq H, \ \hyper{H}_Y \text{ is connected}\}.$$
  \item Sorts are either the ``variable" sort, or a connected subset of $H$, i.e. 
  $$S:=\{ X \subseteq H \ | \ \hyper{H}_X \text{ is connected}\}\cup\{*\}.$$
  \item For $(X,Y) \in F$, we define $\ari(X,Y):=|X|+1$.
  \item All variables $v \in V$ have the same output sort $\outsort(v):=*$, while function symbols $(X,Y) \in F$ have output sort $\outsort(X,Y):=Y$.
  \item For function symbols $(X,Y) \in F$ such that $\hyper{H}_Y,X \leadsto Y_1,\ldots,Y_k$, we define $\insort((X,Y),i):=Y_i$ for $1 \leq i \leq k$, and $\insort((Y,X),i)=*$ for the remaining inputs.
\end{itemize}

Closed terms of output sort $H$ are still in bijection with constructs, i.e. \cref{l:bijection-terms} holds \emph{mutatis mutandis} for this new signature. 
But now one can define rewriting rules that recover precisely MacLane in the case of the associahedra. 







