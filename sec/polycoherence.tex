% !TEX root = ../Coherence.tex

\section{Topological coherence} 
\label{s:polycoherence}

%%%%%%%%%%%%%%%%%%%%%%%%%%%%%%%%%%

\subsection{Coherence \`a la Van Kampen}

Let $X$ be a regular CW complex, and let $X_k$, $k\geq 0$ denote its $k$-skeleton. 
\correction{
For an edge $e$ of $X$, denote its attaching map $f_e : \mathbb{S}^0 \to X_0$.
Consider the free groupoid $\mathcal{F}(X)$ with set of objects $X_0$, and morphisms generated by $\alpha_e: f_e(-1) \to f_e(1)$ associated to each edge $e \in X_1$.} 
A \defn{combinatorial path} on $X$ is a composable sequence of $\alpha$ and $\alpha^{-1}$ morphisms (a \emph{word} in $\alpha$ and $\alpha^{-1}$).
Two combinatorial paths $\gamma, \gamma' \in \mathcal{F}(X)(x,y)$ with the same endpoints are said to be \defn{parallel}.

\correction{
    Let $A$ be a $2$-cell of $X$.
    For any $x \in A_0$, the attaching map $f_A : \mathbb{S}^1 \to X_1$ of $A$ defines a morphism $\gamma_A \in \mathcal{F}(X)(x,x)$, given by the sequence of edges $e_1,\ldots,e_n$ in its image starting at~$x$ and respecting the anti-clockwise orientation of $\mathbb{S}^1$.
    Here, one selects $\alpha_{e_i}$ if the orientation of $f_A$ restricted to $e_i$ agrees with the one of $f_{e_i}$, and $\alpha_{e_i}^{-1}$ otherwise.}
Two parallel combinatorial paths $\gamma, \gamma'$ are said to be \defn{elementary combinatorially homotopic} if they differ exactly by a relation of the form $\gamma_A = \id_x$, for some $2$-cell $A$ \correction{and vertex $x$ of $A$}.
That is, one can rewrite $\gamma$ into $\gamma'$ or $\gamma'$ into $\gamma$ by replacing some (possibly empty) subword of $\gamma$ with an equivalent subword using a relation $\gamma_A = \id_x$.
More generally, two parallel combinatorial paths are \defn{combinatorially homotopic} if they are related by a sequence of elementary combinatorial homotopies.

\correction{
Let $\mathcal{C}(X)$ denote the quotient of the groupoid $\mathcal{F}(X)$ by the relations $\gamma_A=\id_x$ for some choice of $x$, for each $2$-cell $A$ of $X$.
That is, the quotient of $\mathcal{F}(X)$ by the combinatorial homotopy equivalence relation.
Note that the definition of $\mathcal{C}(X)$ does not depend on the choice of $x$, for every $2$-cell $A$.
Indeed, if $x'\neq x \in A_0$ defines a relation $\gamma_A'=\id_{x'}$, we have $\gamma_A'=\delta \gamma_A \delta^{-1}$ in $\mathcal{F}(X)$, where $\delta$ is the oriented path between $x$ and $x'$. 
Thus, a path~$\gamma$ can be rewritten into $\gamma'$ using $\gamma_A=\id_x$ if and only if it can be rewritten using $\gamma_A'=\id_{x'}$.}

\correction{
Let $\Pi(X)$ denote the \defn{fundamental groupoid }of $X$, that is the groupoid with objects the points of $X$ and morphisms the homotopy classes of paths between them.}

\begin{thm}
\label{thm:top-coherence}
    Any two parallel combinatorial paths on $X$ are combinatorially homotopic if and only if every path component of $X$ is simply connected.
\end{thm}

\begin{proof}
    \correction{
    For $Y \subset X$, let us write $\Pi(X)Y$ for the full subcategory of the fundamental groupoid of $X$ spanned by $Y$.
    Then, we have an isomorphism of groupoids \[ \Pi(X)X_0 \cong \mathcal{C}(X) \ . \]}
    To show this, one proceeds in three steps. 
    First, one shows that the fundamental groupoid \correction{$\Pi(X_1)X_0$} of the $1$-skeleton of $X$ is free on the homotopy classes of maps generated by the attaching maps of the $1$-cells, that is, free on the $\alpha$-morphisms \cite[9.1.5]{Brown2006}.
    Thus, one gets \correction{$\Pi(X_1)X_0 \cong \mathcal{F}(X)$}. 
    Second, one shows that the fundamental groupoid $\Pi(X_2)X_0$ of the $2$-skeleton of $X$ is the free groupoid $\Pi(X_1)X_0$ modulo the relations $\gamma_A=1$, for $A$ a $2$-cell of $X$ \cite[9.1.6]{Brown2006}. 
    This is done through repeated application of the Seifert--Van Kampen theorem; one then has \correction{$\Pi(X_2)X_0 \cong \mathcal{C}(X)$}.
    Third, one shows that the inclusion of $X_2$ in $X$ induces an isomorphism of fundamental groupoids \correction{$\Pi(X_2)X_0 \cong \Pi(X)X_0$} \cite[9.1.7]{Brown2006}, which concludes the proof of the isomorphism \correction{$\Pi(X)X_0 \cong \mathcal{C}(X)$}.
    The theorem then follows, since every path component of $X$ is simply connected if and only if its fundamental groupoid $\Pi(X)$ is trivial, \correction{which holds if and only if its full subcategory $\Pi(X)X_0$ is trivial.}  
\end{proof}

\correction{
Let us say that $X$ is \defn{combinatorially connected} if there is a combinatorial path between any two vertices of $X$. 
In the course of the preceding proof, we have in particular showed the following.
\begin{corollary}
    \label{cor:combinatorially-connected}
    A regular CW complex $X$ is combinatorially connected if and only it is connected.
\end{corollary}
}


%%%%%%%%%%%%%%%%%%%%%%%%%%%%%%%%%%

\subsection{Coherence \`a la Morse}

Let $X\subset \R^n$ be a polyhedral complex. 
Let $\vec v \in \R^n$ be \defn{generic} on the edges of $X$, meaning that for any pair of vertices $x,y \in X$ belonging to the same edge of $X$, we have $\langle \vec v , x \rangle \neq \langle \vec v, y\rangle$.  
Such a generic vector $\vec v$ induces a natural orientation on the edges of $X$, directed from the source vertex where the functional $\langle \vec v, - \rangle$ is minimal to the target vertex where it is maximal. 

In general, for any face $F \subset X$ of $X$, there is a unique \defn{source} vertex $\so(F)$ such that all its adjacent edges $e \subset F$ are outgoing, and a unique \defn{sink} vertex $\sk(F)$ whose adjacent edges are all incoming.
When the complex $X$ has a unique \defn{global sink}, a vertex whose adjacent edges $e \subset X$ are all incoming, we will denote it by $\sk(X)$. 

Let $H:=\{y \in \R^n \ | \ \langle \vec v , y \rangle = 0\}$ be the linear hyperplane orthogonal to $\vec v$.  
For every vertex $x \in X$, choose $\varepsilon >0$ such that the interval between $\langle \vec v , x \rangle$ and $\langle \vec v , x \rangle + \varepsilon$ does not contain the image of any other vertex under the ``height" function $\langle \vec v, - \rangle$. 

\begin{definition}
    The \defn{outgoing link} $\oLk(x,X)$ of a vertex $x \in X$ is the intersection $\mathcal{F} \cap (H+x+\varepsilon \vec v)$ of the family of faces $\mathcal{F}(x,X):=\{ F \subset X \ | \ \so(F)=x \}$ with the affine hyperplane $H+x+\varepsilon \vec v$. 
\end{definition}

\correction{
    Recall from \cite[Sec.~2.1]{Ziegler95} that the \defn{vertex figure} $P/x$ of a polytope $P$ at a vertex $x$ is obtained by cutting $P$ by a hyperplane that cuts off the single vertex $x$. 
    Such a cut establishes a bijection between the $(k-1)$-faces of $P/x$ and the $k$-faces of $P$ which contain $x$ \cite[Prop.~2.4]{Ziegler95}. 
\begin{lemma}
    \label{l:vertex-figure}
    For any $k\geq 0$, there is a bijection between the $k$-faces of $\mathcal{F}(x,X)$ and the $(k-1)$-faces of $\oLk(x,X)$.
\end{lemma}
\begin{proof}
    Each maximal face of $\mathcal{F}(x,X)$ with respect to inclusion is a polytope $P$, for which the intersection $P \cap (H+x+\varepsilon \vec v)$ is the vertex figure $P/x$ of $P$ at $x$. 
    By \cite[Prop.~2.4]{Ziegler95}, there is a bijection between the $k$-faces of $P$ and the $(k-1)$-faces of $P/x$.
    Collecting these bijections for all maximal faces of $\mathcal{F}(x,X)$, and making the appropriate identifications, we get the desired global bijection.
\end{proof}}

\correction{In this Section we shall focus on polyhedral complexes whose outgoing links are connected.}

\begin{proposition}
    \label{lemma:outgoing-link}
    Let $X$ be a polyhedral complex.
    If there is a generic vector $\vec v \in \R^n$ such that the outgoing link of every vertex is connected, then every path component of $X$ is simply connected.
\end{proposition}

\begin{proof}
    Let $\vec v \in \R^n$ be generic with respect to $X$, and suppose that the outgoing link of every vertex is connected. 
    Since $\vec v$ is generic on edges, it defines a Morse function $\langle \vec v , -\rangle$ on $X$, in the sense of \cite[Def.~2.2]{bestvinaMorseTheoryFiniteness1997}.
    As in classical Morse theory, one can determine the homotopy type of $X$ by considering its successive level sets. 
    For $t \in \R$ denote by $X_t$ the closed subspace of $X$ containing points $x$ such that $\langle x, \vec v \rangle$ is at least $t$.
    Let $x$ be a vertex of $X$ of height $h=\langle x, \vec v \rangle$.
    Observe first that $X_{h+\epsilon}$, for some small $\epsilon>0$, is homotopy equivalent to $X_{h'}$ where $h' > h$ is the next greater height at which there is a vertex.
    That is, the homotopy type of $X$ can only change at vertices  \cite[Lem.~2.3]{bestvinaMorseTheoryFiniteness1997}.
    Then, one proves that $X_h$ is homotopy equivalent to the pushout of $X_{h+\epsilon}$ with the cone over the outgoing link of $x$ along the outgoing link of $x$  \cite[Lem.~2.5]{bestvinaMorseTheoryFiniteness1997}.
    By our assumption, the outgoing link of $x$ is connected, and thus the cone over it is simply connected. 
    Since the pushout of simply connected spaces over a connected space is always simply connected (this is an application of the Seifert--Van Kampen theorem), we obtain by induction that every path component of $X$ is simply connected \cite[Point (3) of Cor.~2.6]{bestvinaMorseTheoryFiniteness1997}.
\end{proof}

The converse of \cref{lemma:outgoing-link} is not true in general: many simply connected polyhedral complexes, as the one represented in \cref{fig:outgoingpoly}, have disconnected outgoing links, for many (sometimes for all) choices of generic orientation vectors. 

\begin{figure}[h!]
\centering
\resizebox{0.4\linewidth}{!}{
\begin{tikzpicture}
    \node[regular polygon,
    draw,
    regular polygon sides = 8, minimum size = 3cm] (p) at (0,0) {};
    \draw[-] (p.202.5)--(180:4)--(p.157.5);
    \draw[-] (p.157.5)--(135:4)--(p.112.5);
    \draw[-] (p.112.5)--(90:4)--(p.67.5);
    \draw[-] (p.67.5)--(45:4)--(p.22.5);
    \draw[-] (p.22.5)--(0:4)--(p.-22.5);
    \draw[-] (p.-22.5)--(-45:4)--(p.-67.5);
    \draw[-] (p.-67.5)--(-90:4)--(p.-112.5);
    \draw[-] (p.-112.5)--(-135:4)--(p.-157.5);
\end{tikzpicture}}
\caption{A simply connected polyhedral complex which admits disconnected outgoing links for every choice of generic vector.}
\label{fig:outgoingpoly}
\end{figure}

\correction{An important class of complexes which have connected outgoing links are polytopes, which will be our main object of study in the next sections.}

\begin{proposition}
\label{prop:polytopes}
    Let $P \subset \R^n$ be a polytope, and let $\vec v \in \R^n$ be generic with respect to $P$. 
    Then, the outgoing link of every vertex of $P$ is connected.
\end{proposition}

\begin{proof}
    Define the linear hyperplane $H:=\{y \in \R^n \ | \ \langle \vec v, y \rangle = 0\}$, and consider the two half-spaces $H^{-}:=\{y \in \R^n \ | \ \langle \vec v, y \rangle < 0\}$ and $H^{+}:=\{y \in \R^n \ | \ \langle \vec v, y \rangle < 0\}$.
    Since $\vec v$ is not perpendicular to any edge of $P$, it defines a partition of the vertices of the vertex figure $P/x$ into two connected components: the vertices that lie in $H^{-}$, which correspond to incoming edges of $P$ at $x$, and the vertices that lie in $H^{+}$, which correspond to outgoing edges of $P$ at $x$.
    Thus, the outgoing link of $x$ is connected, and the proof is complete.
\end{proof}

\correction{From now on we shall suppose that the polyhedral complexes $X$ that we consider are endowed with a regular CW structure and provided with a generic vector $\vec v$.
Combining \cref{lemma:outgoing-link} with \cref{thm:top-coherence}, we have that any polyhedral complex $X$ whose outgoing links are all connected satisfies the property ``any two parallel combinatorial paths on $X$ are combinatorially homotopic''.
We shall now derive this same result by following an alternative, more combinatorial path, getting as close as possible to the proof of \cite[Thm~3.1]{MacLane63}.}

\correction{Let us start with a small Lemma.
In the rest of this Section we shall use the notion of combinatorially connectedness, which as we have seen in \cref{cor:combinatorially-connected}, is equivalent to connectedness for the spaces we consider.
\begin{lemma}
    \label{l:unique-sink}
    Let $X$ be a polyhedral complex with generic vector $\vec v$ such that the outgoing link of every vertex is combinatorially connected. 
    Let $e,e'$ be two edges of $X$ such that $\so(e)=\so(e')$, and suppose that there are oriented paths from $\sk(e)$ and $\sk(e')$ to local sinks~$s$ and~$s'$, respectively. 
    Then, we have $s=s'$.
\end{lemma}
\begin{proof}
    TBC
\end{proof}
}

A combinatorial path $\gamma$ on $X$ is \defn{oriented} if for any pair $(e, f)$ of consective edges in $\gamma$, we have that $\sk(e)=\so(f)$.  
When no ambiguity arises, we will omit the adjective ``combinatorial" and say only ``oriented path".
Two parallel oriented paths are said to be \defn{elementary combinatorially homotopic} if they are as non-oriented paths. 
They are \defn{combinatorially homotopic} if they are related by a sequence of elementary combinatorial homotopies between oriented paths. 

The following Lemma and its consequence \cref{p:second-proof} translate into topological terms the proof of \cite[Thm~3.1]{MacLane63}.

\begin{lemma}
\label{l:oriented}
    Let $X$ be a polyhedral complex, and let $\vec v$ be generic on the edges of $X$. 
    \correction{Consider the following three properties
    \begin{enumerate}
        \item[(i)] the outgoing link of every vertex is combinatorially connected,
        \item[(ii)] there is a unique global sink $\sk(X)$ in every connected component,
        \item[(iii)] any two parallel oriented combinatorial paths on $X$ are combinatorially homotopic.
    \end{enumerate}
    Then, $X$ satisfies (i) if and only if it satisfies (ii) and (iii).}
\end{lemma}

\begin{proof}
    \correction{First, we prove that (i) implies (ii). Suppose that there are two local sinks $s_1$ and $s_2$.}

    \correction{Second, we prove that (i) implies (iii).}
    Suppose that the outgoing link of every vertex is \correction{combinatorially} connected. 
    Let $\gamma$ and $\gamma'$ be two parallel oriented paths between two vertices $x$ and $y$. 
    We prove that they are combinatorially homotopic. 
    We proceed by induction on the maximal length $m$ of an oriented path between $x$ and $y$ in $X$. 
    Without loss of generality, we can suppose that $y=\sk(X)$, since if $y\neq\sk(X)$ we can always find an oriented path between $y$ and $\sk(X)$.
    The cases when $m=0$ and $m=1$ are trivial. 
    Suppose that the hypothesis holds up to $m=k-1, k\geq 2$, and consider two paths $\gamma$ and $\gamma'$ for which $m=k$. 
    Let $e$ and $e'$ denote the edges of $\gamma$ and $\gamma'$ that are adjacent to $x$. 
    We examine three cases.
    \begin{enumerate}
        \item If $e=e'$, we can apply the induction hypothesis to $\gamma \setminus e$ and $\gamma' \setminus e'$. 
        \item If $e \neq e'$ and both edges are on the same $2$-face $F$ of $X$, then using the induction hypothesis we have that $\gamma$ and $\gamma'$ are respectively combinatorially homotopic to the paths $\delta$ and $\delta'$ defined as follows: they go from $x=\so(F)$ to $\sk(F)$ by the unique path containing $e$ and $e'$, respectively, and then from $\sk(F)$ to $y$ along the same arbitrary oriented path. 
        Since $\delta$ and $\delta'$ are combinatorially homotopic by definition, the conclusion follows from the transitivity of the combinatorial homotopy equivalence relation. 
        \item Suppose that $e\neq e'$, and that $e$ and $e'$ are \emph{not} on the same $2$-face of $X$. 
        Since the outgoing link of $x$ is \correction{combinatorially} connected, there exists a \correction{combinatorial} path $\theta$ between \correction{the vertices corresponding to} $e$ and $e'$ in this link \correction{(\cref{l:vertex-figure})}. 
        For every edge $e_i$ of $X$ in the path $\theta$, choose an oriented path $\gamma_i$ in $X$ from $x$ to $y=\sk(X)$ going through $e_i$. 
        Now apply Point (2) above to every pair of parallel oriented paths $(\gamma_i, \gamma_{i+1})$ with $e_i$ and $e_{i+1}$ consecutive in $\theta$, and conclude again by transitivity of the combinatorial homotopy equivalence relation. 
    \end{enumerate}

    \correction{Finally, we prove that (ii) and (iii) imply (i).}
    Suppose that every pair of parallel oriented combinatorial paths are combinatorially homotopic. 
    We show that for any vertex $x$, its outgoing link is \correction{combinatorially} connected. 
    Indeed, take two edges $e,e'$ of $X$ with source $x$, and consider their extensions to oriented paths $\gamma, \gamma'$ from $x$ to $\sk(X)$. 
    By hypothesis, these two paths are combinatorially homotopic, that is, there is a sequence of parallel oriented paths from $\gamma$ to $\gamma'$. 
    The collection of first edges in each of these paths defines a path between $e$ and $e'$ in the outgoing link of $x$. 
    Thus, this link is connected. 
\end{proof}

\begin{thm}
\label{p:second-proof}
    Let $X$ be a polyhedral complex, and let $\vec v$ be generic on the edges of $X$.
    \correction{Suppose that the outgoing link of every vertex is combinatorially connected.}
    Then, any two parallel combinatorial paths on $X$ are combinatorially homotopic.
\end{thm}

\begin{proof} 
    \correction{Adapt proof}
    By \cref{l:oriented}, the conclusion holds for \emph{oriented} paths.  
    Let us show that this implies the non-oriented version.
    Let $\gamma$ be a (non-oriented) combinatorial path on $X$ between $x$ and $y$.
    For every vertex $z$ along $\gamma$, one can choose an oriented path $\delta_z$ from $z$ to $\sk(X)$. 
    We observe that for any edge $e: z \to z'$ of $\gamma$, the oriented paths $\delta_z$ and $\delta_{z'}e$ are combinatorially homotopic by hypothesis. 
    Going from $x$ to $y$ inductively one edge at a time and using transitivity of the homotopy equivalence relation, one obtains that $\gamma$ is combinatorially homotopic to $\delta_y^{-1}\delta_x$. 
    Taking another combinatorial path $\gamma'$ parallel to $\gamma$, the same argument shows that $\gamma'$ is combinatorial homotopic to $\delta_y^{-1}\delta_x$.
    Thus $\gamma$ and $\gamma'$ are combinatorially homotopic, which completes the proof. 
\end{proof}

As \cref{lemma:outgoing-link} shows, the class of polyhedral complexes to which \cref{p:second-proof} applies is a strict subclass of simply connected complexes.
This implies that the converse of \cref{p:second-proof} does not hold, and thus that Mac Lane's original proof is far from reaching the full generality of \cref{thm:top-coherence}.
However, it will be sufficient for our purposes, presented in the next Section, since as we have seen in \cref{prop:polytopes} it applies to any polytope.
\correction{Moreover, it has an associated rewriting system...}


%%%%%%%%%%%%%%%%%%%%%%%%%%%%%%%%%%

\subsection{Rewriting systems}

\correction{
    We refer to \cite{baaderTermRewritingAll1998} for more details on rewriting systems. 
\begin{definition}
    An \defn{abstract rewriting system} is a set $A$ together with a binary relation $\to$. 
\end{definition}
We denote by $\xrightarrow{*}$ the reflexive and transitive closure of ${\to}$. 
We say that $(A,\to)$ is \defn{locally confluent} (resp.\ \defn{confluent}) if for all $a,a_1,a_2$ such that $a_1 \leftarrow a \to a_2$ (resp.\ $a_1 \xleftarrow{*} a \xrightarrow{*} a_2$), there exists a term $b$ with $a_1 \xrightarrow{*} b\; \xleftarrow{*} a_2$.  
A rewriting system is \defn{terminating} if every reduction sequence $a \to a_1 \to a_2 \to \cdots$ eventually must terminate.}

\correction{
    Given a polyhedral complex $X$ and a generic vector $\vec v$, one can consider the abstract rewriting system defined by $\vec v$ on the vertices of $X$.
\begin{definition}
    The \defn{vertices rewriting system} is the pair $(X_0,\to)$ made of the set of vertices $X_0$ of $X$, together with the following relation: we have $x \to y$ if $x$ and $y$ are vertices of the same edge and $\langle v, x \rangle < \langle v, y \rangle$.
\end{definition}
According to this definition, we have $x \xrightarrow{*} y$ if and only if there is an oriented path from $x$ to $y$ in~$X_1$. 
The hypotheses of \cref{p:second-proof} impose that the rewriting system $(X_0,\to)$ is terminating and confluent.
\begin{proposition}
    If there is a unique sink, then $(X_0,\to)$ is terminating and confluent. 
\end{proposition}
\begin{proof}
    Since $\vec v$ is generic, and thus strictly increasing along edges, it defines a partial order, and since the set $X_0$ is finite, the rewriting system $(X_0,\to)$ is terminating.
    Confluence follows from the existence of a unique sink $\sk(X)$: given any pair of vertices $x,y$, since $\vec v$ is generic there are oriented paths $x \xrightarrow{*} \sk(X) \xleftarrow{*} y$. 
\end{proof}
\begin{corollary}
    The abstract rewriting system on any polytope is terminating and confluent.
\end{corollary}
\begin{proof}
    The existence and uniqueness of a sink is one of the basic, very useful facts about polytopes, see \cite[Theorem 3.7]{Ziegler95}.
\end{proof}
This definition is fairly general, and in practice one considers \emph{term} rewriting systems, which possess much more structure, see \cite{CLA24}.
Aussi, on ne peut pas aller dans l'autre sens; pas necessairement de forme normale unique etc. Polytope SIMPLE est condition necessaire dans l'autre sens}

%Terminating and confluent
%Rewriting rules
%Critical pairs