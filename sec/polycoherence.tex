% !TEX root = ../Coherence.tex

\section{Polytopal coherence theorem} 
\label{s:polycoherence}

\Guillaume{Add figures}

Let $P$ be polytope.
A (non-oriented) \emph{path} on $P$ is the cellular image of a continuous, cellular, injective map $\gamma : [0,1] \to P$. 
The two vertices $\gamma(0)$ and $\gamma(1)$ are called the \emph{endpoints} of the path. 
Two paths $\gamma_1$ and $\gamma_2$ are \emph{parallel} if they have the same endpoints. 

Let $\gamma$ be a path on a $2$-dimensional polytope. 
We observe that there is a unique path $\gamma^\op \neq \gamma$ on $F$ which is parallel to $\gamma$. 

\begin{definition}[Discrete homotopy]
    Two parallel paths are \emph{elementary homotopic} is they differ only on a $2$-face of $P$. 
    More generally, two parallel paths are \emph{homotopic} if they are related by a finite number of elementary homotopies.
\end{definition}

\begin{thm}
    \label{t:polytopal-coherence}
    Any two parallel paths on a polytope are homotopic. 
\end{thm}

\begin{proof}
    Let $\gamma$ and $\gamma'$ be two parallel paths on $P$, denote by $v$ one of their common endpoints, and denote by $e$ and $e'$ the edges of $\gamma$ and $\gamma'$ which contain $v$. 
    We let $m$ denote the maximal length of a path parallel to $\gamma$ and $\gamma'$, and we let $n$ denote the minimal length of a path between between $e$ and $e'$ in the vertex figure of $P$ at $v$ \cite[Section 2.1]{Ziegler95}.
    We proceed by lexicographic induction on $(m,n)$. 
    If $e=e'$, then we can apply the induction hypothesis to the two parallel paths $\gamma - e$ and $\gamma' - e'$ (in this case, $m$ diminishes by $1$ and $n=0$ is constant). 
    If $e \neq e'$, we consider the family $\mathcal{F}$ of $2$-faces defined by a path of minimal length from $e$ to $e'$ in the vertex figure. 
    We write $\gamma = \gamma_1 \gamma_2$, where $\gamma_1$ is the restriction of $\gamma$ to the $2$-face in $\mathcal{F}$ containing $e$. 
    We consider the path $\gamma''=\gamma_1^\op \gamma_2$, which is parallel to $\gamma$ and $\gamma'$, and we apply to it the induction hypothesis (in this case, $m$ is constant and $n$ diminishes by $1$), which finishes the proof. 
\end{proof}

One is often interested in a \emph{coherent families} of polytopes. 
We will say that a family of polytopes $P_\bullet = \{P_i\}_{i \in I}$ is \emph{$n$-coherent} if every $n$-face of a polytope in the family is combinatorially isomorphic to the product of polytopes in the family. 
A family of polytopes is \emph{coherent} if it is $n$-coherent for all $n \geq 0$.

\begin{example}
    Any family of realizations of the associahedra is coherent. 
    The family of standard permutahedra is coherent. 
    More generally, any choice of realizations of graph-associahedra are coherent. \Guillaume{Refs}
\end{example}

We will be interested in $2$-coherence of certain families of polytopes. 

\begin{proposition}
    Hypergraph polytopes are $2$-coherent. 
\end{proposition}

\begin{proof}
    Case by case
\end{proof}

\begin{corollary}
    \label{c:operahedra-2-coh}
    Operahedra and associahedra are $2$-coherent.
\end{corollary}

\begin{proof}
    Case by case
\end{proof}

\begin{thm}
    Coherence theorem for categorified ns operads
\end{thm}

\begin{proof}
    Follows \cref{t:polytopal-coherence} and \cref{c:operahedra-2-coh}.
\end{proof}

\begin{corollary}
    MacLane's coherence theorem for non-unital non-symmetric monoidal categories
\end{corollary}

\begin{rem}
    What about the unital case? 
    What about the symmetric case? --
\end{rem}

\Guillaume{$n$-coherence related to $n$-categories instead of simply categories??!! A commencer par les 2-cat; le theoreme de coherence ne tient plus pour les n plus grands!!}

\subsection{Other examples}

\subsection{Weak Cat-operads}


