% !TEX root = ../Coherence.tex

\section{Topological coherence theorem} 
\label{s:polycoherence}

Let $X$ be a regular CW complex. 
A \emph{combinatorial path} on $X$ is the cellular image of a continuous and cellular map $\gamma : [0,1] \to X$. 
The two vertices $\gamma(0)$ and $\gamma(1)$ are called the \emph{endpoints} of the path. 
Two paths $\gamma_1$ and $\gamma_2$ are \emph{parallel} if they share the same endpoints. 

A combinatorial path $\gamma$ is called \emph{regular} if it is either constant, or injective on the open interval $]0,1[$.
We observe that for $\gamma$ a regular combinatorial path on a $2$-cell $A$ of $X$, there is a unique parallel regular combinatorial path $\gamma^{\op}\neq \gamma$ on $A$. 
We say that $\gamma$ and $\gamma^{\op}$ are related by a \emph{flip}.

\begin{definition}
    Two parallel combinatorial paths are \emph{elementary combinatorially homotopic} is they differ only by a flip on a $2$-face of $X$. 
    More generally, two parallel combinatorial paths are \emph{combinatorially homotopic} if they are related by a finite number of elementary combinatorial homotopies.
\end{definition}

We say that $v \in \R^n$ \emph{orients} $X$, or is an \emph{orientation vector} for $X$, if $v$ is not perpendicular to any edge of $X$.
We say that an oriented polyhedral complex $(X,v)$ is \emph{flow link connected} if the set of outgoing edges of each vertex is connected by $2$-faces. 
In other words, $X$ is flow link connected if for every vertex $x$, and for any pair of outgoing edges $e, e'$ at $x$, there is a ordered sequence of outgoing edges $e=e_0, e_1, \ldots, e_n=e'$ at $x$ such that for all $1\leq i \leq n-1$, the two edges $e_i$ and $e_{i+1}$ are both faces of a $2$-face of $X$.  

\begin{thm}
    \label{t:polytopal-coherence}
    Any two parallel combinatorial paths on a flow link connected polyhedral complex $(X,v)$ are combinatorially homotopic. 
\end{thm}

\begin{proof}
    We proceed in two steps. 
    \begin{enumerate}
        \item We first reduce to the case of \emph{oriented} paths. 
        \item Now let us prove the case of oriented paths. 
    \end{enumerate}
    %Let $\gamma$ and $\gamma'$ be two parallel paths on $P$, denote by $v$ one of their common endpoints, and denote by $e$ and $e'$ the edges of $\gamma$ and $\gamma'$ which contain $v$. 
    %We let $m$ denote the maximal length of a path parallel to $\gamma$ and $\gamma'$, and we let $n$ denote the maximal length of a path between between $e$ and $e'$ in the vertex figure of $P$ at $v$ \cite[Section 2.1]{Ziegler95}.
    %We proceed by lexicographic induction on $(m,n)$. 
    %If $e=e'$, then we can apply the induction hypothesis to the two parallel paths $\gamma - e$ and $\gamma' - e'$ (in this case, $m$ diminishes by $1$ and $n=0$ is constant). 
    %If $e \neq e'$, we consider the family $\mathcal{F}$ of $2$-faces defined by a path of minimal length from $e$ to $e'$ in the vertex figure. 
    %We write $\gamma = \gamma_1 \gamma_2$, where $\gamma_1$ is the restriction of $\gamma$ to the $2$-face in $\mathcal{F}$ containing $e$. 
    %We consider the path $\gamma''=\gamma_1^\op \gamma_2$, which is parallel to $\gamma$ and $\gamma'$, and we apply to it the induction hypothesis (in this case, $m$ is constant and $n$ diminishes by $1$), which finishes the proof. 
\end{proof}

\begin{corollary}
    Any two combinatorial parallel paths on a polytope are combinatorially homotopic. 
\end{corollary}

\begin{proof}
    It suffices to show that any oriented polytope is flow link connected. 
    Examine the vertex figure...
\end{proof}

In fact, the preceding theorem holds in greater generality. 



Let $X$ be a finite CW-complex. 

\begin{thm}
    Any two parallel paths on $X$ are combinatorially homotopic if and only if $X$ is simply connected.
\end{thm}

\begin{proof}
    This is an instance of Seifert--Van Kampen. 
\end{proof}


\Guillaume{Add figures}

