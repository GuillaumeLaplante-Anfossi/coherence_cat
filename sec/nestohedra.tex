% !TEX root = ../Coherence2.tex

\section{Hypergraph polytopes} 
\label{s:hypergraph}

In this section, we recall the definition of hypergraph polytopes. 
We refer to \cite{DP-HP,COI} for more details. 

%%%%%%%%%%%%%%%%%%%%%%%%%%%%%%%%%%%%%%%%%%%

\subsection{Hypergraphs}
A \defn{hypergraph} is given by a finite set $H$ of \defn{vertices} and a subset of \defn{hyperedges} $\hyper{H}\inc {\cal P}(H)\setminus\emptyset$ such that $\Union \hyper{H}=H$. 
We always assume that $\hyper{H}$ is \defn{atomic}, that is $\set{x}\in \hyper{H}$, for all $x\in H$. 
A hyperedge of cardinality 2 is called an \defn{edge}.  
For $X\inc H$, the \defn{plain restriction} of $\hyper{H}$ to $X$ is the set 
$\hyper{H}_X := \setc{Z\in \hyper{H}}{\; Z\inc X}$.

We say that $\hyper{H}$ is \defn{connected} if there is no non-trivial partition $H=X_1\union X_2$ such that $\hyper{H}=\hyper{H}_{X_1}\union \hyper{H}_{X_2}$. 
For each hypergraph, there exists a partition $H=X_1\union\ldots\union X_m$ such that each $\hyper{H}_{X_i}$ is connected and $\hyper{H}=\Union(\hyper{H}_{X_i})$.  
The $\hyper{H}_{X_i}$'s are called the \defn{connected components} of $\hyper{H}$.
For $X\inc H$, we say that a non-empty subset $X$ of vertices is \defn{connected} (resp. a \defn{connected component}) whenever $\hyper{H}_X$ is connected (resp. a connected component of $\hyper{H}$).  
We denote by $\restrH{H}{X}:=\hyper{H}_{H\setminus X}$ the plain restriction of $\hyper{H}$ to $H \setminus X$.
The \defn{saturation} of $\hyper{H}$ is the hypergraph
$\Sat(\hyper{H})=\setc{X}{\emptyset\incs X\inc H\;\mbox{and}\;\hyper{H}_X\;\mbox{is connected}}$.
A hypergraph is called \defn{saturated} when $\hyper{H}=\Sat(\hyper{H})$.  
The \defn{reconnected restriction} of $\hyper{H}$ to $X$ is the set $$\recrestr{\hyper{H}}{X}:=\setc{Z\cap X}{Z\in \Sat(\hyper{H}), Z\cap X\neq\emptyset}.$$

\begin{rem}
    Atomic and saturated hypergraphs are called \defn{building sets} in the nestohedra literature, see for example \cite{P09,FS05}.
\end{rem}

For $X\inc H$, we will express the fact that $\setc{H_i}{i\in I}$ is the set of connected components of $\restrH{H}{X}$ by the notation $\hyper{H},X  \leadsto  \setc{H_i}{i\in I}.$
If $I=\set{1,\ldots,n}$, we write simply $\hyper{H},X  \leadsto H_1,\ldots,H_n$, with no order intended.
We will write $\hyper{H}_i$ for $\hyper{H}_{H_i}$.
In the specific situation where $x,y,z\in H$ and $\hyper{H},\set{x}\leadsto \setc{H_i}{i\in I}$, we shall write
$$\begin{array}{ll}
\xyz{x}{\hyper{H}}{\set{y,z}} & \mathrm{if}\; y,z\in H_i\; \mbox{for some}\; i \in I\\
\xyz{x}{\hyper{H}}{\set{y},\set{z}} & \mbox{otherwise}.
\end{array}$$
In the second case, we will say that $x$ \defn{disconnects} $y$ and $z$ in $\hyper{H}$. 
The reconnected restriction allows one to charaterize the preceding two situations as follows.

\begin{lemma} 
\label{xyz-reconnected} 
We have
$$\begin{array}{lll}
\xyz{x}{\hyper{H}}{\set{y,z}} & \mathrm{iff} & \recrestr{\hyper{H}}{\set{x,y,z}},\set{x}\leadsto\set{y,z} \\
\xyz{x}{\hyper{H}}{\set{y},\set{z}} & \mathrm{iff} & \recrestr{\hyper{H}}{\set{x,y,z}},\set{x}\leadsto \set{y},\set{z}.
\end{array}$$
\end{lemma}

\begin{proof} 
    Let $\hyper{H},x\leadsto H_1,\ldots,H_n$. 
    Suppose $\xyz{x}{\hyper{H}}{\set{y,z}}$. 
    Then there exists $i$ such that $\set{y,z}\inc H_i$, and hence $\set{y,z}\inc H_i\cap\set{x,y,z}$, and in fact $\set{y,z} = H_i\cap\set{x,y,z}$ since $x\not\in H_i$. 
    Thus $\recrestr{\hyper{H}}{\set{x,y,z}},\set{x}\leadsto\set{y,z}$ holds by definition of reconnected restriction.
    If $\xyz{x}{\hyper{H}}{\set{y},\set{z}}$, then there exist $i\neq j$ such that $y\in H_i$ and $z\in H_j$. 
    We then derive likewise $H_i\cap\set{x,y,z}=\set{y}$ and $H_j\cap\set{x,y,z}=\set{z}$ from which $ \recrestr{\hyper{H}}{\set{x,y,z}},\set{x}\leadsto \set{y},\set{z}$ follows.
\end{proof}

%%%%%%%%%%%%%%%%%%%%%%%%%%%%%%%%%%%%%%%%%%%%%%%%%%%%%%%

\subsection{Constructs}
A {connected} hypergraph $\hyper{H}$ gives rise to a set of \defn{constructs}, which are defined inductively as follows.
 
\begin{definition} 
\label{inductive-construct}
%as follows.
Let $\hyper{H}$ be a connected hypergraph and $Y$ be a non-empty subset of $H$.
\begin{enumerate}
\item  If $Y = H$, then the one-node tree $H$ decorated with $H$ is a construct of $\hyper{H}$.
\item If $\hyper{H},Y  \leadsto H_1,\ldots,H_n$, and if $T_1,\ldots,T_n$ are constructs of $\hyper{H}_1,\ldots,\hyper{H}_n$, respectively, then the
tree $Y(T_1,\ldots,T_n)$ whose root is decorated by $Y$, with $n$ outgoing edges on which the respective $T_i\,$'s are grafted is a construct.  
\end{enumerate}
When $Y={z}$ is a singleton, we freely write $z$ in place of $\set{z}$.
A \defn{construction} is a construct all of whose nodes are  decorated with singletons. 
\end{definition}

Since all decorations in a construct are disjoint, we freely identify nodes with subsets of $H$. 
We use the notation $\occ{T}{X}$ to denote the full subtree of $T$ rooted at $X$, defined only if $X$ is indeed a decoration of a node of $T$. 
If $U$ is a (not necessarily full) subtree of $T$, we denote by $\supp(U)$ the union of the decorations of the nodes of $U$.

\begin{rem} \label{subconstruct-restriction}
The intention behind this presentation is algorithmic: a construct is built by picking and removing a non-empty subset $Y$ of $H$, then branching to the connected components of $\restrH{H}{Y}$ and continuing inductively in all the branches.
It follows readily from the definition that $\occ{T}{X}$ is a construct of $\hyper{H}_{\supp(\occ{T}{X})}$.
\end{rem}

\begin{rem}
    The notion of construct is equivalent to the notion of nested set \cite{P09}, and to the notion of tubing in the case where $\hyper{H}$~\cite{CD-CCGA} is a graph.  
    We refer to \cite[Sec.~3.1]{COI} for details.
\end{rem}

%Here we content ourselves with a sketchy description of the
%dictionary (for the readers familiar with nested sets).
%Given a construct $T$, take the set $\setc{\supp(\occ{T}{X})}{X\:\textrm{is a node of}\: T}$. Conversely, read a construct from a nested set $\mathbb{T}$ as follows: for each  $Z\in\mathbb{T}$, consider the elements $Z_1,\ldots, Z_n\in\mathbb{T}$ that are maximal among all elements of $\mathbb{T}$ that are strictly included in $Z$, then $Z\setminus\bigcup(Z_1\cup \ldots \cup Z_n)$ will decorate a node of the corresponding construct. 
%The subface relation formulated in terms of nested sets is just the inclusion of nested sets.

If $X,Y$ are two nodes of a construct $S$ of $\hyper{H}$, $X$ being the father of $Y$, we can define a new construct $T$ by contracting the edge between $X$ and $Y$, and labeling the resulting vertex of~$T$ by the union of the labels of $X$ and $Y$. 

\begin{definition}
    We denote $({\cal A}(\hyper{H}),\preceq)$ the poset of constructs of a connected hypergraph $\hyper{H}$ obtained as the reflexive and transitive closure of the following covering relations: a construct $S$ covers a construct $T$ if $T$ can be obtained from $S$ by contracting an edge.
\end{definition}

%%%%%%%%%%%%%%%%%%%%%%%%%%%%%%%%%%%%%%%%%%%%%%

\subsection{Hypergraph polytopes}

We are now ready to define hypergraph polytopes, a.k.a nestohedra.

\begin{definition}
    A \defn{hypergraph polytope} is a polytope whose face lattice is isomorphic to the poset of constructs of some connected hypergraph $\hyper{H}$.
\end{definition}

Do\v sen and Petri\'c gave polytopal realisations of hypergraph polytopes in ~\cite{DP-HP}.
The idea is that the connected subsets of $\hyper{H}$ specify the faces of a fixed $(|H|-1)$-dimensional simplex, that are to be truncated according to the constructs of $\hyper{H}$.

\begin{rem}
    Quite different geometric realisations of nestohedra were given in \cite{P09}. 
\end{rem}

\begin{example}
    Our key basic examples of hypergraphs form the following ``quatuor'':
$$\begin{array}{lll}
\hyper{S}^n=\set{\set{1},\ldots,\set{n},\set{1,\ldots,n}} \\
\hyper{C}^n= \set{\set{1},\ldots,\set{n},\set{1,2},\ldots,\set{1,2,\ldots,i},\ldots,\set{1,\ldots,n}}\\
\hyper{K}^n= \set{\set{1},\ldots,\set{n},\set{1,2},\ldots,\set{i-1,i},\ldots,\set{n-1,n}}\\
\hyper{P}^n=\setc{X\inc\set{1,\ldots,n}}{1\leq |X|\leq 2}.
\end{array}$$
Their geometric realisations are the $(n-1)$-dimensional simplex,  hypercube,  associahedron, and permutohedron, respectively. 
Note that $\hyper{K}_n$ and $\hyper{P}_n$ are graphs while $\hyper{S}_n$ and $\hyper{C}_n$ are genuine hypergraphs. 
Note also that the definition of $\hyper{S}_n$ and $\hyper{P}_n$ does not depend on any order on the vertices, while the definition of $\hyper{C}_n$ and $\hyper{K}_n$ involves the total (or linear) order on $\set{1\ldots,n}$. 
Another way to say this is that we can replace $\set{1,\ldots,n}$ by any finite set (resp. finite linearly ordered set) and define $\hyper{S}^X$ and $\hyper{P}^X$ (resp. $\hyper{C}^X$, $\hyper{K}^X$).
\end{example}

Many more examples of  hypergraph polytopes are to be found in~\cite{DP-HP,COI,CDOO}, as well as in the abundant literature on nestohedra.


