% !TEX root = ../Coherence2.tex

\section{Anatomy of the 2-skeleton} 
\label{s:anatomy}

In this section we describe all the possible $2$-faces of a hypergraph polytope.
These will be associated with the local confluence diagrams of a rewrite system on constructions in \cref{ss:critical}.

%%%%%%%%%%%%%%%%%%%%%%%%%%%%%%%%%%%%%%%%%%%

\subsection{Two types of two-faces} 
\label{two-types}
The \defn{dimension} of a construct $T$, or equivalently of its corresponding face in the associated hypergraph polytope, is given by
$$\dim T \eqdef \sum_{X\:\mathrm{node\: of}\: T}(|X|-1).$$
In particular, constructions have dimension $0$. 
Constructs of dimension $1$ have a single non-singleton node of the form $\set{x,y}$. 
Constructs $T$ of dimension $2$ are of two kinds:
\begin{itemize}
\item[(A)]  $T$ has exactly two non-singleton nodes $\set{x_1,x_2}$ and $\set{y_1,y_2}$, both of cardinal $2$. 
\item[(B)] $T$ has exactly one non-singleton node $\set{x_1,x_2,x_3}$ of cardinal $3$.
\end{itemize}

\subsubsection*{Type (A)}
\label{ss:typeA}
If $T$ is of type (A), we get the following generic picture.
$$ 
\xymatrix @-1.65pc { x_1(x_2)\cdots y_1(y_2) \ar @{-}[dddd]_{x_1(x_2)\cdots \set{y_1,y_2}} 
 \ar @{-}[rrrr]_{\set{x_1,x_2}\cdots y_1(y_2)} &&&& x_2(x_1)\cdots y_1(y_2) \ar @{-}[dddd]^{x_2(x_1)\cdots \set{y_1,y_2}} \\
 &&&&\\
 && \set{x_1,x_2} \cdots \set{y_1,y_2}&&\\
 &&&&\\
 x_1(x_2)\cdots y_2(y_1)  \ar @{-}[rrrr]_{\set{x_1,x_2}\cdots y_2(y_1)} &&  && x_2(x_1)\cdots y_2(y_1)}
 $$
The central construct $T$, schematised as $\set{x_1,x_2} \cdots \set{y_1,y_2}$, has two distinct nodes $\set{x_1,x_2}$ and $\set{y_1,y_2}$. 
All the other constructs are obtained by replacing in $T$ one or two of these nodes, say $\set{x_1,x_2}$, by a  tree $x_1(x_2)$ or $x_2(x_1)$ and redistributing the children of  $\set{x_1,x_2}$  as children of  $x_1$ or $x_2$, in a unique way dictated by connectivity. 

\subsubsection*{Type (B)}
\label{ss:typeB}
If $T$ is of type (B), then, up to permutation of $x_1,x_2,x_3$, we get four possible shapes corresponding to the number $N$ of  elements in $X\eqdef \set{x_1,x_2,x_3}$ that disconnect the other two in~$\hyper{K}\eqdef \hyper{H}_{\supp(\occ{T}{X})}$.
Here there are no $\cdots$ on the picture, but likewise all the edges and vertices of $T$, considered as a 2-face, are the result of replacing in $T$ the node $X$ with the indicated respective trees and redistributing uniquely the  children of $X$ (see also~\cref{instance-construct}). 
 
\smallskip\noindent
(B1)  When $N=3$, that is when $\xyz{x_1}{\hyper{K}}{\set{x_2}\!,\! \set{x_3}}$, $\xyz{x_2}{\hyper{K}}{\set{x_1}\!,\! \set{x_3}}$, and
 $\xyz{x_3}{\hyper{K}}{\set{x_1}\!,\! \set{x_2}}$, we have
 
 $$\xymatrix @-1.65pc {&& x_1(x_2,x_3) \ar @{-}[ddll]_{\set{x_1,x_2}(x_3)} \ar @{-}[ddrr]^{\set{x_1,x_3}(x_2)}&& \\
 &&&&\\
 x_2(x_3,x_1) \ar @{-}[rrrr]^{\set{x_2,x_3}(x_1)} &&&& x_3(x_1,x_2)}$$

\smallskip\noindent
(B2) When $N=2$, that is when $\xyz{x_1}{\hyper{K}}{\set{x_2}\!,\! \set{x_3}}$, $\xyz{x_2}{\hyper{K}}{\set{x_1,x_3}}$ and
 $\xyz{x_3}{\hyper{K}}{\set{x_1}\!,\! \set{x_2}}$, we have
 
 $$ \xymatrix @-1.65pc {&& x_1(x_2,x_3) \ar @{-}[ddll]_{\set{x_1,x_2}(x_3)} \ar @{-}[ddrr]^{\set{x_1,x_3}(x_2)}&& \\
 &&&&\\
 x_2(x_1(x_3)) \ar @{-}[ddrr]_{x_2(\set{x_1,x_3})}&& \set{x_1,x_2,x_3}  && x_3(x_1,x_2)\ar @{-}[ddll]^{\set{x_2,x_3}(x_1)}\\
 &&&&\\
 &&  x_2(x_3(x_1))&&}$$

\smallskip\noindent
(B3) When $N=1$, that is when $\xyz{x_1}{\hyper{K}}{\set{x_2}\!,\! \set{x_3}}$, $\xyz{x_2}{\hyper{K}}{\set{x_1,x_3}}$ and
 $\xyz{x_3}{\hyper{K}}{\set{x_1,x_2}}$, we have
 
$$
 \xymatrix @-1.65pc {&& x_1(x_2,x_3) \ar @{-}[ddll]_{\set{x_1,x_2}(x_3)} \ar @{-}[ddrr]^{\set{x_1,x_3}(x_2)}&& \\
 &&&&\\
 x_2(x_1(x_3)) \ar @{-}[ddr]^{x_2(\set{x_1,x_3})}&& \set{x_1,x_2,x_3}  && x_3(x_1(x_2))\ar @{-}[ddl]_{x_3(\set{x_1,x_2})}\\
 &&&&\\
 &x_2(x_3(x_1))\ar @{-}[rr]^{\set{x_2,x_3}(x_1)}&  & x_3(x_2(x_1))&}
$$

\smallskip\noindent
(B4) When $N=0$, that is when $\xyz{x_1}{\hyper{K}}{\set{x_2,x_3}}$, $\xyz{x_2}{\hyper{K}}{\set{x_1,x_3}}$ and
 $\xyz{x_3}{\hyper{K}}{\set{x_1,x_2}}$, we have
 
 $$\xymatrix @-1.65pc {&& x_1(x_2(x_3)) \ar @{-}[ddll]_{\set{x_1,x_2}(x_3)} \ar @{-}[ddrr]^{x_1(\set{x_2,x_3})}&& \\
 &&&&\\
 x_2(x_1(x_3)) \ar @{-}[dd]^{x_2(\set{x_1,x_3})}&& \set{x_1,x_2,x_3} && x_1(x_3(x_2))\ar @{-}[dd]_{\set{x_1,x_3}(x_2)}\\
 &&  &&\\
 x_2(x_3(x_1)) \ar @{-}[ddrr]_{\set{x_2,x_3}(x_1)} &&  && \ar @{-}[ddll]^{x_3(\set{x_1,x_2})} x_3(x_1(x_2))\\
 &&&&\\
 &&  x_3(x_2(x_1))&&}
$$

By~\cref{xyz-reconnected}, we can read those pictures as describing the 
respective reconnected restrictions $\recrestr{\hyper{K}}{X}$ of $\hyper{K}$:
$$\begin{array}{lll}
\mathrm{(B1)} & \set{\set{x_1},\set{x_2},\set{x_3},\set{x_1,x_2,x_3}} & \mbox{(2-simplex)}\\
\mathrm{(B2)} & \set{\set{x_1},\set{x_2},\set{x_3},\set{x_1,x_3},\set{x_1,x_2,x_3}} & \mbox{(2-cube})\\
\mathrm{(B3)} & \set{\set{x_1},\set{x_2},\set{x_3},\set{x_1,x_3},\set{x_1,x_2},\set{x_1,x_2,x_3}} & \mbox{(2-associahedron)}\\
\mathrm{(B4)} & \set{\set{x_1},\set{x_2},\set{x_3},\set{x_1,x_3},\set{x_1,x_2},\set{x_2,x_3},\set{x_1,x_2,x_3}} & \mbox{(2-permutahedron)}
\end{array}$$

Incidentally, these  hypergraphs witness the fact that there do exist 2-faces of type (B) of  each  of these types: take $\hyper{H}$ to be one of those four hypergraphs, and $T$ to be their unique construct of dimension 2.

%%%%%%%%%%%%%%%%%%%%%%%%%%%%%%%%%%%%%

\subsection{$X$-faces and shapes}

In the rest of this section, we make the discussion above on $\cdots$ more formal.
For a 3-element subset $X=\set{x_1,x_2,x_3}$ of $H$, we say that a 2-face $T$ of $\hyper{H}$ is an \defn{$X$-face} if its unique non-singleton node is decorated by $X$.  By letting $X$ range over all subsets of $H$ of cardinal 3, we form in this way a partition of all 2-faces of type (B).
If $X$ is the root of $T$, the following  lemma invites us to see $T=X(\ldots)$ as an ``instantiation'' of~$X$ (viewed as the maximum face of $\recrestr{\hyper{H}}{X}$), as we shall see.

\begin{lemma} 
  \label{instance-construct} 
  If $\hyper{H}$ is a connected hypergraph, if $X$ is a subset of $H$ such that $|X|=3$ and~$T$ is a 2-dimensional construct with root $X$, then the map $\psi$ from the poset of  faces of $T$ to the poset of faces of 
$\recrestr{\hyper{H}}{X}$ defined on a face $S$  by pruning all nodes of $S$ that are not subsets of $X$ is an order-isomorphism.
\end{lemma}

\begin{proof}
We shall build an inverse $\phi$ of $\psi$.
When discussing type (B) 2-faces, we have described up to permutation all the possible connected hypergraphs on the set $X$ of vertices and their respective posets of faces. 
We will treat the case where ${\cal A}(\recrestr{\hyper{H}}{X})$ is the poset described in case (B2). The other cases are similar.
Pick the 0-face~$S\eqdef  x_1(x_2,x_3)$. 
We map $S$ to a $0$-face $\phi(S)$ of $T$ as follows. 
Let $\hyper{H},x\leadsto \{H_1,\ldots,H_n\}$. 
By \cref{xyz-reconnected}, we have $\xyz{x_1}{\hyper{H}}{\set{x_2},\set{x_3}}$, hence we have, say $x_2\in H_1$ and $x_3\in H_2$. Then let
  $\hyper{H_1},x_2\leadsto \{H_{1,1},\ldots,H_{1,p}\}$ and $\hyper{H_2},x_3\leadsto \{H_{2,1},\ldots,H_{2,q}\}$.
  Then we have $\hyper{H},X\leadsto \{H_{1,1},\ldots,H_{1,p},H_{2,1},\ldots,H_{2,q},H_3,\ldots H_n\}$, so that $T$ writes as
  $$T= X(T_{1,1},\ldots,T_{1,p},T_{2,1},\ldots,T_{2,q},T_3,\ldots T_n).$$ 
All these data determine uniquely a 0-dimensional subface of $T$, namely
  $$\phi(S)\eqdef x_1(x_2(T_{1,1},\ldots,T_{1,p}),x_3(T_{2,1},\ldots,T_{2,q}),T_3,\ldots T_n).$$
It is plain that $\psi(\phi(S))=S$.
  The same applies to all other 0-dimensional (resp. 1-dimensional) faces of $\recrestr{\hyper{H}}{X}$, establishing $\phi$ as a bijection, which is also easily seen to be monotonic: for example, we have
 $$\phi(\set{x_1,x_2}(x_3))= \set{x_1,x_2}(x_3(T_{2,1},\ldots,T_{2,q}),T_{1,1},\ldots,T_{1,p},T_3,\ldots T_n),$$
 evidencing $\phi(x_1(x_2,x_3))\preceq \phi(\set{x_1,x_2}(x_3))$. 
 The map $\psi$ is also monotonic, since the above pruning does not affect the place where the contraction occurs -- e.g., the edge of $\phi(x_1(x_2,x_3))$ that is contracted to get $\phi(\set{x_1,x_2}(x_3))$ is the edge between $x_1$ and $x_2$, which can thus be contracted in the preimage $x_1(x_2,x_3)$ to yield the preimage $\set{x_1,x_2}(x_3)$.
 Finally, we set
 $$\phi(\set{x_1,x_2,x_3})\eqdef \set{x_1,x_2,x_3}(T_{1,1},\ldots,T_{1,p},T_{2,1},\ldots,T_{2,q},T_3,\ldots T_n)=T,$$
 which concludes the proof.
 \end{proof}

 \begin{corollary} 
  \label{instance-construct-general} 
  If $\hyper{H}$ is a connected hypergraph, if $X$ is a subset of $H$ such that $|X|=3$ and $T$ is an $X$-face of $\hyper{H}$, then the poset of faces of $T$ is isomorphic to the poset of faces of~$\recrestr{(\hyper{H}_{\supp(\occ{T}{X})})}{X}$.
\end{corollary}

\begin{proof} This is an immediate consequence of the previous Lemma and of the observation that the poset of faces of $T$ is isomorphic to the poset of faces of $\occ{T}{X}$.
\end{proof}
 
Based on this Corollary, we shall say, for any $X$-face $T$, that $T$ has \defn{shape} 
$\recrestr{(\hyper{H}_{\supp(\occ{T}{X})})}{X}$.





