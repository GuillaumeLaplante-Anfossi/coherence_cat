% ----------------------------------------------------------------
% AMS-LaTeX Paper ************************************************
% **** -----------------------------------------------------------
\documentclass[10pt]{amsart}
\usepackage{graphicx, mathabx, amssymb,amsfonts,amsmath,amsthm,newlfont}
\usepackage{epsfig,url}
\usepackage{enumerate}
\usepackage[colorlinks=true,linkcolor=red,citecolor=blue]{hyperref}
\usepackage[dvipsnames]{xcolor}
\usepackage{color}

\input{aux/commands}

% ----------------------------------------------------------------

\def\abovespace{\vspace{12pt}}
\def\belowspace{\vspace{8pt}}



\addtolength{\hoffset}{-0.0in} \addtolength{\textwidth}{0in}
\addtolength{\voffset}{-0.0in} \addtolength{\textheight}{0.0in}


% -----------------------------------------------------------------

\title{Topological proofs of categorical coherence}

\author{Pierre-Louis Curien}
\address{IRIF, Universit\'e Paris Diderot and $\pi r^2$ team, Inria, France.}
\email{curien@irif.fr}

\author{Guillaume Laplante-Anfossi}
\address{School of Mathematics and Statistics, Universiy of Melbourne, Victoria, Australia.}
\email{guillaume.laplanteanfossi@unimelb.edu.au}

\date{\today}

\subjclass[2010]{Primary 18F99} 

\keywords{Categorified operads, categorical coherence, Seifert--Van Kampen theorem, MacLane coherence theorem.}

\dedicatory{"We shall construct $KP_n$, as a CW-complex, in Section 2 and show that it is an $(n-1)$-ball. This gives an instant one-step proof of MacLane's theorem in full generality."  \\ --  Mikhail M. Kapranov}

\thanks{The second author was supported by the European Union's Horizon 2020 research and innovation program under the Marie Sklodowska-Curie grant agreement No 754362 \inlinegraphics{EU.png}, by the Natural Sciences and Engineering Research Council of Canada (NSERC) and by the ANR-20-CE40-0016 Higher Algebra, Geometry and Topology.}


\begin{document}

\begin{abstract}
In this note, we give a short topological proof of coherence for categorified non-symmetric operads by using the fact that the diagrams involved are the $1$-skeleton of simply connected CW complexes. 
In particular, we obtain a "one-step" topological proof of MacLane's coherence theorem, as suggested by Kapranov in 1993. 
We use the same method to deduce other categorical coherence results and discuss possible generalisations to higher categories. 
\end{abstract}

\maketitle

\setcounter{tocdepth}{1}
%\tableofcontents

% !TEX root = ../Coherence.tex

\section*{Introduction} 
\label{s:introduction}

The $n$-dimensional permuto-associahedron, a CW-complex whose faces are in bijection with parenthesized \correction{ordered partitions} of $n+1$ letters, was first introduced by M. Kapranov in his study of higher dimensional Yang--Baxter equations, through the moduli spaces of curves $\overline{\mathcal{M}_{0,n+1}}(\R)$ and the solutions of the Knizhnik--Zamolodchikov equation \cite{kapranov1993}.
It was later realized as a convex polytope by V. Reiner and G. M. Ziegler \cite{reinerCoxeterassociahedra1994}, and more recently through the nested braid fan in \cite{CastilloLiu21}.
%as a simple polytope in \cite{baralicSimplePermutoassociahedron2019} and .

The present note stems from a desire to understand the epigraph, taken from the introduction of \cite{kapranov1993}: what is the precise relationship between the permuto-associahedron and Mac Lane's coherence theorem for symmetric monoidal categories? 
We show that the \emph{simple connectedness} of the former implies the latter, thereby refining and proving Kapranov's claim (see \cref{thm:coherence-MacLane}). 

This is done through a general ``topological coherence theorem" which applies to any simply connected, regular CW complex (\cref{thm:top-coherence}).
\correction{We apply this theorem to another family of polytopes, the operahedra, which encode categorified non-symmetric operads,~\cite{DP15,curienSyntacticAspectsHypergraph2019a,laplante-anfossiDiagonalOperahedra2022a}, yielding a ``one-step'' proof of the associated coherence theorem as well.}

\smallskip

\correction{There is little price to pay, though. For both theorems, we provide a precise bijective correspondence between the 1-skeleton (resp. the 2--cells) on the topological side,  and canonical morphisms (resp. bifunctoriality, naturality, and applications of coherence conditions) on the categorical side (\cref{{bijections-Kapranov},prop:bijection-nestings}). In the case of  permuto-associahedra, we note that their 2-skeleton corresponds to other basic canonical morphisms and coherence conditions than those of Mac Lane (hexagons and naturality of the involutive braiding on one hand versus dodecagons on the other hand), and that the equivalence between these presentations holds true but is  non-trivial (see~\cref{rem:Kapranov-to-MacLane}).  There is also yet another equivalent presentation (and hence yet another proof of coherence), due to Barali\'c, Ivanovi\'c and Petri\'c, that matches the 2-skeleton of a different polytope, which unlike the permuto-associahedron is simple~(see \cref{rem:simple-permutoassociahedron} and \cite{baralicSimplePermutoassociahedron2019}).
}

\smallskip
We also investigate a topological incarnation of Mac Lane's original argument, in the spirit of rewriting theory, see \cref{ss:abstract-rewriting}.  
\correction{We concentrate on  a certain family of simply connected polyhedral complexes endowed with an orientation vector:  the ones \correction{whose outgoing links are all connected} (\cref{p:second-proof}). We show that the orientation vector induces a terminating and confluent rewriting system on  their $1$-skeleton and} obtain  from there another general prood of coherence. In particular, this second theorem can be applied to all polytopes, allowing us to give a second, ``rewriting-theoretic'' proof of both previously mentioned coherence results.  \correction{In the case of operahedra, our rewriting proof simplifies the original proof of Do{\v s}en and Petri{\'c}~\cite{DP15} (see~\cref{rem:DPLA}).}

\correction{It is worth noting that, while general polyhedral complexes admit \emph{abstract} rewriting systems on their sets of vertices,
the families of operahedra  (that include associahedra, encoding non-symmetric monoidal categories) further admit  \emph{term} rewriting systems, which exhibit more structure and are the subject of a companion paper~\cite{CLA24}. In contrast, we shall argue that the abstract rewriting approach to \emph{symmetric} monoidal categories is not informative (see \cref{MacLane-Kapranov-Simple}).}

\smallskip
\correction{Using Morse \correction{theory} on affine cell complexes \cite{bestvinaMorseTheoryFiniteness1997},
 we relate our two approaches by showing that the second is (strictly) less general than the first (\cref{lemma:outgoing-link}).}
 
 \smallskip
Our two general topological coherence theorems can be used to prove other categorical results where polytopes appear, such as coherence for monoidal functors between monoidal categories \cite{epsteinFunctorsTensoredCategories1966}, see \cref{sec:further}.
\correction{They also shed light on some statements in the literature, such as the proof of \cite[Prop.~3.9]{KapranovVoevodsky94}, see\cref{sec:higher}.}
This all points towards further investigation of the relationship between $n$-categorical coherence and $n$-connectedness of appropriate spaces.
\correction{While topological proofs of $2$-categorical coherence already appeared in \cite{Gurski11}, higher dimensional results have been obtained recently by S. Barkan in the context of $\infty$-operads~\cite{barkanArityApproximationInfty2022}}, for which the present results could well be the strict, $n=1$ case.


% !TEX root = ../Coherence.tex

\section{Topological coherence} 
\label{s:polycoherence}

\subsection{Coherence \`a la Van Kampen}

Let $X$ be a regular CW complex, and let $X_k$, $k\geq 0$ denote its $k$-skeleton. 
Let $\mathcal{F}(X)$ be the groupoid with set of objects $X_0$ and morphisms spanned by the following set: for each $x \in X_0$, one identity morphism $\id_x : x \to x$; and for each $1$-cell $\alpha \in X_1$, one morphism $\alpha : x \to y$ oriented according to its attaching map, and one inverse morphism $\alpha^{-1} : y \to x$ in the opposite direction. 
In other words, $\mathcal{F}(X)$ is the free groupoid generated by the morphisms $\alpha$. 
A \emph{combinatorial path} on $X$ is a morphism in $\mathcal{F}(X)$, that is, a composable sequence of $\alpha$ and $\alpha^{-1}$ morphisms (a \emph{word} in $\alpha$ and $\alpha^{-1}$).
Two combinatorial paths $\gamma, \gamma' \in \mathcal{F}(X)(x,y)$ with the same endpoints are said to be \emph{parallel}.

The attaching map of a $2$-cell $A$ of $X$ defines a morphism $\gamma_A \in \mathcal{F}(X)(x,x)$ for a certain $x \in A_0$, given by the sequence of $1$-cells in its image.  
Two parallel combinatorial paths $\gamma, \gamma'$ are said to be \emph{elementary combinatorially homotopic} if they differ exactly by a relation of the form $\gamma_A = \id_x$, for some $2$-cell $A$.
That is, one can rewrite $\gamma$ into $\gamma'$ by replacing some (possibly empty) subword of $\gamma$ with an equivalent subword using the relation $\gamma_A = \id_x$.
More generally, two parallel combinatorial paths are \emph{combinatorially homotopic} if they are related by a sequence of elementary combinatorial homotopies.

%what about the inverses? il faut imposer la relation alpha-1 est inverse de alpha

\begin{thm}
\label{thm:top-coherence}
    Any two parallel combinatorial paths on $X$ are combinatorially homotopic if and only if every path component of $X$ is simply connected.
\end{thm}

\begin{proof}
    Let $\Pi(X)$ denote the fundamental groupoid of $X$, that is the groupoid with objects the vertices of $X$ and morphisms the homotopy classes of paths between them.
    Let $\mathcal{C}(X)$ denote the quotient of the groupoid $\mathcal{F}(X)$ by the relation ``being combinatorially homotopic". 
    Then, we have an isomorphism of groupoids \[ \Pi(X) \cong \mathcal{C}(X) \ . \]
    To show this, one can proceed in three steps. 
    First, one shows that the fundamental groupoid $\Pi(X_1)$ of the $1$-skeleton of $X$ is free on the homotopy classes of maps generated by the attaching maps of the $1$-cells, that is, free on the $\alpha$-morphisms \cite[9.1.5]{Brown2006}.
    Thus, one gets $\Pi(X_1) \cong \mathcal{F}(X)$. 
    Second, one shows that the fundamental groupoid $\Pi(X_2)$ of the $2$-skeleton of $X$ is the free groupoid $\Pi(X_1)$ modulo the relations $\gamma_A=1$, for $A$ a $2$-cell of $X$ \cite[9.1.6]{Brown2006}. 
    This is done through repeated application of the Seifert--Van Kampen theorem; one then has $\Pi(X_2) \cong \mathcal{C}(X)$.
    Third, one shows that the inclusion of $X_2$ in $X$ induces an isomorphism of fundamental groupoids $\Pi(X_2) \cong \Pi(X)$ \cite[9.1.7]{Brown2006}, which concludes the proof of the isomorphism $\Pi(X) \cong \mathcal{C}(X)$.
    The theorem then follows, since every path component of $X$ is simply connected if and only if its fundamental groupoid $\Pi(X)$ is trivial.  
\end{proof}

\subsection{Coherence \`a la Morse}

Let $X\subset \R^n$ be a polyhedral complex, and let $\vec v \in \R^n$ be \emph{generic} on the edges of $X$, meaning that for any pair of vertices $x,y \in X$ belonging to the same edge of $X$, we have $\langle \vec v , x \rangle \neq \langle \vec v, y\rangle$.  
Such a generic vector $\vec v$ induces a natural orientation on the edges of $X$, directed from the source vertex where the functional $\langle \vec v, - \rangle$ is minimal to the target vertex where it is maximal. 

In general, for any face $F \subset X$ of $X$, there is a unique \emph{source} vertex $\so(F)$ such that all its adjacent edges $e \subset F$ are outgoing, and a unique \emph{sink} vertex $\sk(F)$ whose adjacent edges are all incoming.
When the complex $X$ has a unique \emph{global sink}, a vertex whose adjacent edges $e \subset X$ are all incoming, we will denote it by $\sk(X)$. 

Let $H:=\{y \in \R^n \ | \ \langle \vec v , y \rangle = 0\}$ be the linear hyperplane orthogonal to $\vec v$.  
For every vertex $x \in X$, choose $\varepsilon >0$ such that the interval between $\langle \vec v , x \rangle$ and $\langle \vec v , x \rangle + \varepsilon$ does not contain the image of any other vertex under the ``height" function $\langle \vec v, - \rangle$. 

\begin{definition}
    The \emph{outgoing link} of a vertex $x \in X$ is the intersection $\mathcal{F} \cap (H+x+\varepsilon \vec v)$ of the family of faces $\mathcal{F}:=\{ F \subset X \ | \ \so(F)=x \}$ with the affine hyperplane $H+x+\varepsilon \vec v$. 
\end{definition}

A combinatorial path $\gamma$ on $X$ is \emph{oriented} if for any pair $(e, f)$ of consective edges in $\gamma$, we have that $\sk(e)=\so(f)$.  
When no ambiguity arises, we will omit the adjective ``combinatorial" and say only ``oriented path".
Two parallel oriented paths are said to be \emph{elementary combinatorially homotopic} if they are as non-oriented paths. 
They are \emph{combinatorially homotopic} if they are related by a sequence of elementary combinatorial homotopies between oriented paths. 

The following Lemma and its consequence \cref{p:second-proof} translate into topological terms the proof of \cite[Theorem 3.1]{MacLane63}.

\begin{lemma}
\label{l:oriented}
    Let $X$ be a polyhedral complex, and let $\vec v$ be generic on the edges of $X$. 
    If there is a unique global sink $\sk(X)$, then the outgoing link of every vertex is connected if and only if any two parallel oriented paths on $X$ are combinatorially homotopic.
\end{lemma}

\begin{proof}
    We prove the first implication. 
    Suppose that the outgoing link of every vertex is connected. 
    Let $\gamma$ and $\gamma'$ be two parallel oriented paths between two vertices $x$ and $y$. 
    We prove that they are combinatorially homotopic. 
    We proceed by induction on the maximal length $m$ of an oriented path between $x$ and $y$ in $X$. 
    Without loss of generality, we can suppose that $y=\sk(X)$, since if $y\neq\sk(X)$ we can always find an oriented path between $y$ and $\sk(X)$.
    The cases when $m=0$ and $m=1$ are trivial. 
    Suppose that the hypothesis holds up to $m=k-1, k\geq 2$, and consider two paths $\gamma$ and $\gamma'$ for which $m=k$. 
    Let $e$ and $e'$ denote the edges of $\gamma$ and $\gamma'$ that are adjacent to $x$. 
    We examine three cases.
    \begin{enumerate}
        \item If $e=e'$, we can apply the induction hypothesis to $\gamma \setminus e$ and $\gamma' \setminus e'$. 
        \item If $e \neq e'$ and both edges are on the same $2$-face $F$ of $X$, then using the induction hypothesis we have that $\gamma$ and $\gamma'$ are respectively combinatorially homotopic to the paths $\delta$ and $\delta'$ defined as follows: they go from $x=\so(F)$ to $\sk(F)$ by the unique path containing $e$ and $e'$, respectively, and then from $\sk(F)$ to $y$ along the same arbitrary oriented path. 
        Since $\delta$ and $\delta'$ are combinatorially homotopic by definition, the conclusion follows from the transitivity of the combinatorial homotopy equivalence relation. 
        \item Suppose that $e\neq e'$, and that $e$ and $e'$ are \emph{not} on the same $2$-face of $X$. 
        Since the outgoing link of $x$ is connected, there exists a path $\theta$ between $e$ and $e'$ in this link. 
        For every edge $e_i$ of $X$ in the path $\theta$, choose an oriented path $\gamma_i$ in $X$ from $x$ to $y=\sk(X)$ going through $e_i$. 
        Now apply Point (2) above to every pair of parallel oriented paths $(\gamma_i, \gamma_{i+1})$ with $e_i$ and $e_{i+1}$ consecutive in $\theta$, and conclude again by transitivity of the combinatorial homotopy equivalence relation. 
    \end{enumerate}

    In the other direction, suppose that every pair of parallel oriented combinatorial paths are combinatorially homotopic. 
    We show that for any vertex $x$, its outgoing link is connected. 
    Indeed, take two edges $e,e'$ of $X$ with source $x$, and consider their extensions to oriented paths $\gamma, \gamma'$ from $x$ to $\sk(X)$. 
    By hypothesis, these two paths are combinatorially homotopic, that is, there is a sequence of parallel oriented paths from $\gamma$ to $\gamma'$. 
    The collection of first edges in each of these paths defines a path between $e$ and $e'$ in the outgoing link of $x$. 
    Thus, this link is connected. 
\end{proof}

\begin{thm}
\label{p:second-proof}
    Let $X$ be a polyhedral complex, and let $\vec v$ be generic on the edges of $X$.
    Suppose that 
    \begin{enumerate}[label=\roman*)]
        \item there is a unique sink $\sk(X)$, and
        \item the outgoing link of every vertex is connected.
    \end{enumerate}
    Then, any two parallel combinatorial paths on $X$ are combinatorially homotopic.
\end{thm}

\begin{proof} 
    By \cref{l:oriented}, the conclusion holds for \emph{oriented} paths.  
    Let us show that this implies the non-oriented version.
    Let $\gamma$ be a (non-oriented) combinatorial path on $X$ between $x$ and $y$.
    For every vertex $z$ along $\gamma$, one can choose an oriented path $\delta_z$ from $z$ to $\sk(X)$. 
    We observe that for any edge $e: z \to z'$ of $\gamma$, the oriented paths $\delta_z$ and $\delta_{z'}e$ are combinatorially homotopic by hypothesis. 
    Going from $x$ to $y$ inductively one edge at a time and using transitivity of the homotopy equivalence relation, one obtains that $\gamma$ is combinatorially homotopic to $\delta_y^{-1}\delta_x$. 
    Taking another combinatorial path $\gamma'$ parallel to $\gamma$, the same argument shows that $\gamma'$ is combinatorial homotopic to $\delta_y^{-1}\delta_x$.
    Thus $\gamma$ and $\gamma'$ are combinatorially homotopic, which completes the proof. 
\end{proof}

\begin{rem}
    One can consider the abstract rewriting system defined by $\vec v$ on the vertices of $X$, saying that $x$ can be rewritten into $y$ if and only if there is an oriented path from $x$ to $y$ in $X$. 
    The hypotheses of \cref{p:second-proof} impose that this rewriting system is terminating and confluent \cite[Definition 2.1.3]{baaderTermRewritingAll1998}.
\end{rem}

The class of polyhedral complexes to which \cref{p:second-proof} applies is a strict subclass of simply connected complexes, as the following proposition shows.

\begin{proposition}
    \label{lemma:outgoing-link}
    Let $X$ be a polyhedral complex.
    If there is a generic vector $\vec v \in \R^n$ such that the outgoing link of every vertex is connected, then every path component of $X$ is simply connected.
\end{proposition}

\begin{proof}
    Let $\vec v \in \R^n$ be generic with respect to $X$, and suppose that the outgoing link of every vertex is connected. 
    Since $\vec v$ is generic on edges, it defines a Morse function $\langle \vec v , -\rangle$ on $X$, in the sense of \cite[Definition 2.2]{bestvinaMorseTheoryFiniteness1997}.
    As in classical Morse theory, one can determine the homotopy type of $X$ by considering its successive level sets. 
    For $t \in \R$ denote by $X_t$ the closed subspace of $X$ containing points $x$ such that $\langle x, \vec v \rangle$ is at least $t$.
    Let $x$ be a vertex of $X$ of height $h=\langle x, \vec v \rangle$.
    Observe first that $X_{h+\epsilon}$, for some small $\epsilon>0$, is homotopy equivalent to $X_{h'}$ where $h' > h$ is the next greater height at which there is a vertex.
    That is, the homotopy type of $X$ can only change at vertices  \cite[Lemma 2.3]{bestvinaMorseTheoryFiniteness1997}.
    Then, one proves that $X_h$ is homotopy equivalent to the pushout of $X_{h+\epsilon}$ with the cone over the outgoing link of $x$ along the outgoing link of $x$  \cite[Lemma 2.5]{bestvinaMorseTheoryFiniteness1997}.
    By our assumption, the outgoing link of $x$ is connected, and thus the cone over it is simply connected. 
    Since the pushout of simply connected spaces over a connected space is always simply connected (this is an application of the Seifert--Van Kampen theorem), we obtain by induction that every path component of $X$ is simply connected \cite[Point (3) of Corollary 2.6]{bestvinaMorseTheoryFiniteness1997}.
\end{proof}

The converse of \cref{lemma:outgoing-link} is not true in general: many simply connected polyhedral complexes, as the one represented in \cref{fig:outgoingpoly}, have disconnected outgoing links, for many (sometimes for all) choices of generic orientation vectors. 

\begin{figure}[h!]
\centering
\begin{tikzpicture}
    \node[regular polygon,
    draw,
    regular polygon sides = 8, minimum size = 3cm] (p) at (0,0) {};
    \draw[-] (p.202.5)--(180:4)--(p.157.5);
    \draw[-] (p.157.5)--(135:4)--(p.112.5);
    \draw[-] (p.112.5)--(90:4)--(p.67.5);
    \draw[-] (p.67.5)--(45:4)--(p.22.5);
    \draw[-] (p.22.5)--(0:4)--(p.-22.5);
    \draw[-] (p.-22.5)--(-45:4)--(p.-67.5);
    \draw[-] (p.-67.5)--(-90:4)--(p.-112.5);
    \draw[-] (p.-112.5)--(-135:4)--(p.-157.5);
\end{tikzpicture}
\caption{A simply connected polyhedral complex which admits disconnected outgoing links for every choice of generic vector.}
\label{fig:outgoingpoly}
\end{figure}

This implies that the converse of \cref{p:second-proof} does not hold, and thus that MacLane's original proof is far from reaching the full generality of \cref{thm:top-coherence}.
However, it will be sufficient for our purposes, presented in the next Section, since it applies to any polytope.

\begin{proposition}
\label{prop:polytopes}
    Let $P \subset \R^n$ be a polytope, and let $\vec v \in \R^n$ be generic with respect to $P$. 
    Then, $P$ admits a unique sink $\sk(P)$, and
    the outgoing link of every vertex is connected.
\end{proposition}

\begin{proof}
    The existence and uniqueness of a sink is one of the basic, very useful facts about polytopes, see \cite[Theorem 3.7]{Ziegler95}.
    For the second part, we first observe that the \emph{link} of a vertex $x$ in a polytope, called the \emph{vertex figure} and denoted $P/x$, is itself a polytope of dimension $\dim P -1$, whose $(k-1)$-faces are in bijection with the $k$-faces of $P$ that contain $x$ \cite[Proposition 2.4]{Ziegler95}. 
    Now, the affine hyperplane $H+x$, where $H:=\{y \in \R^n \ | \ \langle \vec v, y \rangle = 0\}$, defines a partition of the vertices of $P/x$ into two connected components: the vertices that correspond to incoming, resp. outgoing, edges of $P$ at $x$.
\end{proof}
% !TEX root = ../Coherence.tex

\section{Rewriting method for coherence} 
\label{s:catoperads}







\subsection{Coherence for categorified operads}

Recall from [REF] the definition of the $\mathbb{N}$-colored operad $\mathcal{O}$ encoding ns operads. Its minimal resolution is given by the cellular chains on the operahedra [DEF]. 


\begin{thm}[Coherence theorem] 
\end{thm}

\begin{proof} TBC
\end{proof}

Restricted to categorified ns operad concentrated in arity 1, i.e. to monoidal categories, we recover MacLane's original coherence tbeorem [REF].

There is an analogous statement for weak Cat-operads \cite[Proposition 14.2]{DP15}. In the same fashion as for \cref{thm:equivalenceDPGLA}, one can prove that the two statements are equivalent. 


\subsection{Weak Cat-operads}

\begin{definition}[Weak Cat-operad {\cite{DP15}}]
\end{definition}

\begin{thm} \label{thm:equivalenceDPGLA}
    The data of a categorified ns operad and the data of a weak Cat-operad are equivalent.  
\end{thm}

\begin{proof}
    TBC
\end{proof}




%% !TEX root = ../Coherence.tex

\section{Perspectives}


\Guillaume{Flow categories?}


\subsection{Rewriting theory and Koszul duality}
Any generic vector $\vec v$ defines a canonical rewriting system on the vertices of a polyhedral complex $X$, which is moreover always terminating and confluent. 
The confluence of the rewriting system described in the second proof of \cref{thm:coherence-operahedra} can be used to give a proof that the colored operad $\mathcal{O}$ \cite[Definition 4.2]{laplante-anfossiDiagonalOperahedra2022a} encoding non-symmetric operads is Koszul (and which avoids resorting to the diamond lemma!): one can proceed with the rewriting method \cite[Section 8.3]{LodayVallette12} adapted to colored operads \cite{KhariKhoro20}, along the lines of the Remark following \cite[Theorem 9.1.5]{LodayVallette12}. 

In the other way around, one can use the fact that $\mathcal{O}$ is Koszul to prove the coherence theorem for categorified non-symmetric operads, using \cite[Theorem 15]{marklCoherenceConstraintsOperads2001} in a suitable generalization of \cite[Example 16]{marklCoherenceConstraintsOperads2001}.
Thus, the categorical coherence theorem for $\mathcal{O}$-algebras and the fact that $\mathcal{O}$ is Koszul are equivalent.
Considering the large scope of the theory developed in \cite{marklCoherenceConstraintsOperads2001}, this could be a general fact. 

\Guillaume{Application to Drinfeld}
\Guillaume{Morse theory and rewriting; polygraphs, higher dimensional rewriting?}

\subsection{Higher categories} 
\label{sec:higher}
\cref{thm:top-coherence} demonstrates that, in the case of monoidal categories, coherence is equivalent to the vanishing of the first homotopy groups of the associahedra. 
Since the associahedra are contractible, and therefore all their homotopy groups vanish, one could hope for a topological proof of higher dimensional coherence theorems.
Seeing a monoidal category as a bicategory with one object, one can ask about a coherence theorem for tricategories with one object. 
A first look at the diagrams in the beginning of \cite[Section 2]{gordonCoherenceTricategories1995} suggests that such a theorem should be at least related to the vanishing of the second homotopy groups of the associahedra.  
To formulate higher dimensional statements, one needs a good structure of pasting scheme on each associahedron, which is the subject of ongoing work \cite{AMMLA}. 

Recent results of S. Barkan provide evidence for these higher dimensional statements, in the context of $\infty$-operads \cite{barkanArityApproximationInfty2022}.
In this vein, it seems likely that the present results could be interpreted as a strict version and a special case of \cite[Theorem B]{barkanArityApproximationInfty2022}. 
It would be interesting to see how the permutoassociahedra arise in the strictification process, and how they are related to operadic partition complexes.  




%% !TEX root = ../Coherence.tex

\section{Koszulity via rewriting} 
\label{s:rewriting}


Let us take a detour by the Koszul duality theory for operads. One of the standard method for proving that an operad is Koszul is the "rewriting method" \cite[Section 8.3]{LodayVallette12}. The generalization to the colored case was done recently by V. Kharitonov and A. Khoroshkin \cite[Theorem 3.12]{KhariKhoro20}.

\begin{thm}[Rewriting method for colored operads {\cite[Theorem 8.3.1]{LodayVallette12}}] \label{thm:rewriting} Let $\mathcal{O}(E,R)$ be a quadratic colored operad. If its generating space $E$ admits a $\mathbb{K}$-linear ordered basis, for which there exists a suitable order on shuffle trees, such that every critical monomial is confluent then the colored operad $\mathcal{O}$ is Koszul. 
\end{thm}

In this case, the operad $\mathcal{O}$ admits an induced shuffle tree basis sharing nice properties, called a PBW basis, see \cite[Section 8.5.3]{LodayVallette12}. Parler Grobner Basis

\begin{thm} \label{thm:Koszulrewriting} The colored operad $\mathcal{O}$ is a Koszul colored operad. 
\end{thm}
\begin{proof} The partial order on complete nested trees defined in X induces, via the proof of X, a suitable order on shuffle trees, see X. The 1-skeleton of operahedra appears as the application of the rewriting rules given by the sequential and parallel axioms. We consider the family of trees $F=\{\tau \in \mathrm{OT}=\ | \ |V(\tau)|=4\}$ and the oriented operahedra $\{(P_\tau,\vec v)\}_{\tau \in F}$ for any vector $\vec v=(v_1,v_2,v_3)$ such that $v_1>v_2>v_3$. Every critical monomial corresponds to the complete nested trees associated to $\{bot(P_\tau,\vec v)\}_{\tau \in F}$, and X implies that every critical monomial is confluent. We conclude by \cref{thm:rewriting}. 
The complete nested trees associated to $\{top(P_\tau, \vec v)\}_{\tau \in F}$ form the PBW basis of $\mathcal{O}_{ns}$.
\end{proof}

The proof of \cref{thm:rewriting} relies on the Diamond Lemma \cite[Theorem 8.5.5]{LodayVallette12}. Instead, we can use the full power of X.

\begin{proof}[Second proof] Every rewriting diagram associated to a monomial is part of the 1-skeleton of some operahedron of dimension $n\geq 0$. By X, this 1-skeleton is oriented and forms the boundary of a topological $n$-ball. Thus, it has a unique maximal element. 
\end{proof}

Restricting to linear trees, we have that the operad $\mathrm{Ass}$ is Koszul. Restricting to the 2-leveled trees as in X, we obtain that the permutad $\mathrm{permAs}^h$ is Koszul, in the sense of M. Markl \cite[Definition 21]{Markl19}.

\medskip

\cref{thm:Koszulrewriting} gives an alternative proof of \cref{thm:coherence}. 

\begin{proof}[Second proof of {\cref{thm:coherence}}] Let $\mathcal{O}$ be a non-unital categorified ns operad. The pentagonal and hexagonal diagrams commute, and they correspond precisely to the 1-skeleton of the 2-dimensional oriented operahedra $\{(P_\tau,\vec v)\}_{\tau \in F}$. Using the Diamond Lemma as in the proof of \cref{thm:rewriting}, we have that every diagram made up of $\theta$ and $\beta$ arrows commute. 
\end{proof}

Here again, resorting to the Diamond Lemma is not necessary.

\begin{proof}[Third proof of {\cref{thm:coherence}}] Any diagram $D$ made up of $\theta$ and $\beta$ arrows lives on the 1-skeleton of an operahedron of some dimension $n\geq 0$. As this operahedron is topologically a $n$-dimensional ball, the diagram $D$ is obtained by gluing together 1-skeletons of 2-dimensional operahedra, which commute by hypothesis.
\end{proof}

As noted in \cite[Remark p.266]{LodayVallette12} for MacLane's coherence theorem, the proofs of the Koszulity of $\mathcal{O}$ and of the coherence theorem are formally the same, and both can be given "instant one-step proofs" via the underlying polytopes. This suggests a common ground for both statements [Maxime Lucas?].
%% !TEX root = ../Coherence.tex

\section{Koszul duality} 
\label{s:koszul}



Another standard method to prove that an operad is Koszul is the "operadic partition poset method" \cite[Section 8.7]{LodayVallette12}. Its extension to the colored setting is straightforward.

\begin{thm}[Koszul-Cohen-Macaulay criterion {\cite[Theorem 8.7.5]{LodayVallette12}}] \label{thm:koszulmacaulay} Let $\mathrm{O}$ be a quadratic basic-set $X$-colored operad generated by a homogeneous $\mathbb{S}$-set concentrated in arity $k$, with $k\geq 2$. Let $\Pi_{\mathrm{O}}(x_1,\ldots,x_n;x_0)$ denote the colored operadic partition posets.
\begin{itemize}
    \item The linear colored operad $\calO=\KK\mathrm{O}$ is a Koszul colored operad if and only if, for every $n\geq 1$ and every $\omega \in \mathrm{Max}(\Pi_{\mathrm{O}}(x_1,\ldots,x_n;x_0))$, the interval $[\hat 0,\omega]$ is Cohen-Macaulay. 
    \item The top homology groups are isomorphic to the Koszul dual cooperad \[H_{\mathrm{top}}(\Pi_{\mathrm{O}}(x_1,\ldots,x_n;x_0))\cong \calO^{\as}(x_1,\ldots,x_n;x_0) \]
\end{itemize}
\end{thm}

%We can apply this theorem to the operad $\calO_{ns}$ to obtain an alternative proof of \cref{thm:Koszulrewriting}.

\begin{proof}[{\cref{thm:Koszulrewriting}}, third proof] The operad $\calO_{ns}$ is a quadratic basic-set $\mathbb{N}$-colored operad generated by a homogeneous $\mathbb{S}$-set concentrated in arity 2. A colored operadic partition poset $\Pi_{\mathrm{OT}}(x_1,\ldots,x_n;x_0)$ is the poset of nestings of all the possible operadic trees obtained by grafting $n$ corollas with $x_1,\ldots,x_n$ leaves and respecting the planar canonical numbering. A maximal element $\omega$ in $\Pi_{\mathrm{OT}}(x_1,\ldots,x_n;x_0)$ is a particular operadic tree $\tau \in \mathrm{OT}$ with $n$ vertices, and an interval $[\hat 0,\omega]$ is the poset of nestings $\mathrm{N}_\tau$ of $\tau$. This poset is a lattice, thus it is also Cohen-Macaulay. 
\end{proof}

So we see that the posets $[\hat 0,\omega]$ have more structure than the Cohen-Macaulay property, which is a condition equivalent to Koszulity. They are lattices, moreover isomorphic to face lattices of polytopes. This suggests that the operad $\calO_{ns}$ has further algebraic properties. 



\begin{thm} \label{prop:koszulpoincare} Let $\mathrm{O}$ be a quadratic set $X$-colored operad which is also finite in each arity, and let $\calO=\KK \mathrm{O}$ be its algebraic avatar. We suppose that there exists a family $P=\{P_\tau\}_{\tau \in \mathrm{O}}$ of polytopes inedexed by the elements of $\mathrm{O}$, which is endowed with a topological colored operad structure and such that we have an isomorphism of de colored cooperads (resp. operads) \[C^{-\bullet}_{\mathrm{cell}}(P)\cong B\calO \quad (\text{resp.} \ C_{\bullet}^{\mathrm{cell}}(P)\cong \Omega\calO^{\as})\ . \] Then, we have that
    \begin{enumerate}
        \item the colored operad $\calO$ is Koszul, 
        \item the colored operad $\calO$ is self-dual for Koszul duality, 
        \item there is an isomorphism of colored operads (resp. cooperads)  $C_{\bullet}^{\mathrm{cell}}(P)\cong \Omega\calO^{\as} \ (\text{resp.} C^{-\bullet}_{\mathrm{cell}}(P)\cong B\calO)$, and
        \item there is an isomorphism of colored operads $\Omega B \calO \cong C_{\bullet}^{\mathrm{cell}}(P_\mathrm{sub})$\ .
    \end{enumerate} 
\end{thm}
Here, $P_\mathrm{sub}$ stands for the family of polytopes $\{P_\tau^{\mathrm{sub}}\}_{\tau \in \mathrm{O}}$ where each polytope $P_\tau$ is endowed with its dual subdivision.
\begin{proof} 

\leavevmode
    
\begin{enumerate}
    \item The fact that the polytopes are contractible imply that the cohomology (resp. homology) of the bar (resp. cobar) construction is concentrated in syzygy degree zero.
    \item As $\calO$ comes from a set-theoretic colored operad, we have the following isomorphisms of dg colored cooperad 
    \begin{eqnarray*} 
        \calO^{\as}\cong H^{[\bullet]}(B\calO)
        = \bigoplus_{\tau \in \mathrm{O}} H^{[\dim P_\tau-\bullet]}\left(C^{\dim P_\tau -\bullet}_{\mathrm{cell}}(P_\tau)\right)\cong \bigoplus_{\tau \in \mathrm{O}} \KK \cong \calO^{*} \ . 
    \end{eqnarray*} 
    The symmetric statement is obtained by dualization.
    \item The preceding two points now imply the following chain of isomorphims of dg colored operads 
    \[\Omega \mathcal{O}^{\as} \cong (B(\mathcal{O}^{\as})^{*})^{*} \cong (B\mathcal{O}^{!})^{*}\cong (B\mathcal{O})^{*}\cong C_{\mathrm{cell}}^{-\bullet}(P)^{*} \cong C_\bullet^{\mathrm{cell}}(P)  \ . \] 
    \item \Guillaume{TBC}
\end{enumerate}
\end{proof}

\Guillaume{
\begin{itemize}
    \item Remonter au poset des compositions? (--> Giraudo)
    \item Rel\^acher les hypoth\`eses permet-t-il d'appliquer le r\'esultat de Jovana?
    \item Qu'arrive-t-il si on ne demande pas des polytopes mais seulement des CWCX?
\end{itemize}}




On a aussi gratuitement l'analogue de la proposition 9.3.2 du Loday-Vallette. 

%l'orientation compte
%structure d'opérade sur les polytopes abstraits


\begin{thm} \label{thm:OnsKoszul} The colored operad $\calO_{ns}$ is Koszul and self-dual. Moreover, the cellular chains [...].
\end{thm}
\begin{proof} \Guillaume{TBC}
\end{proof}


It seems likely that this theorem can be extended to other known operads governed by a colored operad: Suppose that $\mathcal{P}$ comes from a family of connected simple graphs with substitution. Then, $\mathcal{P}$ is Koszul self-dual and its minimal model is given by the cellular chains on a family of polytopes. 
\begin{proof}[Idea of proof] The bar construction on $\mathcal{P}$ corresponds to the nestings of the underlying graphs. Using the line graphs duality of Proposition \ref{proplinegraph} these can be realized by graph-associahedra. The cellular chains on this family of polytopes then gives us the minimal model $\Omega \mathcal{P}^{\as}$ of $\mathcal{P}$.
\end{proof} 
Some recent work go towards this direction. For instance: 
-constructs in J. Obradovi\'c, P.-L. Curien and J. Ivanovi\'c in \cite{CIO18}.
-minimal model in  J. Obradovi\'c in \cite{Obradovic19} 
-minimal models in  J. Obradovi\'c, M. Markl and M. Batanin in \cite{BMO20}
-and the work of Ward
-V-infinity dioperad
-Pilaud--LA dioperad




%% !TEX root = ../Coherence.tex

\section{Poincar\'e duality} 
\label{s:poincare}


The Poincar\'e isomorphism between chains and cochains is precisely here the isomorphism establishing Koszul self-duality. This is particularly apparent from the last isomorphism $\Omega B \calO \cong C_{\bullet}^{\mathrm{cell}}(P_\mathrm{sub})$.

The intersection pairing gives the equivalence between the resolution and the operad. 

-Link with Salvatore--Ching for $E_n$


\bigskip

\emph{Acknowledgements.}   
We would like to thank all the participants of the Roberta Seminar held on September 30th, 2020 in Paris for planting the seeds of the present paper.  
We would like to thank Andrea Bianchi for enlightening discussions, and for reminding us of the Seifert--Van Kampen theorem. 
Finally, we are grateful to Zoran Petri{\'c} for precious discussions on weak Cat-operads.

\bibliographystyle{amsalpha}

\bibliography{Coherence}


\end{document}



