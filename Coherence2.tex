% ----------------------------------------------------------------
% AMS-LaTeX Paper ************************************************
% **** -----------------------------------------------------------
\documentclass[11pt]{amsart}
\usepackage{graphicx, mathabx, amssymb,amsfonts,amsmath,amsthm,newlfont}
\usepackage{epsfig,url}
\usepackage{enumerate,enumitem}
\usepackage[colorlinks=true,linkcolor=red,citecolor=blue]{hyperref}
\usepackage[dvipsnames]{xcolor}
\usepackage{color}


%Clever ref
\usepackage[noabbrev,capitalize]{cleveref}

\usepackage[all,2cell]{xy} \UseAllTwocells \SilentMatrices

\usepackage{pstricks,pst-node,pst-tree}


%MARGINS

\setlength{\textwidth}{\paperwidth}
\addtolength{\textwidth}{-2.5in}
\calclayout

% ----------------------------------------------------------------
\vfuzz2pt % Don't report over-full v-boxes if over-edge is small
\hfuzz2pt % Don't report over-full h-boxes if over-edge is small
% THEOREMS -------------------------------------------------------
\newtheorem{thm}{Theorem}[section]
\newtheorem{corollary}[thm]{Corollary}
\newtheorem{lemma}[thm]{Lemma}
\newtheorem{proposition}[thm]{Proposition}
\newtheorem{Questions}[thm]{Questions}
\theoremstyle{definition}
\newtheorem{definition}[thm]{Definition}

\newtheorem{conjecture}{Conjecture} 
\newtheorem{QQ}{Question} 
\newtheorem{prob}{Problem}
\newtheorem{ex}[thm]{Examples}
\newtheorem{example}[thm]{Example}
\newtheorem{policy}{Policy}
\theoremstyle{remark}
\newtheorem{rem}[thm]{Remark}
\newtheorem{caveat}[thm]{Caveat}
\numberwithin{equation}{section}
% MATH -----------------------------------------------------------
\newcommand{\norm}[1]{\left\Vert#1\right\Vert}
\newcommand{\abs}[1]{\left\vert#1\right\vert}
\newcommand{\set}[1]{\left\{#1\right\}}

\newcommand{\To}{\longrightarrow}
\newcommand*{\Longhookrightarrow}{\ensuremath{\lhook\joinrel\relbar\joinrel\rightarrow}}
\newcommand{\Z}{\mathbb Z}
\newcommand{\Q}{\mathbb Q}
\newcommand{\C}{\mathbb C}
\newcommand{\Ok}{\mathcal O}
\newcommand{\ai}{\mathfrak{a}}
\newcommand{\bi}{\mathfrak{b}}
\newcommand{\R}{\mathbb R}
\newcommand{\N}{\mathbb N}
\newcommand{\AM}{A}
\newcommand{\xx}{\mathsf{x}}
\newcommand{\eqv}{\mathrm{ev}}
\font \rus= wncyr10
\newcommand{\sha}{\, \hbox{\rus x} \,}

\newcommand{\ari}{\mathrm{ar}} %arity
\newcommand{\obj}{\mathrm{Ob}} %object

\newcommand{\GC}{\mathcal{GC}}
\newcommand{\q}{/\!/}

\newcommand{\tr}{\mathrm{tr}}
\newcommand{\id}{\mathrm{id}}

\newcommand{\can}{\mathrm{can}}

\newcommand{\mm}{\mathfrak{m}}

\newcommand{\GL}{\mathrm{GL}}
\newcommand{\LP}{L}
\newcommand{\FL}{F\!L}
\newcommand{\mc}{\mu}

\newcommand{\0}{\color{blue}{\mathsf{0}}}

%%%%Macros PL
\newcommand{\Alt}{ \mid\!\!\mid  } 
\newcommand{\inc}{\subseteq}
 \newcommand{\incs}{\subsetneq}
\newcommand{\union}{\cup}
\newcommand{\Union}{\bigcup}	
\newcommand{\comp}{\circ}
\newcommand{\setc}[2]{\set{#1 \mid #2}}

\newcommand \seq[2]{\shortstack{$#1$ \\ \mbox{}\\
                    \mbox{}\hrulefill\mbox{}\\ \mbox{}\\ $#2$}}
\newcommand{\cat}[1]{{\mathbb #1}}
\newcommand{\dl}{[\![} 			
\newcommand{\dr}{]\!]} 
\newcommand{\hyper}[1]{{\mathbb #1}}	
\newcommand{\restrH}[2]{\hyper{#1}\backslash #2}

%Operades

\def\calO{\mathcal{O}}
\newcommand{\KK}{\mathbb{K}}
\newcommand{\opd}[1]{\mathcal{#1}}

%Antischrieck
\newcommand{\as}{{\scriptstyle \text{\rm !`}}}

%Definitions
\definecolor{darkblue}{rgb}{0,0,0.7} % darkblue color
\newcommand{\darkblue}{\color{darkblue}} % darkblue command
\newcommand{\defn}[1]{{\darkblue \emph{#1}}}

%Commentaires 

\newcommand{\Guillaume}[1]{\textcolor{magenta}{\underline{Guillaume}: #1}}
\newcommand{\correction}[1]{\textcolor{red}{#1}}

%Source and sink
\newcommand{\so}{\mathrm{sc}} 
\newcommand{\sk}{\mathrm{sk}} 


\newcommand{\op}{\mathrm{op}}

%PL
\newcommand{\occ}[2]{#1/#2}
\newcommand{\recrestr}[2]{#1_{\cap #2}}
\newcommand{\xyz}[3]{#1\stackrel{#2}{\rightsquigarrow}#3}

\newcommand{\PL}[1]{{\color{red}{#1}}}

\newcommand{\calP}{\mathcal{P}}
\newcommand{\calH}{\mathcal{H}}
\newcommand{\calT}{\mathcal{T}}

%Operators
\DeclareMathOperator{\Sat}{Sat}
\DeclareMathOperator{\supp}{supp}
\DeclareMathOperator{\Ter}{Ter}
\DeclareMathOperator{\outsort}{out}
\DeclareMathOperator{\insort}{in}
\DeclareMathOperator{\var}{var}

%Drapeau européen

\usepackage{graphicx,calc}
\newlength\myheight
\newlength\mydepth
\settototalheight\myheight{Xygp}
\settodepth\mydepth{Xygp}
\setlength\fboxsep{0pt}
\newcommand*\inlinegraphics[1]{%
  \settototalheight\myheight{Xygp}%
  \settodepth\mydepth{Xygp}%
  \raisebox{-\mydepth}{\includegraphics[height=\myheight]{#1}}%
}

%Dessins

\usepackage{tikz}
\usepackage{tikz-cd}
\usepackage{pgfplots}
\usepackage{pgfplotstable}
\tikzset{math3d/.style=
    {x= {(-0.353cm,-0.353cm)}, z={(0cm,1cm)},y={(1cm,0cm)}}}
\tikzset{JLL3d/.style=
    {x= {(0.4cm,-0.2cm)}, z={(0cm,1cm)},y={(-1cm,0cm)}}}
\usetikzlibrary{calc}
\usetikzlibrary{shapes,shapes.geometric,fit,positioning,calc,matrix}
\tikzset{
  optree/.style={scale=.5,thick,grow'=up,level distance=10mm,inner sep=1pt},
  comp/.style={draw=none,circle,fill,line width=0,inner sep=0pt},
  dot/.style={draw,circle,fill,inner sep=0pt,minimum width=3pt},
  circ/.style={draw,circle,inner sep=1pt,minimum width=4mm},
  emptycirc/.style={draw,circle,inner sep=1pt,minimum width=2mm},
  root/.style={level distance=10mm,inner sep=1pt},
  leaf/.style={draw=none,circle,fill,line width=0,inner sep=0pt},
  nodot/.style={draw,circle,inner sep=1pt},
}

\pgfplotsset{compat=1.12}





% ----------------------------------------------------------------

\def\abovespace{\vspace{12pt}}
\def\belowspace{\vspace{8pt}}



\addtolength{\hoffset}{-0.0in} \addtolength{\textwidth}{0in}
\addtolength{\voffset}{-0.0in} \addtolength{\textheight}{0.0in}


% -----------------------------------------------------------------

\title{Term rewriting on nestohedra}

\author{Pierre-Louis Curien}
\address{IRIF, Universit\'e Paris Diderot and $\pi r^2$ team, Inria, France.}
\email{curien@irif.fr}

\author{Guillaume Laplante-Anfossi}
\address{School of Mathematics and Statistics, University of Melbourne, Victoria, Australia.}
\email{guillaume.laplanteanfossi@unimelb.edu.au}

\date{\today}

\subjclass[2020]{Primary 18N20, Secondary 52B11?} 

\keywords{Term rewriting, nestohedra, hypergraph polytopes, categorified operads, categorical coherence, MacLane coherence theorem.}

\thanks{The second author was supported by the Australian Research Council Future Fellowship FT210100256 and the Andrew Sisson Fund.}


\begin{document}

\begin{abstract}
We define term rewriting systems on the vertices and faces of nestohedra, and show that the former are confluent. 
While the associated poset on vertices generalizes Barnard--McConville's flip order for graph-associahedra, the preorder on faces likely generalizes the facial weak order for permutahedra. 
Moreover, we define and study contextual families of nestohedra, whose local confluence diagrams satisfy a certain uniformity condition. 
The proof of confluence of their rewriting systems reproduce, via Huet's correspondence, categorical coherence theorems for monoidal categories, categorified permutads and operads.
\end{abstract}

\maketitle

\setcounter{tocdepth}{1}
%\tableofcontents

% !TEX root = ../Coherence2.tex

\section*{Introduction} 
\label{s:introduction}



\subsection*{From rewriting to coherence}

In his seminal notes~\cite{Huet-notes-cat} for a graduate course at Université Paris 7, Gérard Huet explained Mac Lane's proof of the coherence theorem for monoidal categories through the lenses of equational reasoning and  term rewriting theory. 
Huet remarked that Mac Lane's pentagons can be read as local confluence diagrams, and that
\begin{enumerate}
\item proving  the coherence statement in the case of canonical natural transformations $\lambda:F\rightarrow G$, where $\lambda$ is defined using the associator only and where $G$ is a normal form, amounts to annotating the proof of Newman's lemma with  explicit names for the rewriting steps, 
\item the proof of the general case of the coherence theorem mimicks the proof of the Church-Rosser property, which states that if two terms $P,Q$ can be proved equal in the equational theory obtained by forgetting the orientation of the rewriting rules, then there is some $N$ such that $P\rightarrow\ldots\rightarrow N$ and
$Q\rightarrow\ldots\rightarrow N$.
\end{enumerate}
Moreover, in order to check local confluence, it is enough to check local confluence of {\em critical pairs}, which are minimal situations in which $M\rightarrow P$ and $M\rightarrow Q$ and the respective subterms of $M$ to which the two reductions are applied overlap. Huet observed that Mac Lane's pentagon expresses the unique critical pair of  the rewriting system given by the associator.

\subsection*{Coherence and polytopes}

In a previous paper~\cite{CLA1}, we discussed combinatorial topological proofs of coherence theorems.
In particular, we gave a topological proof of Mac Lane's coherence theorem by using the fact that all diagrams involved live on the $2$-skeleton of a family of polytopes, the associahedra. 
Here, 
\begin{itemize}
\item[(0)] 0-cells correspond to functors, 
\item[(1)] paths in the 1-skeleton correspond to natural transformations,  
\item[(2)] coherence conditions in situation correspond to 2-faces,  
\end{itemize}
and the coherence statement amounts to asking whether any two parallel cellular can be related by repeatedly replacing a portion of a path fitting on the boundary of a $2$-cell by the complementary path on that same boundary.
In fact, our topological/combinatorial results can be applied to give ``one step proofs'' (quoting Kapranov~\cite{kapranov1993}) of a number of other categorical coherence theorems. 

\subsection*{Rewriting on nestohedra}

It is therefore natural to ask if there is a general way to associate a term rewriting system to a polytope, yielding the above coherence results via Huet's correspondence for different families of interest. 
In this paper, we give a positive answer to this question for the family of hypergraph polytopes, a.k.a nestohedra. 
We construct term rewriting systems on the vertices and faces of hypergraph polytopes (\cref{ss:rewriting-constructs,ss:rewriting-constructions}), and show that the former are confluent (\cref{thm:confluent}). 
We characterize their critical pairs as certain types of $2$-faces (\cref{thm:critical-pairs}).
The rewrite rules on the vertices generalize Barnard--McConville's \emph{flip order} on the vertices of graph-associahedra \cite{Barnard-McConville}, and are induced by an orientation vector (\cref{Tamari-orientation-vector}).
Meanwhile, the rewrite rules on the faces seem to generalize the \emph{facial weak order} on the faces of permutahedra \cite{KrobLatapyNovelliPhanSchwer,PalaciosRonco,DermenjianHohlwegPilaud}.

To further study coherence, we restrict our attention to subfamilies of nestohedra that we call \emph{contextual} (\cref{def:contextual-family}).
These include associahedra, permutahedra and operahedra (\cref{thm:examples}), whose term rewriting systems provide, via Huet's correspondence, coherence theorems for monoidal categories, categorified permutads and operads, respectively. 
The idea behind the notion of contextual nestohedra is to require local confluence diagrams, which correspond to $2$-faces of the polytopes, to satisfy a certain uniformity condition, allowing the distinction between ``coherence condition'' and ``coherence condition in situation'' to be preserved.
The observation that this distinction was lost when using topological methods \cite{CLA1} is what initially prompted the present study.  

%Hypergraph polytopes are \emph{generalized permutahedra}~\cite{P09}. 

\subsection*{Notations}

We use $\prec$ to denote cover relations in a poset, and $|-|$ to denote the cardinality of a set.


% !TEX root = ../Coherence2.tex

\section{Hypergraph polytopes} 
\label{s:hypergraph}

In this section, we recall the definition of hypergraph polytopes. 
We refer to \cite{DP-HP,COI} for more details. 

%%%%%%%%%%%%%%%%%%%%%%%%%%%%%%%%%%%%%%%%%%%

\subsection{Hypergraphs}
A \defn{hypergraph} is given by a finite set $H$ of \defn{vertices} and a subset of \defn{hyperedges} $\hyper{H}\inc {\cal P}(H)\setminus\emptyset$ such that $\Union \hyper{H}=H$. 
We always assume that $\hyper{H}$ is \defn{atomic}, that is $\set{x}\in \hyper{H}$, for all $x\in H$. 
A hyperedge of cardinality 2 is called an \defn{edge}.  
For $X\inc H$, the \defn{plain restriction} of $\hyper{H}$ to $X$ is the set 
$\hyper{H}_X := \setc{Z\in \hyper{H}}{\; Z\inc X}$.

We say that $\hyper{H}$ is \defn{connected} if there is no non-trivial partition $H=X_1\union X_2$ such that $\hyper{H}=\hyper{H}_{X_1}\union \hyper{H}_{X_2}$. 
For each hypergraph, there exists a partition $H=X_1\union\ldots\union X_m$ such that each $\hyper{H}_{X_i}$ is connected and $\hyper{H}=\Union(\hyper{H}_{X_i})$.  
The $\hyper{H}_{X_i}$'s are called the \defn{connected components} of $\hyper{H}$.
For $X\inc H$, we say that a non-empty subset $X$ of vertices is \defn{connected} (resp. a \defn{connected component}) whenever $\hyper{H}_X$ is connected (resp. a connected component of $\hyper{H}$).  
We denote by $\restrH{H}{X}:=\hyper{H}_{H\setminus X}$ the plain restriction of $\hyper{H}$ to $H \setminus X$.
The \defn{saturation} of $\hyper{H}$ is the hypergraph
$\Sat(\hyper{H})=\setc{X}{\emptyset\incs X\inc H\;\mbox{and}\;\hyper{H}_X\;\mbox{is connected}}$.
A hypergraph is called \defn{saturated} when $\hyper{H}=\Sat(\hyper{H})$.  
The \defn{reconnected restriction} of $\hyper{H}$ to $X$ is the set $$\recrestr{\hyper{H}}{X}:=\setc{Z\cap X}{Z\in \Sat(\hyper{H}), Z\cap X\neq\emptyset}.$$

\begin{rem}
    Atomic and saturated hypergraphs are called \defn{building sets} in the nestohedra literature, see for example \cite{P09,FS05}.
\end{rem}

For $X\inc H$, we will express the fact that $\setc{H_i}{i\in I}$ is the set of connected components of $\restrH{H}{X}$ by the notation $\hyper{H},X  \leadsto  \setc{H_i}{i\in I}.$
If $I=\set{1,\ldots,n}$, we write simply $\hyper{H},X  \leadsto H_1,\ldots,H_n$, with no order intended.
We will write $\hyper{H}_i$ for $\hyper{H}_{H_i}$.
In the specific situation where $x,y,z\in H$ and $\hyper{H},\set{x}\leadsto \setc{H_i}{i\in I}$, we shall write
$$\begin{array}{ll}
\xyz{x}{\hyper{H}}{\set{y,z}} & \mathrm{if}\; y,z\in H_i\; \mbox{for some}\; i \in I\\
\xyz{x}{\hyper{H}}{\set{y},\set{z}} & \mbox{otherwise}.
\end{array}$$
In the second case, we will say that $x$ \defn{disconnects} $y$ and $z$ in $\hyper{H}$. 
The reconnected restriction allows one to charaterize the preceding two situations as follows.

\begin{lemma} 
\label{xyz-reconnected} 
We have
$$\begin{array}{lll}
\xyz{x}{\hyper{H}}{\set{y,z}} & \mathrm{iff} & \recrestr{\hyper{H}}{\set{x,y,z}},\set{x}\leadsto\set{y,z} \\
\xyz{x}{\hyper{H}}{\set{y},\set{z}} & \mathrm{iff} & \recrestr{\hyper{H}}{\set{x,y,z}},\set{x}\leadsto \set{y},\set{z}.
\end{array}$$
\end{lemma}

\begin{proof} 
    Let $\hyper{H},x\leadsto H_1,\ldots,H_n$. 
    Suppose $\xyz{x}{\hyper{H}}{\set{y,z}}$. 
    Then there exists $i$ such that $\set{y,z}\inc H_i$, and hence $\set{y,z}\inc H_i\cap\set{x,y,z}$, and in fact $\set{y,z} = H_i\cap\set{x,y,z}$ since $x\not\in H_i$. 
    Thus $\recrestr{\hyper{H}}{\set{x,y,z}},\set{x}\leadsto\set{y,z}$ holds by definition of reconnected restriction.
    If $\xyz{x}{\hyper{H}}{\set{y},\set{z}}$, then there exist $i\neq j$ such that $y\in H_i$ and $z\in H_j$. 
    We then derive likewise $H_i\cap\set{x,y,z}=\set{y}$ and $H_j\cap\set{x,y,z}=\set{z}$ from which $ \recrestr{\hyper{H}}{\set{x,y,z}},\set{x}\leadsto \set{y},\set{z}$ follows.
\end{proof}

%%%%%%%%%%%%%%%%%%%%%%%%%%%%%%%%%%%%%%%%%%%%%%%%%%%%%%%

\subsection{Constructs}
A {connected} hypergraph $\hyper{H}$ gives rise to a set of \defn{constructs}, which are defined inductively as follows.
 
\begin{definition} 
\label{inductive-construct}
%as follows.
Let $\hyper{H}$ be a connected hypergraph and $Y$ be a non-empty subset of $H$.
\begin{enumerate}
\item  If $Y = H$, then the one-node tree $H$ decorated with $H$ is a construct of $\hyper{H}$.
\item If $\hyper{H},Y  \leadsto H_1,\ldots,H_n$, and if $T_1,\ldots,T_n$ are constructs of $\hyper{H}_1,\ldots,\hyper{H}_n$, respectively, then the
tree $Y(T_1,\ldots,T_n)$ whose root is decorated by $Y$, with $n$ outgoing edges on which the respective $T_i\,$'s are grafted is a construct.  
\end{enumerate}
When $Y={z}$ is a singleton, we freely write $z$ in place of $\set{z}$.
A \defn{construction} is a construct all of whose nodes are  decorated with singletons. 
\end{definition}

Since all decorations in a construct are disjoint, we freely identify nodes with subsets of $H$. 
We use the notation $\occ{T}{X}$ to denote the full subtree of $T$ rooted at $X$, defined only if $X$ is indeed a decoration of a node of $T$. 
If $U$ is a (not necessarily full) subtree of $T$, we denote by $\supp(U)$ the union of the decorations of the nodes of $U$.

\begin{rem} \label{subconstruct-restriction}
The intention behind this presentation is algorithmic: a construct is built by picking and removing a non-empty subset $Y$ of $H$, then branching to the connected components of $\restrH{H}{Y}$ and continuing inductively in all the branches.
It follows readily from the definition that $\occ{T}{X}$ is a construct of $\hyper{H}_{\supp(\occ{T}{X})}$.
\end{rem}

\begin{rem}
    The notion of construct is equivalent to the notion of nested set \cite{P09}, and to the notion of tubing in the case where $\hyper{H}$~\cite{CD-CCGA} is a graph.  
    We refer to \cite[Sec.~3.1]{COI} for details.
\end{rem}

%Here we content ourselves with a sketchy description of the
%dictionary (for the readers familiar with nested sets).
%Given a construct $T$, take the set $\setc{\supp(\occ{T}{X})}{X\:\textrm{is a node of}\: T}$. Conversely, read a construct from a nested set $\mathbb{T}$ as follows: for each  $Z\in\mathbb{T}$, consider the elements $Z_1,\ldots, Z_n\in\mathbb{T}$ that are maximal among all elements of $\mathbb{T}$ that are strictly included in $Z$, then $Z\setminus\bigcup(Z_1\cup \ldots \cup Z_n)$ will decorate a node of the corresponding construct. 
%The subface relation formulated in terms of nested sets is just the inclusion of nested sets.

If $X,Y$ are two nodes of a construct $S$ of $\hyper{H}$, $X$ being the father of $Y$, we can define a new construct $T$ by contracting the edge between $X$ and $Y$, and labeling the resulting vertex of~$T$ by the union of the labels of $X$ and $Y$. 

\begin{definition}
    We denote $({\cal A}(\hyper{H}),\preceq)$ the poset of constructs of a connected hypergraph $\hyper{H}$ obtained as the reflexive and transitive closure of the following covering relations: a construct $S$ covers a construct $T$ if $T$ can be obtained from $S$ by contracting an edge.
\end{definition}

%%%%%%%%%%%%%%%%%%%%%%%%%%%%%%%%%%%%%%%%%%%%%%

\subsection{Hypergraph polytopes}

We are now ready to define hypergraph polytopes, a.k.a nestohedra.

\begin{definition}
    A \defn{hypergraph polytope} is a polytope whose face lattice is isomorphic to the poset of constructs of some connected hypergraph $\hyper{H}$.
\end{definition}

Do\v sen and Petri\'c gave polytopal realisations of hypergraph polytopes in ~\cite{DP-HP}.
The idea is that the connected subsets of $\hyper{H}$ specify the faces of a fixed $(|H|-1)$-dimensional simplex, that are to be truncated according to the constructs of $\hyper{H}$.

\begin{rem}
    Quite different geometric realisations of nestohedra were given in \cite{P09}. 
\end{rem}

\begin{example}
    Our key basic examples of hypergraphs form the following ``quatuor'':
$$\begin{array}{lll}
\hyper{S}^n=\set{\set{1},\ldots,\set{n},\set{1,\ldots,n}} \\
\hyper{C}^n= \set{\set{1},\ldots,\set{n},\set{1,2},\ldots,\set{1,2,\ldots,i},\ldots,\set{1,\ldots,n}}\\
\hyper{K}^n= \set{\set{1},\ldots,\set{n},\set{1,2},\ldots,\set{i-1,i},\ldots,\set{n-1,n}}\\
\hyper{P}^n=\setc{X\inc\set{1,\ldots,n}}{1\leq |X|\leq 2}.
\end{array}$$
Their geometric realisations are the $(n-1)$-dimensional simplex,  hypercube,  associahedron, and permutohedron, respectively. 
Note that $\hyper{K}_n$ and $\hyper{P}_n$ are graphs while $\hyper{S}_n$ and $\hyper{C}_n$ are genuine hypergraphs. 
Note also that the definition of $\hyper{S}_n$ and $\hyper{P}_n$ does not depend on any order on the vertices, while the definition of $\hyper{C}_n$ and $\hyper{K}_n$ involves the total (or linear) order on $\set{1\ldots,n}$. 
Another way to say this is that we can replace $\set{1,\ldots,n}$ by any finite set (resp. finite linearly ordered set) and define $\hyper{S}^X$ and $\hyper{P}^X$ (resp. $\hyper{C}^X$, $\hyper{K}^X$).
\end{example}

Many more examples of  hypergraph polytopes are to be found in~\cite{DP-HP,COI,CDOO}, as well as in the abundant literature on nestohedra.



% !TEX root = ../Coherence2.tex

\section{Dimensional faces} 
\label{s:2faces}

In this section, we describe the $2$-dimensional faces of hypergraph polytopes.

%%%%%%%%%%%%%%%%%%%%%%%%%%%%%%%%%%%%%%%%%%%

\subsection{Two types of $2$-faces}
woijew






% !TEX root = ../Coherence2.tex

\section{Hypergraphic rewrite systems} 
\label{s:rewriting}

We associate to each hypergraph a term rewrite system given by its constructs.

%%%%%%%%%%%%%%%%%%%%%%%%%%%%%%%%%%%%%%%

\subsection{Definition}

A \defn{signature} $\Sigma$ is a tuple $(V,F,S,\ari,\sort,\arisort)$ made of 
\begin{itemize}
  \item a set $V$ of \defn{variables},
  \item a non-empty set $F$ of \defn{function symbols}, and
  \item a set $S$ of \defn{sorts},
\end{itemize}
together with an \defn{arity}, \defn{sort} and \defn{arity-sort} functions
\begin{itemize}
  \item $\ari : F \to \mathbb{N}$,
  \item $\sort : F \cup V \to S$,
  \item $\arisort : F \to \prod_{n \geq 0} S^{n}$,
\end{itemize}
such that for $f \in F$, we have $\arisort(f) \in S^{\ari(f)}$. 
The $i$th component of $\arisort(f)$ is denoted $\arisort(f,i)$.
The set $\Ter(\Sigma)$ of \defn{terms} over a signature $\Sigma$ is defined inductively as follows. 
\begin{enumerate}
  \item If $t \in V$ is a variable, then $t$ is a term.
  \item If $f \in F$ is an arity $n$ function symbol, and $t_1,\ldots,t_n$ are terms such that $\sort(t_i)=\arisort(f,i)$, then $f(t_1,\ldots,t_n)$ is a term, and $\sort(f(t_1,\ldots,t_n)):=\sort(f)$.
\end{enumerate}
For a term $t \in \Ter(\Sigma)$, its set of \defn{variables} is defined as 
\begin{equation*}
  \var(t) := 
  \begin{cases}
    \{t\} & \text{ if } t \in V, \\
    \bigcup_{1 \leq i \leq n}\var(t_i) & \text{ if } t=f(t_1,\ldots,t_n).
  \end{cases}
\end{equation*}
A \defn{rewrite rule} over $\Sigma$ is an ordered pair $(l,r)$ of terms in $\Ter(\Sigma)$, denoted $l \to r$, such that
\begin{enumerate}
  \item the first term $l$ is not a variable, that is $l \notin V$.
  \item the variables of the second term are already in the first term, that is $\var(r) \subseteq \var(l)$.
\end{enumerate}

\begin{definition}
  A (many-sorted) \defn{term rewrite system} is a pair $(\Sigma,R)$ made of a signature and a set of rewrite rules $R$ over $\Sigma$.
\end{definition}

Let $\hyper{H}$ be a connected hypergraph. 
Consider the following \defn{signature} $\Sigma_\hyper{H}$ of $\hyper{H}$,  
\begin{itemize}
  \item variables are pairs $(X,Y)$ of non-empty subsets $X \subseteq Y \subseteq H$,
  \item function symbols are non-empty subsets $Y \subseteq H$,
  \item sorts are non-empty subsets $Y \subseteq H$,
  \item $\ari(Y)$ is the number of connected components of $\hyper{H}\setminus Y$,
  \item $\sort(X,Y):=Y$, $\sort(Y):=Y$.
  \item $\arisort(Y,i):=H_i$
\end{itemize}

\begin{lemma}
  There is a bijection between the set of terms of sort $H$ over $\Sigma_\hyper{H}$ and the set of constructs of $\hyper{H}$.
\end{lemma}

\begin{proof}
  
\end{proof}

contextuel c'est $\hyper{H}_{\cap X}=(\hyper{H}_E)_{\cap X}$ pour tout $X$ de taille $3$ et tout sous-ensemble connexe $E$ de cardinalite plus grande ou egale a $3$.
%contextuel c'est par rapport à la cohérence! Pas au système de réécriture

%je veux que les termes soient les constructs

%instantiation 
%in context 
%critical pair

%Est-ce qu'on n'est pas en train de montrer que c'est une opérade colorée...


%% !TEX root = ../Coherence2.tex

\section{The hypergraph operad} 
\label{s:hyperoperad}

We define an operad structure on the sets of faces of hypergraph polytopes.

%%%%%%%%%%%%%%%%%%%%%%%%%%%%%%%%%%%%%%%

\subsection{Definition}

We define a $S$-colored operad, where $S$ is the set of all connected hypergraphs.

\begin{definition}
  The \defn{homotopy hypergraph operad} $\calH_\infty$ is the free colored operad on generators 
  $$\{ X \ | \ \emptyset \neq X \subseteq H , \ \hyper{H} \text{ is a connected hypergraph}\},$$
where the input colors of $X$ are the connected components of $\hyper{H}\setminus X$ and the ouput color of $X$ is $H$.
It's differential is given by the boundary map of hypergraph polytopes, where we consider an operation as part of the hypergraph made of the reconnected complement of its set of vertices in the output color of its root. 
\end{definition}


Note that restricting the definitio above to subsets $X$ of cardinal $1$, we obtain a suboperad $\calH_\infty^1 \subset \calH_\infty$. 

\begin{definition}
  The \defn{hypergraph operad} $\calH$ is the quotient of the free colored operad $\calH_\infty^1$ by the operadic ideal generated by the relations $$x(y)=y(x).$$
\end{definition}

\begin{thm}
  The operad $\calH$ is Koszul.
\end{thm}

\begin{proof}
  By \cref{thm:confluent}.
\end{proof}

\begin{thm}
  The homotopy hypergraph operad $\calH_\infty$ is the minimal model of the hypergraph operad $\calH$.
\end{thm}

\begin{proof}
  Hypergraph polytopes are contractible. 
\end{proof}

\begin{thm}
  Categorified $\calH$-algebras are coherent.
\end{thm}

\begin{proof}
  Via Huet's correspondence from the rewriting system, or two different geometric proofs from \cite{CLA1}.
\end{proof}

\begin{example}
  Restricting to appropriate families of hypergraphs, we recover results for reconnectads, Batanin--Markl--Obradovic, modular operahedra, operahedra, associahedra, cubes, simplices... 
\end{example}

Now, we can all specialize these results to contextual nestohedra!

\begin{figure}[h!]
  \begin{center}
  \resizebox{\linewidth}{!}{
   \begin{tikzpicture}[scale=6.5]
      \node (P1) at (0,1) {$(a \otimes b) \otimes c$};
      \node (P2) at (-0.5,0.866) {$a \otimes (b \otimes c)$};
      \node (P3) at (-0.866,0.5) {$a \otimes (c \otimes b)$};
      \node (P4) at (-1,0) {$(a \otimes c) \otimes b$};
      \node (P5) at (-0.866,-0.5) {$(c \otimes a) \otimes b$} ;
      \node (P6) at (-0.5,-0.866) {$c \otimes (a \otimes b)$};
      \node (P7) at (0,-1) {$c \otimes (b \otimes a)$};
      \node (P8) at (0.5,-0.866) {$(c \otimes b) \otimes a$};
      \node (P9) at (0.866,-0.5) {$(b \otimes c) \otimes a$};
      \node (P10) at (1,0) {$b \otimes (c \otimes a)$};
      \node (P11) at (0.866,0.5) {$b \otimes (a \otimes c)$} ;
      \node (P12) at (0.5,0.866) {$(b \otimes a) \otimes c$};
      \draw[->] (P1)--(P2) node[midway,above left] {$\beta$};
      \draw[->] (P2)--(P3) node[midway,above left] {$1\otimes\tau$};
      \draw[->] (P3)--(P4) node[midway,above left] {$\beta^{-1}$};
      \draw[->] (P4)--(P5) node[midway,below left] {$\tau\otimes 1$};
      \draw[->] (P5)--(P6) node[midway,below left] {$\beta$};
      \draw[->] (P6)--(P7) node[midway,below left] {$1\otimes\tau$};
      \draw[->] (P1)--(P12) node[midway,above right] {$\tau\otimes 1$};
      \draw[->] (P12)--(P11) node[midway,above right] {$\beta$};
      \draw[->] (P11)--(P10) node[midway,above right] {$1\otimes\tau$};
      \draw[->] (P10)--(P9) node[midway,below right] {$\beta^{-1}$};
      \draw[->] (P9)--(P8) node[midway,below right] {$\tau\otimes 1$};
      \draw[->] (P8)--(P7) node[midway,below right] {$\beta$};
      \draw[->,dashed] (P2)--(P9) node[midway,above right] {$\tau$};
      \draw[->,dashed] (P3)--(P8) node[midway,below left] {$\tau$};
  \end{tikzpicture}
  \quad\quad
   \begin{tikzpicture}[scale=6.5]
      \node (P1) at (0,1) {$(\set{a}  \set{b})  \set{c}$};
      \node (P2) at (-0.5,0.866) {$\set{a}  (\set{b}  \set{c})$};
      \node (P3) at (-0.866,0.5) {$\set{a}  (\set{c}  \set{b})$};
      \node (P4) at (-1,0) {$(\set{a}  \set{c})  \set{b}$};
      \node (P5) at (-0.866,-0.5) {$(\set{c}  \set{a})  \set{b}$} ;
      \node (P6) at (-0.5,-0.866) {$\set{c}  (\set{a}  \set{b})$};
      \node (P7) at (0,-1) {$\set{c}  (\set{b}  \set{a})$};
      \node (P8) at (0.5,-0.866) {$(\set{c}  \set{b})  \set{a}$};
      \node (P9) at (0.866,-0.5) {$(\set{b}  \set{c})  \set{a}$};
      \node (P10) at (1,0) {$\set{b}  (\set{c}  \set{a})$};
      \node (P11) at (0.866,0.5) {$\set{b}  (\set{a}  \set{c})$} ;
      \node (P12) at (0.5,0.866) {$(\set{b}  \set{a})  \set{c}$};
      \node (P13) at (0,0) {$\set{a,b,c}$} ;
      \draw[-] (P1)--(P2) node[midway,above left] {$\set{a}\set{b}\set{c}$};
      \draw[-] (P2)--(P3) node[midway,above left] {$\set{a}\set{b,c}$};
      \draw[-] (P3)--(P4) node[midway,above left] {$\set{a}\set{c}\set{b}$};
      \draw[-] (P4)--(P5) node[midway,below left] {$\set{a,c}\set{b}$};
      \draw[-] (P5)--(P6) node[midway,below left] {$\set{c}\set{a}\set{b}$};
      \draw[-] (P6)--(P7) node[midway,below left] {$\set{c}\set{a,b}$};
      \draw[-] (P1)--(P12) node[midway,above right] {$\set{a,b}\set{c}$};
      \draw[-] (P12)--(P11) node[midway,above right] {$\set{b}\set{a}\set{c}$};
      \draw[-] (P11)--(P10) node[midway,above right] {$\set{b}\set{a,c}$};
      \draw[-] (P10)--(P9) node[midway,below right] {$\set{b}\set{c}\set{a}$};
      \draw[-] (P9)--(P8) node[midway,below right] {$\set{b,c}\set{a}$};
      \draw[-] (P8)--(P7) node[midway,below right] {$\set{c}\set{b}\set{a}$};
  \end{tikzpicture}} 
  \end{center}
  \caption{Kapranov dodecagons.}
  \label{fig:dodecagon}
  \end{figure}
% !TEX root = ../Coherence2.tex

\section{Contextual nestohedra} 
\label{s:contextual}

We define special families of hypergraph polytopes which we call ``contextual'', and exhibit several examples.

%%%%%%%%%%%%%%%%%%%%%%%%%%%%%%%%%%%%%%%

\subsection{Definition}

For a 3-element subset $X=\set{x_1,x_2,x_3}$ of $H$, we say that a 2-face $T$ of $\hyper{H}$ is an \defn{$X$-face} if its unique non-singleton node is decorated by $X$.  
If $X$ is the root of $T$, the following  lemma allows us to see $T=X(\ldots)$ as an ``instantiation'' of $X$, viewed as the maximum face of $\recrestr{\hyper{H}}{X}$.

\begin{lemma} 
  \label{instance-construct} 
  If $\hyper{H}$ is a connected hypergraph, if $X$ is a subset of $H$ such that $|X|=3$ and $T$ is a 2-dimensional construct with root $X$, then the poset of faces of $T$ is isomorphic to the poset of faces of $\recrestr{\hyper{H}}{X}$.
\end{lemma}

\begin{proof}
In Section~\ref{ss:type-B}, we have described up to permutation all the possible connected hypergraphs on the set $X$ of vertices and their respective posets of faces. 
We will treat the case where ${\cal A}(\recrestr{\hyper{H}}{X})$ is the poset of (c), the others are similar.
Pick the 0-face~$S:= x_1(x_2,x_3)$. 
We map $S$ to a $0$-face $\phi(S)$ of $T$ as follows. 
Let $\hyper{H},\set{x}\leadsto H_1,\ldots,H_n$. 
By \cref{xyz-reconnected}, we have $\xyz{x_1}{\hyper{H}}{\set{x_2},\set{x_3}}$, hence we have, say $x_2\in H_1$ and $x_3\in H_2$. Then let
  $\hyper{H_1},\set{x_2}\leadsto H_{1,1},\ldots,H_{1,p}$ and $\hyper{H_2},\set{x_3}\leadsto H_{2,1},\ldots,H_{2,q}$.
  Then we have $\hyper{H},X\leadsto H_{1,1},\ldots,H_{1,p},H_{2,1},\ldots,H_{2,q},H_3,\ldots H_n$, so that $T$ writes as
  $$T= X(T_{1,1},\ldots,T_{1,p},T_{2,1},\ldots,T_{2,q},T_3,\ldots T_n).$$ 
All these data determine uniquely a 0-dimensional subface of $T$, namely
  $$\phi(S)=x_1(x_2(T_{1,1},\ldots,T_{1,p}),x_3(T_{2,1},\ldots,T_{2,q}),T_3,\ldots T_n)$$
 and one recovers $S$
  by  pruning in $\phi(S)$ all nodes except those decorated by subsets of $X$.
  The same applies to all other 0-dimensional (resp. 1-dimensional) faces of $\recrestr{\hyper{H}}{X}$, establishing $\phi$ as a bijection, which is also easily seen to be monotonic: for example, we have
 $$\phi(\set{x_1,x_2}(x_3))= \set{x_1,x_2}(x_3(T_{2,1},\ldots,T_{2,q}),T_{1,1},\ldots,T_{1,p},T_3,\ldots T_n),$$
 evidencing $\phi(x_1(x_2,x_3))\preceq \phi(\set{x_1,x_2}(x_3))$. 
 The inverse of $\phi$ is also monotonic, since the above pruning does not affect the place where the contraction occurs -- e.g., the edge of $\phi(x_1(x_2,x_3))$ that is contracted to get $\phi(\set{x_1,x_2}(x_3))$ is the edge between $x_1$ and $x_2$, which can thus be contracted in the preimage $x_1(x_2,x_3)$ to yield the preimage $\set{x_1,x_2}(x_3)$.
 Finally, we set
 $$\phi(\set{x_1,x_2,x_3})=\set{x_1,x_2,x_3}(T_{1,1},\ldots,T_{1,p},T_{2,1},\ldots,T_{2,q},T_3,\ldots T_n)=T.$$
 \end{proof}

Now, if $T'$ is another $X$-face and $\occ{T'}{X}\neq T'$, then we would like to see  $T'$ as $\occ{T'}{X}$ in context, and hence $T'$ as ``$X$ in situation''.  
For this to hold, $T'$ should be of the same form as $T$. 
Setting $Y:=\supp(\occ{T}{X})$, we have that the poset of faces of $T'$ is isomorphic to the poset of faces of $\occ{T'}{X}$, which is a construct of $\hyper{H}_Y$ and has $X$ as root, so is in turn isomorphic to the poset of faces of $\recrestr{\hyper{(\hyper{H}_Y)}}{X}$ (\cref{instance-construct}).
Is it always the case that $\recrestr{(\hyper{H}_Y)}{X}=\recrestr{\hyper{H}}{X}$ for~$Y$ connected in $\hyper{H}$? 
The following examples give a negative answer.

\begin{example} \label{non-contextual-1}
Consider the hypergraph 
\[
  \hyper{H}:= \set{\set{x},\set{y},\set{z},\set{u},\set{x,y,z}, \set{x,u,z}},
  \]
the set $X:=\set{x,y,z}$ and the two $X$-faces $S:=u(X)$ and $T:=X(u)$. 
Then $\occ{S}{X}$ is a construct of $\hyper{K}:=\restrH{H}{\set{u}}$ while $\occ{T}{X}=T$ is a construct of $\hyper{H}$.
But we have $\xyz{y}{\hyper{K}}{\set{x}\!,\!\set{z}}$ while $\xyz{y}{\hyper{H}}{\set{x,z}}$, and $S$ is a triangle while $T$ is a quadrilateral, as
$\recrestr{\hyper{K}}{\set{x,y,z}} = \hyper{K}  =  \set{\set{x},\set{y},\set{z},\set{x,y,z}}$ and $\recrestr{\hyper{H}}{\set{x,y,z}}  =  \set{\set{x},\set{y},\set{z},\set{u},\set{x,z}\set{x,y,z}}$.
\end{example}
  
\begin{example} 
  \label{non-contextual-2}
Consider the graph 
$$\set{\set{x},\set{y},\set{z},\set{u},\set{x,y}, \set{y,z}, \set{x,u}, \set{u,z}}$$
Then exactly the same data as in Example \ref{non-contextual-1} provide evidence that this graph, whose realisation is the three-dimensional cyclohedron, is not contextual. 
\end{example}

This motivates the following definition.

\begin{definition} 
A connected hypergraph $\hyper{H}$ is \defn{contextual} if for all connected subsets $Y\inc H$ of cardinal $|Y|\geq 3$, and for all $3$-elements subsets $X=\{x,y,z\} \subseteq Y$, we have 
$$\begin{array}{lll}
  \xyz{x}{{\hyper{H}_Y}}{\set{y,z}} & \Leftrightarrow & \xyz{x}{\hyper{H}}{\set{y,z}}.
  \end{array}$$
\end{definition}

\begin{lemma} \label{context-lemma}
  A connected hypergraph $\hyper{H}$ is contextual if for all connected subsets $Y \subseteq H$ of cardinal $|Y|\geq 3$, and for all subsets $X \subseteq Y$ of cardinal $|X|=3$, we have
  $$\hyper{H}_{\cap X} = (\hyper{H}_Y)_{\cap X}.$$ 
\end{lemma}

\begin{proof} 
  This is a direct consequence of Lemma~\ref{xyz-reconnected}.
\end{proof}

\begin{proposition} \label{situation-construct}
Let $\hyper{H}$ be a contextual hypergraph.
If $X$ is a subset of $H$ such that $|X|=3$ and $T$ is an $X$-face of $\hyper{H}$, then the poset of faces of $T$ is isomorphic to the poset of faces of~$\recrestr{\hyper{H}}{X}$.
\end{proposition}

\begin{proof} 
  Let $\hyper{K}:=\hyper{H}_{\supp(S)}$ where $S:=\occ{T}{X}$. 
  By \cref{subconstruct-restriction}, $S$ is a constuct of $\hyper{K}$. By definition of the face relation, and since the only non-singleton (and hence ``splittable'') node of $T$ is $X$, we have that the poset of subfaces of $T$ is isomorphic to the poset of subfaces of $S$, which by \Cref{instance-construct} is isomorphic to ${\cal A}(\recrestr{\hyper{K}}{X})$, which is isomorphic to ${\cal A}(\recrestr{\hyper{H}}{X})$ since~$\hyper{H}$ is contextual.
\end{proof}

Proposition~\ref{situation-construct} allows us to see all $X$-faces as ``instantiations in context'' of $\recrestr{\hyper{H}}{X}$, which therefore acts  as a rule or axiom in the terminology of equational theories.


%%%%%%%%%%%%%%%%%%%%%%%%%%%%%%%%%%%%%%%%%%%%%%%%%%%%%%%

\subsection{Contextual families}

Motivated by the examples presented in \cref{ss:hypergraph-polytopes,ss:examples} and their associated categorical coherence theorems listed in \cref{table:contextual-hyper}, we define now the notion of a contextual \emph{family} of nestohedra.

Identifying an hypergraph $\hyper{H}$ with the maximal construct $T$ of $({\cal A}(\hyper{H}),\preceq)$, we say that $\hyper{H}$ has \defn{dimension} $\dim T$.
For a family of hypergraphs $\calH$, we denote by $\calH(n)$ the subset of hypergraphs of dimension $n \geq 0$.

We will consider families of ordered hypergraphs. 
Note that when $\hyper{H}$ is ordered, all the restrictions $\hyper{H}_X$ and reconnected restrictions $\hyper{H}_{\cap X}$ are naturally ordered hypergraphs.

\begin{definition}
  \label{def:contextual-family}
    A family $\calH$ of ordered hypergraphs is \defn{contextual} if 
    \begin{enumerate}
      \item any ordered hypergraph $\hyper{H} \in \calH$ is contextual.
      \item for any $\hyper{H} \in \calH$ an any $X \subseteq H$, all the connected components of $\hyper{H}\setminus X$ are in $\calH$.
      \item we have $\{\hyper{H}_{\cap X} \ | \ X \subset H, |X|=3, \hyper{H} \in \calH \} \subseteq \calH(2)$.
    \end{enumerate}
\end{definition}

The term rewrite systems from \cref{ss:rewriting-constructs,ss:rewriting-constructions} can be adapted to a rewrite system on \emph{all} hypergraphs of $\calH$.
We shall focus on the constructions rewrite system. 

\begin{definition}
  For a contextual family of hypergraphs $\calH$, we consider the \defn{constructions signature}~$\Sigma_\calH^c$ defined by the following data:
  \begin{itemize}
    \item Variables and sorts are elements of $\calH$, 
    \item Function symbols are pairs of an hypergraph $\hyper{H} \in \calH$ and one of its elements 
    $$F:=\{(x,\hyper{H}) \ | \ x \in H, \ \hyper{H} \in \calH \}.$$
    \item For $(x,\hyper{H}) \in F$, we define $\ari(x,H)$ as the number of connected components of~$\hyper{H} \setminus {x}$.
    \item Variables $\hyper{H} \in V$ are their own output sort $\outsort(\hyper{H}):=\hyper{H}$, while function symbols~$(x,\hyper{H}) \in F$ have output sort $\outsort(x,\hyper{H}):=\hyper{H}$.
    \item For function symbols $(x,\hyper{H}) \in F$ such that $\hyper{H},x \leadsto H_1,\ldots,H_n$, and for $1 \leq i \leq n$, we define $\insort((x,\hyper{H}),i):=\hyper{H}_i$.
  \end{itemize}
\end{definition}

It is clear from \cref{def:contextual-family} and the fact that the (reconnected) restriction of a contextual hypergraph is contextual, that this signature is well-defined.
Moreover, it is straightforward to adapt \cref{l:bijection-constructions} and \cref{def:rules-2} to obtain a term rewrite system $(\Sigma_\calH,R_\calH)$ on the constructions of $\calH$. 

From \cref{thm:critical-pairs}, we have that all local confluence diagrams for $(\Sigma_\calH,R_\calH)$ have the form of some $X$-face, for $X \subseteq H$, $|X|=3$ and $\hyper{H} \in \calH$. 
The fact that $\calH$ is contextual imposes an additional uniformity constraint on these diagrams.

\begin{thm} 
Let $\calH$ be a contextual family of ordered hypergraphs.
For any $\hyper{H} \in \calH$ and subset $X \subseteq H$ with $|X|=3$, all the $X$-local confluence diagrams have the same form $\hyper{H}_{\cap X}$. 
\end{thm}

\begin{proof} 
  The proof is an easy consequence of Lemma~\ref{instance-construct} and \cref{situation-construct}. 
\end{proof}

\cref{thm:confluent} implies that $(\Sigma_\calH,R_\calH)$ is confluent.
Moreover, in virtue of Condition (3) in \cref{def:contextual-family} all the possible forms of the local confluence diagrams of $(\Sigma_\calH,R_\calH)$ are in~$\calH(2)$.
We argue that these diagrams should be called \emph{coherence conditions}, in view of the following examples of contextual families and their coherence theorems. 

%%%%%%%%%%%%%%%%%%%%%%%%%%%%%%%%%%%%%%%

\subsection{Examples}
\label{ss:examples}

We call \defn{contextual graph-associahedra} (resp.\ \defn{contextual nestohedra}) the hypergraph polytopes whose underlying hypergraph is a connected (hyper)graph which is moreover contextual.
Here, we include a copy of each (hyper)graph for each possible total order on its vertices.
Recall that simplices, cubes, associahedra, permutahedra and operahedra were introduced in \cref{ss:hypergraph-polytopes}.

\begin{samepage}
  \begin{thm}
    \label{thm:examples}
    The following families of hypergraph polytopes are contextual:
    \begin{enumerate}[label=(\alph*)]
      \item simplices,
      \item cubes,
      \item associahedra,
      \item permutahedra,
      \item operahedra,
      \item contextual graph-associahedra,
      \item contextual nestohedra.
    \end{enumerate}
  \end{thm}
\end{samepage}

\begin{proof}
  Let us proceed one family at a time.
  For each one, we check Conditions (1)-(3) in \cref{def:contextual-family}.
  We consider sets of vertices to be $H=\{1,\ldots,n\}$.
  \begin{enumerate}[label=(\alph*)]
    \item Conditions (1)-(3) follow easily from the fact that hyperedges of simplices are all either singletons or the maximal hyperedge.
    \item We first prove Condition (1). 
    Note $\hyper{C}_n$ is saturated, and that $(\hyper{C}_n)_{\set{1,\ldots,m}}=\hyper{C}_m$ if $m\leq n$.
    So we have to check that for all $m\leq n$ and all $i,j,k\leq m$, we have $\xyz{k}{{\hyper{C}_n}}{\set{i,j}}$ iff
    $\xyz{k}{{\hyper{C}_m}}{\set{i,j}}$, which follows immediately from the observation that for all $p\geq m$ we have
    $\xyz{k}{{\hyper{C}_p}}{\set{i,j}}$ iff $i<k$ and $j<k$.
    For Conditions (2) and (3) it suffices to observe that the connected components of $\hyper{C}_n\setminus X$, for some $X$, are all cubes $\hyper{C}_m$ with $m<n$.
    \item[(c)-(e)] Conditions (1) and (2) follow from the fact that any connected subgraph of a linear (resp. complete, clawfree block) graph is a linear (resp. complete, clawfree block) graph. 
    Condition (3) follows from the fact that any reconnected complement of a subset in a linear (resp. complete, clawfree block) graph is a linear (resp. complete, clawfree block) graph.
    %As to Condition (1), it is proved in~\cite[Lem.~12]{COI} that the connected subsets of $\hyper{L}({\cal T})$ are in bijective correspondence with the subtrees of $\cal T$ having at least two nodes, through a map $E\mapsto  {\cal T}_E$ such that $\hyper{L}({\cal T})_E=\hyper{L}({\cal T_E})$. Suppose, say, that $\set{x,y,z}\inc E$ and $\xyz{x}{\hyper{L}({\cal T})_E}{\set{y},\set{z}}$. Then it means on the tree side that after removing the edge $x$ from ${\cal T}_E$, resulting in two disjoint subtrees ${\cal T}_E^1$ and ${\cal T}_E^2$ of ${\cal T}_E$, we have, say, $y\in{\cal T}_E^1$ and $z\in{\cal T}_E^2$. 
    %On the other hand, removing $x$ from ${\cal T}$ results in subtrees ${\cal T}^1$ and ${\cal T}^2$, containing ${\cal T}_E^1$ and ${\cal T}_E^2$, respectively. 
    %Therefore $\xyz{x}{\hyper{L}({\cal T}))}{\set{y},\set{z}}$. 
    %And vice-versa.
    \item[(f)-(g)] This is immediate from the definitions.
  \end{enumerate}
\end{proof}

%toute restriction d'un hypergraphe contextuel est contextuelle

\begin{rem}
  Note that contextual (hyper)graphs do not contain all graph-associahedra.
  For instance, we have seen in \cref{non-contextual-2} that the cyclohedra are not contextual. 
  It would be interesting to characterize combinatorially contextual (hyper)graphs.
\end{rem}


%%%%%%%%%%%%%%%%%%%%%%%%%%%%%%%%%%%%

\subsection{Categorical coherence}
\label{ss:coherence}

Let us quickly recall MacLane's coherence theorem.
The scene is the data of a category $\mathbf C$, a bifunctor $\otimes:\mathbf{C}^2\rightarrow \mathbf C$ and a natural iso $\alpha$ from the functor
$(X,Y,Z)\mapsto (X\otimes Y)\otimes Z$ to the functor  $(X,Y,Z)\mapsto X \otimes (Y\otimes Z)$. 
The coherence theorem states that for any two functors $F,G$ from $\mathbf{C}^{n+1}$ to $\mathbf{C}$ arising from $n$ iterations of $\otimes$, any two  natural transformations $\lambda_1,\lambda_2$ from $F$ to $G$  ``written using $\alpha$ or its inverse'' are equal, provided the statement holds in the following special case  -- called  {\em coherence condition}: 
\begin{itemize}
\item $F=(X,Y,Z,U)\mapsto ((X\otimes Y)\otimes Z)\otimes U,$ 
\item $G=(X,Y,Z,U)\mapsto X\otimes (Y\otimes (Z\otimes U)),$ 
\item $\lambda_1=(X\otimes\alpha_{Y,Z,U})\circ\alpha_{X,Y\otimes Z,U} \circ (\alpha_{X,Y,Z}\otimes U),$ and
\item $\lambda_2= \alpha_{X,Y,Z\otimes U}\circ \alpha_{X\otimes Y,Z,U},$ 
\end{itemize}
i.e., provided the following diagram (Mac Lane's pentagon) commutes:
\begin{center}
\vspace{-.5cm}
$$
 \xymatrix @-1.65pc {&& ((X\otimes Y)\otimes Z)\otimes U \ar @{->}[ddll]^{\alpha_{X,Y,Z}\otimes U} \ar @{->}[dddrr]^{\alpha_{X\otimes Y,Z,U}}&& \\
 &&&&\\
 (X\otimes (Y\otimes Z))\otimes U  \ar @{->}[dd]^{\alpha_{X,Y\otimes Z,U} }&&   && \\
 &&&&(X\otimes Y)\otimes (Z\otimes U) \ar @{->}[dddll]_{\alpha_{X,Y,Z\otimes U}}\\
 X\otimes ((Y\otimes Z)\otimes U) \ar @{->}[ddrr]_{X\otimes\alpha_{Y,Z,U}} &  &\\
 &&&&\\
 &&X\otimes (Y\otimes (Z\otimes U))&&}
$$
\end{center}

Via Huet's correspondence \cite{Huet-notes-cat}, the annotated proof of confluence of the rewriting system $(\Sigma_\calH, R_\calH)$ associated to the contextual family of associahedra $\calH$ provides a proof of MacLane's coherence theorem, with the pentagon in $\calH(2)$ acting as the coherence condition. 
The following examples explain the translation between the language of hypergraph polytopes and the language of monoidal categories. 

\begin{example}
Consider the linear tree

\vspace{-.7cm}
\begin{center}
$$\xymatrix @-1.65pc {{\cal T} & := &X \ar @{-}[rr]^{1}&& Y \ar @{-}[rr]^{2}&& Z \ar @{-}[rr]^{3}&& U}
  $$
\end{center}
%\vspace{-.2cm}
Then $\hyper{L}({\cal T})$ is the associahedron $\hyper{K}^3$. 
The constructs of ${\cal T}$ decorate a pentagon as follows
%$$
% \xymatrix @-1.65pc {&& 3(2(1)) \ar @{->}[ddll]_{3(\set{1,2})} \ar @{->}[dddrr]^{\set{2,3}(1)}&& \\
% &&&&\\
%3(1(2)) \ar @{->}[dd]^{\set{1,3}(2)} }&&   && \\
% &&&&2(1,3) \ar @{->}[dddll]_{\set{1,2}(3)}\\
%1(3(2)) \ar @{->}[ddrr]^{1(2,3)} &  &\\
% &&&&\\
% &&1(2(3))&&}
%$$
\begin{center}
$$\xymatrix @-1.65pc {&& 3(2(1)) \ar @{->}[ddll]_{3(\set{1,2})} \ar @{->}[dddrr]^{\set{2,3}(1)}&& \\
 &&&&\\
3(1(2))   \ar @{->}[dd]^{\set{1,3}(2)}&&   && \\
 &&&&2(1,3) \ar @{->}[dddll]^{\set{1,2}(3)}\\
 1(3(2)) \ar @{->}[ddrr]^{1(\set{2,3})} &  &\\
 &&&&\\
 &&1(2(3))&&}$$
\end{center}
and are in bijective correspondence with the vertices and edges of Mac Lane's pentagon. 
The encoding is given as follows:
  \begin{itemize}
  \item $(X\otimes_1 Y)\otimes_2 (Z\otimes_3 U)$, where we annotaded the ``compositions'' $\otimes$ with the vertices of $\hyper{K}^3$, can be written $\otimes_2(\otimes_1(X,Y),\otimes_3(Z,U))$ in prefix (or tree) notation. Then we get 
 $2(1,3)$ by removing the leaf nodes of that tree.
 \item $\alpha_{X,Y,Z}\otimes_3 U$ can be interpreted as $(X\otimes_1 Y\otimes_2 Z)\otimes_3 U$ (a non fully parenthesed expression), which likewise 
 translates as $3(\set{1,2})$,where $3(-)$ makes the job of contextualisation.
 \item Likewise, we can move from
$\alpha_{X,Y\otimes_2 Z,U}$ to $X\otimes_1(Y\otimes_2 Z)\otimes_3 U$ to $\set{1,3}(2)$, where $2$ makes the job of instantiation.
\end{itemize}
\end{example}

\begin{example}
Taking the 4-dimensional associahedron $\hyper{K}^5$ (with vertex set $\set{0,1,2,3,4}$), we get the following instance in context of $\hyper{K}^3=\recrestr{\hyper{K}^5}{\set{1,2,3}}$, i.e. of Mac Lane's condition:
\begin{center}
$$\xymatrix @-1.65pc {&& 4(3(2(1(0)))) \ar @{->}[ddll]_{4(3(\set{1,2}(0)))} \ar @{->}[dddrr]^{4(\set{2,3}(1(0)))}&& \\
 &&&&\\
4(3(1(0,2)))  \ar @{->}[dd]^{4(\set{1,3}(0,2))}&&   && \\
 &&&&4(2(1(0),3)) \ar @{->}[dddll]^{4(\set{1,2}(0,3))}\\
 4(1(0,3(2))) \ar @{->}[ddrr]^{4(1(0,\set{2,3}))} &  &\\
 &&&&\\
 &&4(1(0,2(3)))&&}$$
\end{center}
We recover the (encoding of the) edge 
 $$
 \xymatrix @-2pc {&& ((((X_1\otimes_0 X_2)\otimes_1 Y)\otimes_2 Z)\otimes_3 U)\otimes_4 V \ar @{->}[dddddll]_(.6){(\alpha_{(X_1\otimes X_2),Y,Z}\otimes U)\otimes V\quad} && \\
 &&&&\\
 &&&&\\
  &&&&\\
    &&&&\\
( ((X_1\otimes_0 X_2)\otimes_1 (Y\otimes_2 Z))\otimes_3 U)\otimes_4 V  &&   && \\
\ }
$$
as the top left edge above.
\end{example}

Here, the fact that the family of associahedra is contextual implies in particular that the local confluence diagram associated to the expression
$$ (-\otimes_0 - \otimes_1 - \otimes_2 (U \otimes_3 V)),$$
which takes place on the $4$-dimensional associahedron, has the same form as the local confluence diagram associated to the expression
$$ ((X \otimes_0 Y \otimes_1 Z \otimes_2 U) \otimes_3 -),$$
and thus that the former one can be seen as an ``instance'' of MacLane's pentagon, while the latter can be seen as the same pentagon ``in context''.
As we have seen in \cref{non-contextual-2}, this interpretation does not hold anymore if one considers the cycle graph instead of the linear graph (that is, if one was to identify $X$ and $V$ in the expressions above). 

A similar description holds for the permutahedra and the operahedra \cite{CLA1}, giving coherence theorems for categorified permutads and operads, respectively. 
All contextual families of nestohedra and their associated coherence theorems considered so far are summarized in \cref{table:contextual-hyper}.

\begin{table}[h!]
	\begin{center}
	\begin{tabular}{c|c|c}
	Family & Algebraic structure & Coherence theorem \\
	\hline
	Simplices & - & - \\
	Cubes & - & - \\
	Associahedra & Monoidal category & \cite{MacLane63} \\
	Permutahedra & Categorified permutads & \cite{CLA1} \\
	Operahedra & Categorified operads & \cite{DP15,CLA1} \\
	Contextual graph-associahedra & - & - \\
	Contextual nestohedra & - & - 
	\end{tabular}
	\end{center}
  \caption{Families of contextual hypergraphs, the categorical structures that they encode, and their associated coherence theorems.}
  \label{table:contextual-hyper}
\end{table}

\begin{rem}
  It would be interesting to complete \cref{table:contextual-hyper} with appropriate structures and coherence theorems.
  It seems likely that contextual graph-associahedra would be related to a certain type of categorified reconnectads \cite{DotsenkoKeilthyLyskov}.
\end{rem}





%% !TEX root = ../Coherence2.tex

\section{Categorical coherence} 
\label{s:coherence}

%%%%%%%%%%%%%%%%%%%%%%%%%%%%%%%%%%%%%%

\begin{table}[h!]
	\begin{center}
	\begin{tabular}{c|c|c}
	Family & Algebraic structure & Coherence theorem \\
	\hline
	Simplices & - & - \\
	Cubes & - & - \\
	Associahedra & Monoidal category & \cite{MacLane63} \\
	Permutahedra & Categorified permutads & \cite{CLA1} \\
	Operahedra & Categorified operads & \cite{DP15,CLA1} \\
	Contextual graph-associahedra & Categorified reconnectads? & - \\
	Contextual nestohedra & - & - 
	\end{tabular}
	\end{center}
\end{table}

Conjectures: 
\begin{itemize}
  \item contextual graph-associahedra give coherence for (some specific) categorified reconnectads \cite{DotsenkoKeilthyLyskov}
  \item contextual nestohedra give coherence for a hypergraphic generalization
\end{itemize}




\subsection{Interpretation}

Associahedra form a subfamily of operahedra: those obtained from linear trees.  Consider the linear tree

\vspace{-1cm}
\begin{center}
$$\xymatrix @-1.65pc {{\cal L} & = &X \ar @{-}[rr]^{1}&& Y \ar @{-}[rr]^{2}&& Z \ar @{-}[rr]^{3}&& U}
 $$
 \end{center}
 \vspace{-.2cm}
 
 \noindent
 (represented horizontally).  Then $\hyper{G}({\cal L})$ is the associahedron $\hyper{K}^3$. The constructs of
  ${\cal L}$ decorate a pentagon as follows:

%$$
% \xymatrix @-1.65pc {&& 3(2(1)) \ar @{->}[ddll]_{3(\set{1,2})} \ar @{->}[dddrr]^{\set{2,3}(1)}&& \\
% &&&&\\
%3(1(2)) \ar @{->}[dd]^{\set{1,3}(2)} }&&   && \\
% &&&&2(1,3) \ar @{->}[dddll]_{\set{1,2}(3)}\\
%1(3(2)) \ar @{->}[ddrr]^{1(2,3)} &  &\\
% &&&&\\
% &&1(2(3))&&}
%$$
\begin{center}
$$\xymatrix @-1.65pc {&& 3(2(1)) \ar @{->}[ddll]_{3(\set{1,2})} \ar @{->}[dddrr]^{\set{2,3}(1)}&& \\
 &&&&\\
3(1(2))   \ar @{->}[dd]^{\set{1,3}(2)}&&   && \\
 &&&&2(1,3) \ar @{->}[dddll]^{\set{1,2}(3)}\\
 1(3(2)) \ar @{->}[ddrr]^{1(\set{2,3})} &  &\\
 &&&&\\
 &&1(2(3))&&}$$
\end{center}
and are in bijective correspondence with the vertices and edges of Mac Lane's pentagon (cf. Section \ref{preamble-section}). 
  Let us sketch the encoding:
  \begin{itemize}
  \item $(X\otimes_1 Y)\otimes_2 (Z\otimes_3 U)$, where we annotaded the ``compositions'' $\otimes$ with the vertices of $\hyper{K}^3$, can be written $\otimes_2(\otimes_1(X,Y),\otimes_3(Z,U))$ in prefix (or tree) notation. Then we get 
 $2(1,3)$ by removing the leaf nodes of that tree.
 \item $\alpha_{X,Y,Z}\otimes_3 U$ can be interpreted as $(X\otimes_1 Y\otimes_2 Z)\otimes_3 U$ (a non fully parenthesed expression), which likewise 
 translates as $3(\set{1,2})$,where $3(\_)$ makes the job of contextualisation.
 \item Likewise, we can move from
$\alpha_{X,Y\otimes_2 Z,U}$ to $X\otimes_1(Y\otimes_2 Z)\otimes_3 U$ to $\set{1,3}(2)$, where $2$ makes the job of instantiation.
\end{itemize}

Taking the 4-dimensional associahedron $\hyper{K}^5$ (with vertex set $\set{0,1,2,3,4}$), we get the following instance in context of $\hyper{K}^3=\recrestr{\hyper{K}^5}{\set{1,2,3}}$, i.e. of Mac Lane's condition:
\begin{center}
$$\xymatrix @-1.65pc {&& 4(3(2(1(0)))) \ar @{->}[ddll]_{4(3(\set{1,2}(0)))} \ar @{->}[dddrr]^{4(\set{2,3}(1(0)))}&& \\
 &&&&\\
4(3(1(0,2)))  \ar @{->}[dd]^{4(\set{1,3}(0,2))}&&   && \\
 &&&&4(2(1(0),3)) \ar @{->}[dddll]^{4(\set{1,2}(0,3))}\\
 4(1(0,3(2))) \ar @{->}[ddrr]^{4(1(0,\set{2,3}))} &  &\\
 &&&&\\
 &&4(1(0,2(3)))&&}$$
\end{center}
We recover the (encoding of the) edge 
 $$
 \xymatrix @-2pc {&& ((((X_1\otimes_0 X_2)\otimes_1 Y)\otimes_2 Z)\otimes_3 U)\otimes_4 V \ar @{->}[dddddll]_(.6){(\alpha_{(X_1\otimes X_2),Y,Z}\otimes U)\otimes V\quad} && \\
 &&&&\\
 &&&&\\
  &&&&\\
    &&&&\\
( ((X_1\otimes_0 X_2)\otimes_1 (Y\otimes_2 Z))\otimes_3 U)\otimes_4 V  &&   && \\
\ }
$$
displayed of Section \ref{s:introduction} as the top left edge above.


\section{Recovering MacLane}

Let $\hyper{H}$ be a connected hypergraph. 
Suppose that for any connected subset $Y$, and $X \subseteq Y$, the number of connected components of~$\hyper{H}_Y\setminus X$ is less than or equal to $|X|+1$.
Then, we can consider the following variation on \cref{def:signature-hyper}.
Consider the signature $\Sigma_\hyper{H}$ made of the following data: 
\begin{itemize}
  \item Variables are any set $V$. 
  \item Function symbols are pairs of a connected subset of $H$ and one of its subsets 
  $$F:=\{(X,Y) \ | \ X \subseteq Y \subseteq H, \ \hyper{H}_Y \text{ is connected}\}.$$
  \item Sorts are either the ``variable" sort, or a connected subset of $H$, i.e. 
  $$S:=\{ X \subseteq H \ | \ \hyper{H}_X \text{ is connected}\}\cup\{*\}.$$
  \item For $(X,Y) \in F$, we define $\ari(X,Y):=|X|+1$.
  \item All variables $v \in V$ have the same output sort $\outsort(v):=*$, while function symbols $(X,Y) \in F$ have output sort $\outsort(X,Y):=Y$.
  \item For function symbols $(X,Y) \in F$ such that $\hyper{H}_Y,X \leadsto Y_1,\ldots,Y_k$, we define $\insort((X,Y),i):=Y_i$ for $1 \leq i \leq k$, and $\insort((Y,X),i)=*$ for the remaining inputs.
\end{itemize}

Closed terms of output sort $H$ are still in bijection with constructs, i.e. \cref{l:bijection-terms} holds \emph{mutatis mutandis} for this new signature. 
But now one can define rewriting rules that recover precisely MacLane in the case of the associahedra. 








\bigskip

%\emph{Acknowledgements.}   

\bibliographystyle{amsalpha}

\bibliography{Coherence2}


\end{document}



